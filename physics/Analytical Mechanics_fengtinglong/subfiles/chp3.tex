\chapter{对称性与守恒律}
\section{Largrange函数的性质}
我们已经知道将系统的Largrange函数代入Euler-Largrange方程即可得到体系的运动方程,而通过研究E-L方程还可以发现Largrange函数一些有用的性质.由$\delta S=\int_{t_1}^{t_2} (\delta q \frac{\partial L}{\partial q} + \delta \dot{q} \frac{\partial L}{\partial q})\,\mathrm{d}t $知,若$L'=L+\frac{\mathrm{d}}{\mathrm{d}t}f(q,t)$,则
\begin{equation}
\delta S=\int_{t_1}^{t_2} (\delta q \frac{\partial L}{\partial q} + \delta \dot{q} \frac{\partial L}{\partial q})\,\mathrm{d}t
+ \int_{t_1}^{t_2}\frac{\mathrm{d}}{\mathrm{d}t}\left(\delta q \frac{\partial f}{\partial q}\right)\,\mathrm{d}t
\end{equation}
第二项可化为
\begin{equation}
\int_{t_1}^{t_2}\frac{\mathrm{d}}{\mathrm{d}t}\left(\delta q \frac{\partial f}{\partial q}\right)\,\mathrm{d}t\,\mathrm{d}t
= 
\left.
\frac{\partial f}{\partial q}\delta q 
\right|_{t_1}^{t_2}
= 0
\end{equation}
故于Largrange函数而言,增加一项关于坐标和时间的函数的全导数并不影响运动方程.这将是一个有用的性质,下面正式进入对称性与守恒量部分.

\section{时间平移不变性与能量守恒}
若一个体系的Largrange函数不显含$t$,则称其具有时间平移不变性(时间平移对称性),则其关于时间的全导数
\begin{equation}
\frac{\mathrm{d}L}{\mathrm{d}t} = 
\frac{\partial L}{\partial q}\dot{q} +
\frac{\partial L}{\partial \dot{q}}\ddot{q}
\end{equation}
则
\begin{equation}
\frac{\mathrm{d}L}{\mathrm{d}t} = 
\frac{\partial L}{\partial q}\dot{q} +
\frac{\mathrm{d}}{\mathrm{d}t}\left(\frac{\partial L}{\partial \dot{q}}\dot{q}\right) - 
\left[
\frac{\mathrm{d}}{\mathrm{d}t}\left(\frac{\partial L}{\partial \dot{q}}\right)
\right]\dot{q}
\end{equation}
代入E-L方程,得
\begin{equation}
\frac{\mathrm{d}L}{\mathrm{d}t} = 
\frac{\mathrm{d}}{\mathrm{d}t}\left(\frac{\partial L}{\partial \dot{q}}\dot{q}\right) = 
\frac{\mathrm{d}}{\mathrm{d}t}\left(\frac{\partial T}{\partial \dot{q}}\dot{q}\right)
\end{equation}
由于$T$是关于$\dot{q}$的2次齐次函数,由Euler定理得
\begin{equation}
\frac{\mathrm{d}L}{\mathrm{d}t} =
\frac{\mathrm{d}}{\mathrm{d}t}(2T)
\end{equation}
故
\begin{equation}
\frac{\mathrm{d}}{\mathrm{d}t}(2T-L) = 0
\end{equation}
即
\begin{equation}
T+V = Const
\end{equation}
令$T+V=E$.称$E$为体系的能量,得到能量守恒定律.

\section{空间平移不变性与动量守恒}
若一个体系的Largrange函数不显含位置坐标$\boldsymbol{x}_i$,则称其具有空间平移不变性,则
\begin{equation}
\partial L = 
\frac{\partial L}{\partial\boldsymbol{x}_i}\delta \boldsymbol{x}_i = 0
\end{equation}
代入E-L方程,得
\begin{equation}
\partial L = 
\frac{\mathrm{d}}{\mathrm{d}t}
\left(
\frac{\partial L}{\partial \boldsymbol{v}_i}
\right)\delta\boldsymbol{x}_i = 0
\end{equation}
由于$\delta\boldsymbol{x}_i$是任意的,故
\begin{equation}
\sum_{i}\frac{\mathrm{d}}{\mathrm{d}t}
\left(
\frac{\partial L}{\partial \boldsymbol{v}_i}
\right) = 0
\end{equation}
即
\begin{equation}
\sum_{i}\frac{\partial L}{\partial \boldsymbol{v}_i} = Const= \boldsymbol{p}
\end{equation}
称$\boldsymbol{p}$为系统的动量,得到动量守恒定律.若用广义坐标$q_j$代替$\boldsymbol{x}_i$,即可得到广义动量$\sum_{j}\frac{\partial L}{\partial \boldsymbol{q}_j} = Const$

\section{空间各向同性与角动量守恒}
若将一个体系旋转一个微小角度$\delta\phi$,而其Largrange函数不发生变化,则称其具有空间各向同性.显然
\begin{equation}
\delta\boldsymbol{x} = |\delta\boldsymbol{\phi}| |\boldsymbol{x}|\sin\theta,\quad 
\theta = \langle \delta\boldsymbol{\phi},\boldsymbol{x} \rangle
\end{equation}
故
\begin{equation}
\delta\boldsymbol{x} = \delta\boldsymbol{\phi} \times \boldsymbol{x},\quad 
\delta\boldsymbol{v} = \delta\boldsymbol{\phi} \times \boldsymbol{v}
\end{equation}
由于
\begin{equation}
\delta L = \frac{\partial L}{\partial\boldsymbol{x}_i}\delta\boldsymbol{x}_i + 
\frac{\partial L}{\partial\boldsymbol{v}_i}\delta\boldsymbol{v}_i = 0
\end{equation}
故
\begin{equation}
\frac{\partial L}{\partial\boldsymbol{x}_i}(\delta\boldsymbol{\phi} \times \boldsymbol{x}_i) + 
\frac{\partial L}{\partial\boldsymbol{v}_i}(\delta\boldsymbol{\phi} \times \boldsymbol{v}_i) = 0
\end{equation}
代入E-L方程
\begin{equation}
\frac{\mathrm{d}}{\mathrm{d}t} \left(\frac{\partial L}{\partial\boldsymbol{v}_i}\right)(\delta\boldsymbol{\phi} \times \boldsymbol{x}_i) + 
\frac{\partial L}{\partial\boldsymbol{v}_i}
\left[ 
\frac{\mathrm{d}}{\mathrm{d}t}(\delta\boldsymbol{\phi} \times \boldsymbol{x}_i) - 
\frac{\mathrm{d}}{\mathrm{d}t}(\delta\boldsymbol{\phi}) \times \boldsymbol{x}_i
\right] = 0
\end{equation}
显然$\frac{\mathrm{d}}{\mathrm{d}t}(\delta\boldsymbol{\phi})=0$,故
\begin{equation}
\frac{\mathrm{d}}{\mathrm{d}t} \left(\frac{\partial L}{\partial\boldsymbol{v}_i}\right)(\delta\boldsymbol{\phi} \times \boldsymbol{x}_i) + 
\frac{\partial L}{\partial\boldsymbol{v}_i}
\left[ 
\frac{\mathrm{d}}{\mathrm{d}t}(\delta\boldsymbol{\phi} \times \boldsymbol{x}_i)
\right] = 0
\end{equation}
即
\begin{equation}
\frac{\mathrm{d}}{\mathrm{d}t} [\delta\boldsymbol{\phi}\cdot(\boldsymbol{x}_i \times \boldsymbol{p}_i)] = 
\frac{\mathrm{d}}{\mathrm{d}t} (\boldsymbol{x}_i \times \boldsymbol{p}_i)\cdot\delta\boldsymbol{\phi} = 0
\end{equation}
由于$\delta\boldsymbol{\phi}$是任意的,故
\begin{equation}
\frac{\mathrm{d}}{\mathrm{d}t} (\boldsymbol{x}_i \times \boldsymbol{p}_i) = 0
\end{equation}
得到
\begin{equation}
\boldsymbol{x}_i \times \boldsymbol{p}_i = Const = \boldsymbol{L}
\end{equation}
称$\boldsymbol{L}$为系统的角动量,得到角动量守恒定律.

\section{不同参考系中$E,\boldsymbol{p},\boldsymbol{L}$间的关系}
假设参考系$K'$相对于惯性系$K$,以$\boldsymbol{V}$运动,即
\begin{equation}
\boldsymbol{v_0} = \boldsymbol{v'} + \boldsymbol{V}
\end{equation}
则系统的动量为
\begin{equation}
\boldsymbol{p}_i = m_i \boldsymbol{v_0}_i = m_i\boldsymbol{v'}_i + \sum_{i}m_i\boldsymbol{V}
\end{equation}
若在$K'$系中系统动量为0,即$\boldsymbol{p'}=0$,只需令
\begin{equation}
\boldsymbol{V} = \frac{m_i\boldsymbol{v_0}_i}{\sum_{i}m_i}
\end{equation}
称其为质心速度$\boldsymbol{V}_c$,显然,它对时间的原函数是
\begin{equation}
\boldsymbol{x}_c = \frac{m_i\boldsymbol{x_0}_i}{\sum_{i}m_i}
\end{equation}
称其为质心坐标$\boldsymbol{x}_c$.下文称$\sum m_i$为$\mu$.对于能量$E$,有
\begin{align}
E_0 = {} & \frac{1}{2}m_i{\boldsymbol{v_0}_i}^2 + V \notag \\
= {} & \frac{1}{2}m_i{\boldsymbol{v'}_i}^2 + m_i v'_i \boldsymbol{V} + \frac{1}{2}\mu\boldsymbol{V}^2 + V
\end{align}
对于$\boldsymbol{V} = \boldsymbol{V}_c$(即质心系),有
\begin{equation}
E_0 = E' + \frac{1}{2}\mu{\boldsymbol{V}_c}^2
\end{equation}
对于角动量$\boldsymbol{L}$,有
\begin{equation}
\boldsymbol{L_0} = \boldsymbol{x_0}_i \times m_i \boldsymbol{v_0}_i
\end{equation}
在质心系中,有
\begin{equation}
\boldsymbol{L_0} = \boldsymbol{x'}_i \times m_i \boldsymbol{v_0}_i + \boldsymbol{x}_c \times m_i \boldsymbol{v_0}_i
\end{equation}
即
\begin{equation}
\boldsymbol{L_0} = m_i\boldsymbol{x'}_i \times \boldsymbol{v'}_i + \boldsymbol{x}_c \times \mu\boldsymbol{V_c}
\end{equation}
由于$\mu=\sum_{i}m_i,\ x_c=\frac{\boldsymbol{x}_i m_i}{\mu},\ V_c=\frac{m_i {v_0}_i}{\mu}$,所以
\begin{equation}
\boldsymbol{L_0} = \boldsymbol{L}' + \frac{(m_i \boldsymbol{x_0}_i)\times (m_j \boldsymbol{v_0}_j)}{\sum_{i}m_i}
\end{equation}
其中$\boldsymbol{L_0}$的第一项成为内禀角动量.

下面看转动参考系的情况.假设参考系$K$相对于$K'$以$\boldsymbol{\Omega}$的角速度转动,显然有
\begin{equation}
\boldsymbol{v}' = \boldsymbol{v} + \boldsymbol{\Omega}\times\boldsymbol{x}
\label{3.32}
\end{equation}
我们直接考虑Largrange函数的形式,先看$K'$系,
\begin{equation}
L = \frac{1}{2}m\boldsymbol{v_0}^2 - V = \frac{1}{2}m(\boldsymbol{v'}+\boldsymbol{v})^2 - V
= \frac{1}{2}m\boldsymbol{v'}^2 + m\boldsymbol{v'}\cdot\boldsymbol{V} + \frac{1}{2}m\boldsymbol{V}^2 - V
\end{equation}
由于$\boldsymbol{V}^2(t)$可看成时间的全导数,故略去,得
\begin{equation}
L = \frac{1}{2}m\boldsymbol{v'}^2 + m\frac{\mathrm{d}}{\mathrm{d}t}(\boldsymbol{x'}\cdot\boldsymbol{V}) - m\boldsymbol{x}\cdot \frac{\mathrm{d}}{\mathrm{d}t}(\boldsymbol{V}) - V
\end{equation}
即
\begin{equation}
L = \frac{1}{2}m\boldsymbol{v'}^2 - m\boldsymbol{x}\cdot\boldsymbol{\dot{V}} - V
\label{3.35}
\end{equation}
由于$\frac{\mathrm{d}}{\mathrm{d}t}\left( \frac{\partial L}{\partial\boldsymbol{v'}} \right) = 
\frac{\partial L}{\partial \boldsymbol{x'}}$,所以
\begin{equation}
\frac{\mathrm{d}}{\mathrm{d}t}(m\boldsymbol{v'}) = -m\boldsymbol{\dot{V}} - 
\frac{\partial V}{\partial\boldsymbol{x}}
\end{equation}
其中,称$-m\boldsymbol{\dot{V}}$为平动引起的惯性力.再考虑$K$系的情况,将\eqref{3.32}式代入\eqref{3.35}式,得
\begin{align}
L & = \frac{1}{2}m\boldsymbol{v}^2 + m\boldsymbol{v}\cdot(\boldsymbol{\Omega}\times\boldsymbol{x}) + \frac{1}{2}m(\boldsymbol{\Omega}\times\boldsymbol{x})^2 - m\boldsymbol{x}\cdot\boldsymbol{\dot{V}} - V \\
\frac{\mathrm{d}}{\mathrm{d}t}\left( \frac{\partial L}{\partial\boldsymbol{v}} \right) & = 
m\frac{\mathrm{d}\boldsymbol{v}}{\mathrm{d}t} + m\boldsymbol{\dot{\Omega}}\times\boldsymbol{x} + m\boldsymbol{\Omega}\times\boldsymbol{v} \\
\frac{\partial L}{\partial\boldsymbol{x}} & = m(\boldsymbol{v}\times\boldsymbol{\Omega}) + m(\boldsymbol{\Omega}\times\boldsymbol{x})\times\boldsymbol{\Omega} - m\boldsymbol{\dot{V}} - \frac{\partial V}{\partial\boldsymbol{x}}
\end{align}
整理得
\begin{equation}
m\frac{\mathrm{d}\boldsymbol{v}}{\mathrm{d}t} = -\frac{\partial V}{\partial\boldsymbol{x}} -m\boldsymbol{\dot{V}} + 
2m(\boldsymbol{v}\times\boldsymbol{\Omega}) + m(\boldsymbol{\Omega}\times\boldsymbol{x})\times\boldsymbol{\Omega} - m\boldsymbol{\dot{\Omega}}\times\boldsymbol{x}
\end{equation}
其中$2m(\boldsymbol{v}\times\boldsymbol{\Omega})$称为科里奥利力,$m(\boldsymbol{\Omega}\times\boldsymbol{x})\times\boldsymbol{\Omega}$称为惯性离心力.再考虑能量的情况:
\begin{align}
E & = \frac{\partial L}{\partial\boldsymbol{v}}\boldsymbol{v} - L\\
E & = \frac{1}{2}m\boldsymbol{v}^2 - \frac{1}{2}m(\boldsymbol{\Omega}\times\boldsymbol{x})^2 + m\boldsymbol{x}'\boldsymbol{\dot{V}} + V
\end{align}
假设$K'$相对于$K_0$无平动加速度,则
\begin{equation}
E = \frac{1}{2}m\boldsymbol{v}^2 - \frac{1}{2}m(\boldsymbol{\Omega}\times\boldsymbol{x})^2 + V
\end{equation}
称$-\frac{1}{2}m(\boldsymbol{\Omega}\times\boldsymbol{x})^2$为惯性离心力势能.代入\eqref{3.32}式,得
\begin{equation}
E = E_0 - \boldsymbol{\Omega}\cdot\boldsymbol{L}
\end{equation}

\section{力学相似性}
考虑变换$x'=\lambda_1 x,\ t'=\lambda_2 t$,假设$V(x_1,x_2,\cdots,x_N)$是$k$次齐次函数,则
\begin{equation}
V(x'_1,x'_2,\cdots,x'_N) = \lambda_1^k V(x_1,x_2,\cdots,x_N)
\end{equation}
同理,
\begin{equation}
T(v'_1,v'_2,\cdots,v'_N) = \left(\frac{\lambda_1}{\lambda_2}\right)^k 
T(v_1,v_2,\cdots,v_N)
\end{equation}
若
\begin{equation}
\lambda_2 = \lambda_1^{1-\frac{k}{2}}
\end{equation}
则Largrange函数只是整体乘一个系数,运动方程不变,一个有价值的应用是
\begin{equation}
\frac{t'}{t} = \left(\frac{x'}{x}\right)^{1-\frac{k}{2}}
\end{equation}
$k=-1$时,得到开普勒第三定律.另一个应用与能量平均值有关,由于 %笔记误为“由关”.
\begin{equation}
\frac{\partial L}{\partial \boldsymbol{v}_i}\boldsymbol{v}_i = \frac{\mathrm{d}}{\mathrm{d}t}
\left(
\frac{\partial L}{\partial \boldsymbol{v}_i}\boldsymbol{x}_i
\right)
- \frac{\partial L}{\partial \boldsymbol{x}_i}\boldsymbol{x}_i
\end{equation}
两边对时间求平均,得
\begin{equation}
2\langle T \rangle = \lim_{t \rightarrow +\infty} \left(\frac{\partial L}{\partial \boldsymbol{v}_i}\boldsymbol{x}_i\right)\frac{1}{t} + k\langle V \rangle
\end{equation}
由于$\lim\limits_{t \rightarrow +\infty} \left(\frac{\partial L}{\partial \boldsymbol{v}_i}\boldsymbol{x}_i\right)\frac{1}{t} = 0$,所以
\begin{equation}
2\langle T \rangle = k\langle V \rangle
\end{equation}
称其为位力定理.

\section*{附:齐次函数的Euler定理}

定义:$f(\lambda x_1,\lambda x_2,\dots,\lambda x_i) = \lambda^k f(x_1,x_2,\dots,x_i)$,则$f(x_1,x_2,\dots,x_i)$为$k$次齐次函数.

两边对$\lambda$求偏导,得
\begin{align}
\sum\limits_{i}\frac{\partial f(\lambda x_1,\lambda x_2,\dots,\lambda x_i)}{\partial(\lambda x_i)}
\frac{\partial(\lambda x_i)}{\partial \lambda} & = k\lambda^{k-1}f(x_1,x_2,\dots,x_i)\\
\sum_{i}\frac{\partial f(\lambda x_1,\lambda x_2,\dots,\lambda x_i)}{\partial(\lambda x_i)}\cdot
\lambda x_i & = k\lambda^{k}f(x_1,x_2,\dots,x_i)
\end{align}
令$\lambda = 1$,得
\begin{equation}
\sum\limits_{i}\frac{\partial f(x_1, x_2,\dots,x_i)}{\partial x_i}\cdot
\lambda x_i = kf(x_1,x_2,\dots,x_i)
\end{equation}