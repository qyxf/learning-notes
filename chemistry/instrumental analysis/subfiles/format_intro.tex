\chapter*{排版格式介绍}
\begin{introduction}
	\item 概念及定义
	\item 公式
	\item 笔记
	\item 例子
\end{introduction}

以上是内容提要,大致概括本章的考点。

\section*{几种环境}
我们将把每一个大问题作为一个section。

排版过程中的定义我们使用如下格式:

\begin{definition*}{概念及定义}{definition}
	
\end{definition*}

或者直接使用
\begin{emptytcb*}{原理}{principle}
	
\end{emptytcb*}

一些比较重要的公式我们使用如下格式
\begin{theorem*}{公式/定理}{theo}
	
\end{theorem*}

因为保留了老师提供的提纲的顺序,我们并没有编号,否则会有点乱。

\begin{note}
需要补充、解释的我们加在这里
\end{note}

\begin{example}
	举例、例题我们放在这里
\end{example}