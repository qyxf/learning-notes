\chapter{原子光谱}

\begin{introduction}
	\item 原子发射光谱的分析方法
	\item 原子发射光谱的仪器
	\item 原子吸收光谱分析
\end{introduction}


%设计方案:
%1.定义、概念、原理,做个环境
%2.公式、符号说明环境
%3.步骤、结构、小点:列表
%4.对比:表格

\section{原子发射光谱的分析方法}
%1、原子发射光谱定性分析:不同元素的原子能级结构不同,因此能级跃迁所产生的谱线具有不同的波长特征。根据谱线特征可以进行发射光谱定性分析。什么是共振线?什么是自吸收?原子发射光谱定量分析:罗马金-赛伯公式

不同元素的原子能级结构不同,因此能级跃迁所产生的谱线具有不同的波长特征。每一种元素的原子都有其特征光谱。
\subsection{定性分析}

\begin{definition*}{共振线(特征谱线)}{}
	元素由基态到第一激发态的跃迁对应的谱线称为共振线。
	\begin{itemize}
		\item 这种跃迁最容易发生,需要的能量最低,产生的谱线也最强。
	\end{itemize}
\end{definition*}

\begin{definition*}{灵敏线}{}
	元素特征谱线中强度较大的称为元素的灵敏线。

\begin{itemize}
	\item 如果在光谱中检出了某元素的灵敏线,可以确证试样中存在该元素,但是至少要有两条灵敏线出现,才可以确认该元素的存在。
	\item 如果未检出灵敏线,说明试样中不存在备件元素或元素含量在灵敏度以下。
\end{itemize}
\end{definition*}

\subsection{定量分析}

\begin{theorem*}{罗马金-赛伯公式}{}
	\begin{gather*}
		I=A\cdot c^b\\
		\lg I = b \lg c + \lg A
	\end{gather*}
	
	$I$是两个能级之间的谱线强度;$A$代表两个能级间每个原子单位时间内发生$A$次跃迁(即跃迁几率);$c$是样品中分析物的浓度;$b$是自吸系数,随浓度$c$增加而减小,当浓度很小而无自吸时,$b=1$。
\end{theorem*}


一般采用内标法、标准加入法进行分析。

\begin{definition*}{自吸收}{}
	光源等离子体中心部位原子发射的光子通过温度较低的外层时,被外层基态原子吸收的现象。
\end{definition*}

\section{原子发射光谱的仪器}
%2、原子发射光谱仪的光源作用?有哪些类型?

原子发射光谱的光源称为激发光源。
\begin{itemize}
	\item 作用:\textbf{提供试样中被测元素蒸发、原子化和原子激发发光所需要的的能量}。
	\item 要求:灵敏度高、重现性好、光谱背景小,结构简单、操作安全。
	\item 类型:火焰光源、电弧光源、高压电容火花光源、辉光放电光源、电感耦合高频等离子体光源(ICP光源)
\end{itemize}

\subsection{直流电弧光源}
\begin{itemize}
	\item 电机温度高,弧焰中心温度为$5000\sim 7000\mathrm{K}$,有利于试样的蒸发;
	\item 除石墨电极产生的氰带光谱外,背景比较浅;
	\item 电弧在电极表面无常游动,且有分馏效应\footnote{不同物质因沸点不同而导致蒸发速度不同},重现性比较差;
	\item 谱线容易发生自吸收现象。
	\item 常用于定性分析以及矿石、矿物难熔物质中痕量组分的定量测定。
\end{itemize}

\begin{figure}[!h]
	\centering
	\includegraphics[width=0.6\linewidth]{image/chp8_circuit_diagram1.eps}
%	\caption{}
	\label{fig:chp8circuitdiagram1}
\end{figure}


\subsection{低压交流电弧光源}

\begin{itemize}
	\item 电极温度较直流电弧略低;
	\item 因电弧弧温较高,灵敏度比直流电弧高;
	\item 弧焰稳定,适于定量分析。
\end{itemize}

\begin{figure}[!h]
	\centering
	\includegraphics[width=0.7\linewidth]{image/chp8_circuit_diagram2}
%	\caption{}
	\label{fig:chp8circuitdiagram2}
\end{figure}

\subsection{高压电容火花光源}
\begin{itemize}
	\item 火花作用于电极的面积小,时间短,电极温度低,不适于难蒸发的物质;
	\item 火花放电的能量高,能激发电位很高的原子线或离子线;
	\item 稳定性好,适于定量分析;
	\item 电极面积小,适于微区分析。
\end{itemize}

\begin{figure}[!h]
	\centering
	\includegraphics[width=0.7\linewidth]{image/chp8_circuit_diagram3}
	\caption{(a)稳定间隙控制的火花电路;(b)旋转间隙控制的火花电路\\
		E:电源;R:可变电阻;T:升压变压器;D:扼流圈;C:可变电容;L:可变电感;L’:高阻抗自感线圈;G:分析间隙;G’:控制间隙;G1,G2:断续控制间隙;M:同步电机带动的断续器}
	\label{fig:chp8circuitdiagram3}
\end{figure}

电弧和火花光源适用于固体样品分析,但温度低,基体影响严重,需要寻找更高蒸发、原子化和激发的光源。

\begin{note}
	基体效应:指试样组成对谱线强度的影响,主要发生在试样的蒸发和激发过程中。试样中占大多数的物质的沸点高低决定蒸发温度的高低;主体成分的电离电位越高,光源激发温度越高,影响谱线强度。不同蒸发顺序也影响谱线强度。
\end{note}

\subsection{辉光放电光源}
Grimm辉光放电管用于固体样品表面分析,能检测$\ce{B,C,Si,P,S}$等元素。

\begin{figure}[!h]
	\centering
	\includegraphics[width=0.55\linewidth]{image/chp8_huiguang}
%	\caption{}
	\label{fig:chp8huiguang}
\end{figure}


\subsection{ICP光源(inductively coupled plasma)}
\begin{itemize}
	\item 组成:高频发生器和感应圈、等离子炬管和供气系统、试样引入系统;
	\item 优势:
	\begin{itemize}
		\item 具有趋肤效应\footnote{涡流主要集中在等离子体的表面层,气溶胶从中心通道进入},自吸效应小;
		\item 温度高,基底成分被分解,减小基底效应;
		\item 不需要电极,无电极污染、加热方式具有良好稳定性;
		\item 电子密度高,电离干扰可不考虑。
	\end{itemize}
	\item 缺点:固体进样较困难,对气体和非金属灵敏度低;雾化效率低;设备和维持费高。
\end{itemize}

\begin{figure}[!h]
	\centering
	\includegraphics[width=0.7\linewidth]{image/chp8_ICP}
%	\caption{}
	\label{fig:chp8icp}
\end{figure}

\begin{itemize}
	\item 外管:通冷却气$\ce{Ar}$使等离子体离开外层石英管内壁,避免它烧毁石英管。
	\item 中层石英管:出口做成喇叭形,通入$\ce{Ar}$气维持等离子体作用。
	\item 内层石英管:把载气载带试样气溶胶(由气动或超声雾化器产生)注入等离子体内。
	\item 内焰区:(测光区)分析物原子化、激发、电离与辐射的主要区域。
	\item 焰心区:(预热区)等离子体与高频感应线圈耦合获得能量的区域;试样气溶胶被预热、挥发溶剂、蒸发溶质。
	\item 尾焰区:温度低,只能激发低能级的谱线。
\end{itemize}


\section{原子吸收光谱}
%3、原子吸收光谱定量分析?原子吸收光谱仪的结构组成?光源——空心阴极灯。

\subsection{原子吸收光谱定量分析}
原子吸收光谱产生于基态原子对特征谱线的吸收。

实验条件一定时,基本关系式可以简写为$A=kc$,即吸光度和(质量体积)浓度成正比。

原子吸收光谱轮廓图如图\ref{fig:chp8absorption},以原子吸收谱线的中心波长和半宽度来表征。
\textit{中心波长}由原子能级决定;\textit{半宽度}指在中心波长附近,极大吸收系数一半处,吸收光谱线轮廓上两点之间的频率差或波长差。

\begin{figure}[!h]
	\centering
	\includegraphics[width=0.4\linewidth]{image/chp8_absorption}
	\caption{}
	\label{fig:chp8absorption}
\end{figure}

原子光谱分析的优点是:
\begin{itemize}
	\item 检出限低,灵敏度高。检出限最低可达$10^{-10}\sim 10^{-14}\mathrm{g}$;
	\item 测量精度好;
	\item 分析速度快;
	\item 应用范围广,可测定金属元素,也可通过间接原子吸收法测非金属和有机化合物等;
	\item 仪器简单,操作方便;
\end{itemize}

\subsection{原子吸收光谱仪的结构}

如图\ref{fig:chp8instrument}
\begin{figure}[!h]
	\centering
	\includegraphics[width=0.7\linewidth]{image/chp8_instrument}
	\caption{原子吸收光谱仪}
	\label{fig:chp8instrument}
\end{figure}

\subsubsection{光源}
空心阴极灯是最理想、应用最广的光源,用来发射被测元素的特征共振辐射。它满足对光源的各项基本要求:
\begin{itemize}
	\item 发射的共振辐射的半宽度要明显小于吸收线的半宽度;
	\item 辐射强度大;
	\item 背景低,低于特征共辐射强度的1\%;
	\item 稳定性好;
	\item 使用寿命长于5A·h
\end{itemize}

工作原理:
\begin{itemize}
	\item 由一个由被测元素材料制成的空心阴极和一个由钛、锆、钽或其他材料制作的阳极。
	\item 玻璃管内由260$\sim$1300Pa的惰性气体氖或氩来载带电流,使阴极产生溅射及激发原子发射特征的锐线光谱。
	\item 云母屏蔽片来使放电限制在阴极腔内,同时使阴极定位。
	\item 采用低压辉光放电,集中于阴极空腔内。
	\item 光源调制:使用脉冲供电方式来改善放电特性,同时便于使有用的原子吸收信号与原子化器的滞留发射信号区分开。
\end{itemize}

\begin{figure}[!h]
	\centering
	\includegraphics[width=0.4\linewidth]{image/chp8_ICP_lamp}
	\caption{}
	\label{fig:chp8icplamp}
\end{figure}

\subsubsection{原子化器}
作用:提供能量,使试样干燥、蒸发和原子化。

有三种原子化的方法:

\begin{itemize}
	\item 火焰原子化法
	\begin{itemize}
		\item 采用最多的是乙炔-空气火焰,其燃烧稳定,重现性好,噪声低,燃烧速度不太高,温度足够高,对大多数元素有足够灵敏度。
		\item 氢-空气火焰是氧化性火焰,燃烧速度高于乙炔-空气,优点是背景较弱,透射性好,但温度较低。
		\item 乙炔-氧化亚氮高温火焰温度在三者中最高,燃烧速度不快,可测定70多种元素。
	\end{itemize}
	\item 非火焰原子化法
	常用管式石墨炉。
	\begin{itemize}
		\item 优点是试样原子化在惰性气体和强还原性介质进行,利于氧化物分解和自由原子生成;用样量小,样品利用率高,绝对灵敏度高;固、液试样均可直接进样。
		\item 缺点是试样组成不均匀性影响较大,背景吸收强,精密度不如火焰原子化法。
	\end{itemize}
	\item 低温原子化法
		利于某些元素(如$\ce{Hg}$)单质或其氢化物(如$\ce{AsH3}$)在低温下的易挥发性,将其导入气体流动吸收池反应出单质或氢化物后,进行原子化。目前通过该方法测定$\ce{Hg, As, Sb, Se, Sn, Bi, Ge, Pb, Te}$等。
\end{itemize}

\subsubsection{分光器}
将所需要的共振吸收线分离出来,由入射和出射狭缝、反射镜和色散元件(光栅)组成。

\subsubsection{检测系统}
\begin{itemize}
	\item 光电倍增管
	
		光电倍增管的外壳由玻璃或石英制成,内部抽真空,阴极涂有能发射电子的光敏物质,如$\ce{Sb-Cs}$或$\ce{Ag-O-Cs}$等,在阴极C和阳极A间装有一系列次级电子发射极,即电子倍增极D1、D2 … 等。阴极C和阳极A之间加有约1000V的直流电压,当辐射光子撞击光阴极C时发射光电子,该光电子被电场加速落在第一被增极D1上,撞击出更多的二次电子,依次类推,阳极最后收集到的电子数将是阴极发出的电子数的$10^5\sim 10^8$倍。
		\begin{figure}[!h]
			\centering
			\includegraphics[width=0.7\linewidth]{image/chp8_guangdianbeizengguan}
			\caption{光电倍增管}
			\label{fig:chp8guangdianbeizengguan}
		\end{figure}
	\item CCD
	
		是一种电荷耦合器件,是在大规模硅集成电路工艺基础上研制而成的模拟集成电路芯片。由于其输人面空域上逐点紧密排布着对光信号敏感的像元,因此它对光信号的积分与感光板的情形颇相似。
		但是,它可以借助必要的光学和电路系统,将光谱信息进行光电转换、储存和传输,在其输出端产生波长-强度二维信号,信号经放大和计算机处理后在末端显示器上同步显示出人眼可见的图谱。
		
		优点是速度快、动态响应范围和灵敏度均可能达到或超过光电倍增管,且性能稳定、体积小、耐用。
\end{itemize}

