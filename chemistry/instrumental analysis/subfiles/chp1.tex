\chapter{色谱}

\begin{introduction}
	\item 基本概念与术语(理解)
	\item 塔板理论和速率理论(掌握)
	\item 气相色谱仪的结构、检测器特点、检测方法(掌握)
	\item 高效液相色谱法的特点(了解)、主要部件及分析流程(熟悉)
	\item 几种色谱法的应用特点(熟悉)
	\item 色谱技术在生物分析中的应用(了解)
	\item 毛细管电动色谱、毛细管电泳(了解)
\end{introduction}


%设计方案:
%1.定义、概念、原理,做个环境
%2.公式、符号说明环境
%3.步骤、结构、小点:列表
%4.对比:表格



\section{色谱基本概念与术语}
%1、理解基本概念与术语:
%色谱流出曲线 (chromatogram):指样品注入色谱柱后,信号随时间变化的曲线。

\subsection{分类}
色谱(Chromatography) 法是一种重要的分离分析方法,它是根据组分在两相中作用能力不同而达到分离目的的。
\begin{itemize}
	\item 按流动相分:气相色谱(GC)、液相色谱(LC)、超临界流体色谱(SFC);
	\item 按机理分:吸附色谱、分配色谱、离子交换色谱、排阻(凝胶)色谱等等;
	\item 按固定相在支持体中的形状分:柱色谱、平板色谱(纸色谱 薄层色谱);
	\item 按按分离效率分:经典液相色谱和高效液相色谱(HPLC)。
\end{itemize}

研究核心:选择最合适的色谱体系和条件,在最短的时间内达到最佳的分离效果。

\subsection{概念}

\begin{enumerate}
	\item 色谱流出曲线 (chromatogram):指样品注入色谱柱后,信号随时间变化的曲线(一般为高斯分布曲线)。
	\item 基线:无组分通过色谱柱时,检测器的噪声随时间变化的曲线。
	\item 峰宽:峰底宽$W_b$、峰半宽$W_{1/2}$、标准偏差$\sigma$的关系是(如图\ref{fig:chp1peak})
	\begin{theorem*}{峰宽的关系}{}
		\begin{gather*}
		W_b=4\sigma\\
		W_{1/2}=2\sqrt{2\ln 2}\sigma
		\end{gather*}
	\end{theorem*}
	\begin{figure}[!h]
		\centering
		\includegraphics[width=0.55\linewidth]{image/chp1_peak}
		\caption{色谱峰示意图}
		\label{fig:chp1peak}
	\end{figure}
	\item 保留值
	\begin{itemize}
		\item 保留时间\footnote{保留:retention}$t_R$:进样到出现色谱峰的时间。
		\item 保留体积$V_R$:进样到出现色谱峰时消耗的流动相体积。
		
		和保留时间的关系:$V=t\times F$,$F$是流动相线速度。
		\item 死时间$t_0$($t_M$):流动相流过色谱柱的时间。(另一种理解是不被固定相吸附的组分如甲烷、空气的保留时间。)
		\item 死体积$V_0$($V_M$):色谱柱的空隙体积。
		\item 校正保留时间:$t'_R=t_R- t_0$。
		\item 校正保留体积:$V'_R=V_R- V_0$,
	\end{itemize}
		
	校正:死时间/死体积反映柱和仪器系统的几何特性,与被测组分性质无关,故通过校正来更好地反应被测组分的保留特性。
	\item 相对保留值
	
	某一组分1的校对保留值和标准物2的校对保留值之比,称为组分1对2的相对保留值。相对保留值仅随柱温及固定相变化。
	\begin{theorem*}{相对保留值}{}
		\begin{equation*}
			r_{1,2}=\dfrac{t'_{R_1}}{t'_{R_2}}=\dfrac{V'_{R_1}}{V'_{R_2}}
		\end{equation*}
	\end{theorem*}
	\item 分配系数与分配比(容量因子)
	
	分配系数$K$:一定$T$、$p$,两相达平衡后,组分在固定相和流动相\footnote{流动相:moving phase;固定相:stationary phase}质量体积浓度的比值。	
	\begin{equation*}
		K=\dfrac{C_s}{C_m}
	\end{equation*}
	
	分配比$k$:一定$T$、$p$,两相达平衡后,组分在固定相质量($p$)和流动相质量($q$)的比值。
	\begin{equation*}
	k=\dfrac{p}{q}
	\end{equation*}
	
	$K$与$k$的关系\footnote{都是对于单个物质而言}:
	\begin{equation*}
	K=\dfrac{C_s}{C_m} =\dfrac{p/ V_s}{q/ V_M}=k \dfrac{V_0}{V_s}
	\end{equation*}
	\begin{theorem*}{$k$与保留值的关系}{}
	\begin{gather*}
		k=\dfrac{t'_R}{t_0} =\dfrac{t_R-t_0}{t_0}\\
		t_R=t_0 (1+k)		
	\end{gather*}
	\end{theorem*}
	
	\item 分离效能的指标
	\begin{itemize}
		\item 选择性(相对保留值):相对保留值越大,选择性越好。仅由两组分热力学性质决定,与色谱柱无关。
		\item 峰宽度
		\item 分离度:考虑了保留时间和峰宽度,是一个综合指标。
		\begin{theorem*}{分离度}{}
			\begin{equation}
			R=\dfrac{t_{R_2}-t_{R_1}}{(W_{b_2}+W_{b_1})/2}\label{eqn:1.1}
			\end{equation}
			这是最原始的表达式。
		\end{theorem*}
		
				
		\begin{itemize}
			\item $R<1.0$,两峰明显重叠;
			\item $R=1.0$,两峰分离度达$97.7\%$;
			\item $R\geqslant 1.5$,两峰完全分开。
		\end{itemize}
	\end{itemize}
\end{enumerate}

\section{色谱两大理论}
%2.色谱两大理论—塔板理论和速率理论。塔板理论描述色谱流出曲线的方程,并通过这一方程各参数来研究影响分离的因素,要会计算。速率理论(van Deemter方程)是描述提高柱效的途径,理解范第姆特方程中各参数的含义?

\subsection{塔板理论}
\begin{itemize}
	\item 目的:从理论上得出描述色谱流出曲线的方程,并通过这一方程各参数来研究影响分离的因素。
	\item 两大假设:色谱柱存在多级“塔板”,每级塔板包含一个流动相和固定相,各自含一定量的各种组分;每种组分通过时,\textbf{在每级塔板处,两相间达到一次平衡}。
\end{itemize}

简单推导如下:

设某一组分的分配比$k=\frac{p}{q}$,经过一次转移(流到下一个塔板)后,0级塔板上组分的百分含量为$p'=\frac{p}{q+p}=\frac{k}{1+k}$,1级塔板上组分的百分含量为$q'=\frac{q}{q+p}=\frac{1}{1+k}$。经多次转移,组分在各级塔板的百分含量将符合二项分布,即${(p'+q')^N}$,$N$为转移次数。任一级塔板$r$对应的百分含量为:

\begin{equation*}
	f_{N,r}=\dfrac{N!}{r!(N-r)!} {(\dfrac{1}{1+k})}^r {(\dfrac{k}{1+k})}^{N-r}
\end{equation*}
以此作图即得流出曲线。

$N$特别大时,将呈正态分布:
\begin{equation*}
	C=C_{max}\times \mathrm{exp}\{-\dfrac{n(t_R-t)^2}{2t_R^2}\}
\end{equation*}

$C_{max}$其实就是峰高。对比标准正态分布曲线得:
\begin{equation*}
	{\sigma}^2=\dfrac{t_R^2}{n}
\end{equation*}

由$W_b$,$W_{1/2}$与$\sigma$的关系,得理论塔板数$n$的计算方法:

\begin{theorem*}{塔板数计算公式}{}
	\begin{equation}
	n={\left(\dfrac{t_R}{\sigma}\right)}^2=16{\left(\dfrac{t_R}{W_b}\right)}^2=5.54{\left(\dfrac{t_R}{W_{1/2}}\right)}^2\label{eqn:1.2}
	\end{equation}
	注意$t_R$是热力学常数,即样品、热力学条件不变时为定值。
	
	有效塔板数$n_{eff}$为:
	\begin{equation*}
	n_{eff}={\left(\dfrac{t'_R}{\sigma}\right)}^2=16{\left(\dfrac{t'_R}{W_b}\right)}^2=5.54{\left(\dfrac{t'_R}{W_{1/2}}\right)}^2
	\end{equation*}
	
	$n$与$n_{eff}$的关系为:
	\begin{equation}
		\dfrac{n_{eff}}{n}={\left(\dfrac{k}{k+1}\right)}^2
	\end{equation}
	固定物质时,二者成正比。
\end{theorem*}

小结:
$H=L/n$为塔板高度,$L$为色谱柱长,其他条件不变只是增长色谱柱时,$H$不变。

影响色谱柱效率的是$H$($n$),$H$越小,$n$越大,色谱峰越窄,分离效率越好。

从理论上可以通过该方程预测具有不同分配系数$K$的两种物质在塔板数为$n$的色谱柱上分离的情况。其中色谱分离不限于液-固相,也可是气-液相。

塔板理论的缺陷:半经验性理论;忽略了纵向扩散的影响;假设不可能完全实现;无法给出影响塔板高度的因素等。

\subsection{速率理论}

\begin{theorem*}{Van Deemter方程}{}
	\begin{equation*}
		H=A+\dfrac{B}{u}+Cu
	\end{equation*}
	$u$是流动相线速度。
\end{theorem*}

\begin{enumerate}
	\item $A$是涡流扩散项:固定相填充不均匀引起的峰展宽,与颗粒直径正相关。使用较细粒度和颗粒均匀的填料,并尽量填充均匀,可减少涡流扩散,提高柱效。对于空心毛细管柱,$A$项为$0$。
	\begin{equation*}
		A=2\lambda d_p
	\end{equation*}
	$\lambda$:填充的不规则因子;$d_p$:固定相颗粒粒径
	\item $B/u$是纵向分子扩散项:由浓度差引起,分子延纵向扩散形成的展宽。由于组分在液相中扩散系数很低,因此液相色谱中可忽略$B$。
	\begin{equation*}
		B=2rD_m
	\end{equation*}
	$r$:弯曲因子(填充柱$r<1$,空心柱$r=1$),$D_m$:组分在流动相的扩散系数
	\item $Cu$是传质阻力\footnote{传质:溶解、扩散、转移的过程。传质阻力:影响传质过程的阻力。}项:组分在流动相和固定相之间传质的阻力。在非平衡状态下使有些分子较快向前移动,而另一些滞后,引起峰展宽。
	\begin{equation*}
		C=q\dfrac{k}{(1+k)^2}\dfrac{d_f^2}{D_s}+\omega\dfrac{k}{(1+k)^2}\dfrac{d_p^2}{D_m}
	\end{equation*}
	两项分别为固定相和流动相传质阻力。
\end{enumerate}

$H$对$u$求导可推出:
\begin{gather*}
	H_{min}=A+2\sqrt{BC}\\
	u_{opt}=\sqrt{\dfrac{B}{C}}
\end{gather*}

\subsection{分离条件的选择}
联立上述公式\footnote{这里近似$W_{b_1}\approx W_{b_2}$,代入式\eqref{eqn:1.1},\eqref{eqn:1.2}即可推得。需注意,$n_{eff}$与被分离的组分无关,只和色谱柱、条件有关,此时代入$n_{eff}=16{\left(\dfrac{t'_R}{W}\right)}^2$对于不同组分会算出不同的结果,这是前一步近似导致的。}可得:

\begin{theorem*}{色谱分离的基本方程}{}
	\begin{equation}
	R=\dfrac{(r_{2,1}-1)k}{4r_{2,1} (1+k)} \sqrt{n}
	\end{equation}
	
	可以变形为
	\begin{equation}
	n_{eff}=16R^2\dfrac{r_{2,1}^2}{(r_{2,1}-1)^2}
	\end{equation}
\end{theorem*}

备考时,需要大家理解两个理论中公式的含义,并掌握计算的方法。
\vspace{3pt}
\begin{example}
	有一根$1$m长的柱子,分离组份1和2,色谱图数据为:$t_M=5$s,$t_1=45$s,$t_2=49$s,$W_1=W_2=5$s。若欲得到$R=1.2$的分离度,有效塔板数应为多少?色谱柱要加长到多长?
	
	\solve 
	首先计算相对保留值$$r_{2,1}=\dfrac{t_2-t_M}{t_1-t_M}=1.1$$
	
	$R=1.2$时对应的塔板数:$$n'=16R^2\dfrac{r_{2,1}^2}{(r_{2,1}-1)^2}=2788$$
	
	计算1m长柱子中有效塔板数$$n=16{\left(\dfrac{t_2-t_M}{W_2}\right)}^2=1239$$
	
	所以$$L=\dfrac{n'}{n}\times 1=2.25(\mathrm{m})$$
	
	%注意:此处用组分2计算塔板数是由两公式联立的变形结果,过程比较简单,可尝试自己推导。
\end{example}


\section{气相色谱仪}
%3、掌握气相色谱仪的结构组成,气相色谱常用的检测器及其特点(什么物质用什么类型的检测器)?什么是担体?气相色谱分离物质,用极性或者非极性固定液,物质流出的先后?
\subsection{气相色谱仪的结构组成}

\begin{figure}[!h]
	\centering
	\includegraphics[width=0.7\linewidth]{image/chp1_GC_apa}
	\caption{气相色谱仪}
	\label{fig:chp1gcapa}
\end{figure}
\begin{enumerate}
	\item 载气系统:要求纯净(净化器)、稳定(稳压阀或双路气)。
	
	常用$\ce{H2,N2,He}$。
	\item 进样系统:进样装置和汽化室。
	
	进样通常用微量注射器和进样阀将样品引入。液体样品引入后需要瞬间汽化,汽化在汽化室进行。对汽化室有如下要求:体积小、热容大、对样品无催化作用。
	
	对高分子样品,则采用裂解装置(管式炉、热丝、居里点裂解器等)。
	\item 分离系统:色谱柱和固定相。
	
	色谱柱包括填充柱和毛细管柱。毛细管柱较细长。
	
	固定相有固体固定相和液体固定相。固体固定相是固体吸附剂。液体固定相由担体和固定液组成。
	
	\begin{definition*}{担体}{}
		担体是一种多孔的、化学惰性的固体颗粒,可以提供较大表面积的惰性表面以\textbf{承担固定液}。
	\end{definition*}
	\item 控温系统:控制恒温或程序升温。
	
	$K$是热力学常数,温度越高,$K$值越小,保留时间越短。因此可通过柱温调节分离程度。
	\item 检测器:将分离后各组分的量转变为电信号并记录。
	
	要求灵敏度高、线性范围宽、响应速度快、结构简单、通用性强。
\end{enumerate}

\subsection{气相色谱常用的检测器}
检测器的性能指标:
\begin{itemize}
	\item 灵敏度$S$:样品量变化引起信号变化程度越大,灵敏度越高。$S=\dfrac{\Delta R}{\Delta  Q}$,$R$:峰高或面积;$Q$:浓度或质量
	\item 检测限:三倍噪音相当的物质的量称为检测限。$D=\dfrac{3N}{S}$,$N$为噪音,单位为$\mathrm{mV}$
	\item 线性范围:指检测器信号与样品浓度之间成正比关系的范围。
\end{itemize}

常用检测器:
\begin{enumerate}
	\item 热导检测器(TCD)
	\begin{itemize}
		\item 原理:基于各物质热导系数的不同
		\item 特点:结构简单;灵敏度不高
		\item 检测物质:对所有物质都有响应(无机物和有机物)
	\end{itemize}
	\item 氢火焰离子化检测器(FID)
	\begin{itemize}
		\item 原理:有机物在火焰中电离形成离子流,根据离子流的出现和大小进行分析。
		\item 特点:
		\begin{itemize}
			\item 灵敏度高($10^{-12}\mathrm{g/s}$),线性范围宽
			\item 不能检测惰性气体、空气、$\ce{H2O,CO,CO2,NO,SO2,H2S}$等。
		\end{itemize}
		\item 检测物质:适于有机物的检测
	\end{itemize}
	\item 电子俘获检测器(ECD)
	\begin{itemize}
		\item 原理:载气在$\beta$-射线源下电离形成稳定的基流,卤素、$\ce{S,P,O,N}$等电负性高的原子捕获电子形成负离子并与载气正离子结合,使基流信号下降,据此检测组分。
		\item 特点:
		\begin{itemize}
			\item 对卤素、$\ce{S,P,O,N}$有很强的响应
			\item 灵敏度高,可用于痕量农药残留的分析
			\item 线性范围较窄
		\end{itemize}
		\item 检测物质:含卤素、$\ce{S,P,O,N}$等电负性较强原子的物质
	\end{itemize}
	\item 火焰光度检测器(FPD)
	\begin{itemize}
		\item 原理:S、P在燃烧中被激发,从而发生特征的光信号(S-394nm,P-526nm)	
		\item 检测物质:含硫、磷的化合物
	\end{itemize}
\end{enumerate}

\subsection{气相色谱的分离}
极性原则(选择固定液):
\begin{itemize}
	\item 非极性组分分离:用非极性固定液,出峰顺序由蒸汽压决定,沸点高的保留时间长。
	\item 中等极性组分分离:用中等极性固定相,沸点与分子间力同时起作用。
	\item 强极性组分分离:用强极性固定相,分子间力起作用,按极性大小出峰,极性小的先出峰。
	\item 极性和非极性分离:用极性固定相,非极性先出峰。
	\item 能形成氢键的试样:选择极性或氢键型固定液,不易形成氢键的后出峰。
\end{itemize}

如何判断极性:

\begin{example}
选取两种分析对象A,B:$\beta,\beta’$-氧二丙氰、角沙烷,以待测固定液为固定液制成色谱柱,求三种固定液中的:
\begin{equation*}
	q=\lg⁡\dfrac{t_R (A)}{t_R (B)}
\end{equation*}
则该固定液的相对极性$P_x$为
\begin{equation*}
	P_x=100-100 \dfrac{(q_{\beta\beta}-q_x)}{(q_{\beta\beta}-q_j)}
\end{equation*}
\end{example}

\noindent 小练习:
\begin{example}
	已知在柱温为50\textcelsius 和其他给定条件下,测定$t_M=0.42$min。用环己烷与苯在$\beta,\beta’$-氧二丙氰柱上测得$q_1=1.0086$,在角鲨烷上测得$q_2=0.179$,在癸二酸壬酯柱上测的$t_R(\text{环己烷})=4.22$min,$t_R(\text{苯})=6.22$min,计算癸二酸壬酯的相对极性。
\end{example}

\section{高效液相色谱法}
%3.了解高效液相色谱法的特点,熟悉高效液相色谱仪的主要部件及分析流程,理解液相色谱流动相的选择?

气相色谱只适合分析较易挥发、且化学性质稳定的有机化合物,而高效液相色谱法(High Performance Liquid Chromatography, HPLC)则适合于分析那些用气相色谱难以分析的物质,如挥发性差、极性强、具有生物活性、热稳定性差的物质。

特点:
\begin{itemize}
	\item 色谱柱可反复使用,流动相可选择范围宽,流出组分容易收集;
	\item 分离效率高,灵敏度高;
	\item 操作自动化,应用范围广。
\end{itemize}

\subsection{主要部件}
\begin{enumerate}
	\item 输液系统
	\begin{itemize}
		\item 高压输液泵:以稳定的流速或压力将流动相输送到色谱系统。
		\item 在线脱气装置:也使用超声、真空等脱气方式。脱气的目的是去除气泡,保证流动相流速稳定,减小噪音。
		\item 梯度洗脱装置:通过两个输液泵流速的变化,改变流动相洗脱能力,作用与气相色谱的程序升温类似。
	\end{itemize}
	\item 进样系统
	
	通常采用六通阀。
	\item 色谱柱
	
	是核心部件。要求柱效高、柱容量大、性能稳定。柱性能与柱结构、填料特性、填充质量和使用条件有关。
	%填料补充
	\item 检测器
	
	连续监测流出物的组成和含量变化的装置。
	\begin{itemize}
		\item 紫外-可见检测器
		\item 荧光检测器:灵敏度高,选择性好,适用于药物、生化样品的分析。
		\item 蒸发光散射检测器:适用于无紫外吸收、无电活性、不发荧光的样品的检测。
		\item 电化学检测器
		\item 质谱
	\end{itemize}
	\item 数据处理
\end{enumerate}

\subsection{分析流程}
由泵将储液瓶中的溶剂吸入色谱系统,然后输出,经流量与压力测量之后,导入进样器。被测物由进样器注入,并随流动相通过色谱柱,在柱上进行分离后进入检测器,检测信号由数据处理设备采集与处理,并记录色谱图。废液流入废液瓶。遇到复杂的混合物分离(极性范围比较宽)还可用梯度控制器作梯度洗脱。这和气相色谱的程序升温类似,不同的是气相色谱改变温度,而HPLC改变的是流动相极性,使样品各组分在最佳条件下得以分离。

\subsection{流动相的选择}
\begin{itemize}
	\item 对样品有一定溶解度;
	\item 适用于选用的检测器,如用紫外检测时,不能选择对紫外光有吸收的溶剂;
	\item 化学惰性好,液液色谱中不能与固定相互溶,硅胶吸附剂不能用碱性溶剂,氧化铝吸附剂不能用酸性溶剂。
	\item 黏度低。黏度太大会降低样品的扩散系数,导致传质减慢,柱效降低,同时柱压也会升高。
	\item 高纯度。宜采用色谱纯试剂,否则会导致噪音增加,干扰定性、定量。
	\item 安全低毒,环境友好。
\end{itemize}

\section{几种色谱法}
%4.熟悉吸附色谱法、分配色谱法、离子交换色谱法和体积排除色谱法的应用特点,选择分离类型的原则,了解色谱技术在生物分析中的应用。
\subsection{吸附色谱法(absorption chromatography)}

\begin{itemize}
	\item 原理:各组分在固定相表面的吸附作用不同;
	\item 固定相:活性硅胶、氧化铝、活性炭、聚乙烯、聚酰胺等固体吸附剂,所以吸附色谱也称液固吸附色谱。活性硅胶最常用;
	\item 流动相:弱极性有机溶剂或非极性溶剂与极性溶剂的混合物,如正构烷烃(己烷、戊烷、庚烷等)、二氯甲烷/甲醇、乙酸乙酯/乙腈等;
	\item 应用特点:用于结构异构体分离和族分离。如农药异构体分离、石油中烷、烯、芳烃的分离。缺点是易产生不对称峰和拖尾现象。
\end{itemize}

\subsection{分配色谱法}

\begin{itemize}
	\item 原理:样品分子在流动相、固定相间溶解度不同(分配作用)。可分为液-液分配色谱和键合固定相分配色谱。
	\item 固定相:
	\begin{itemize}
		\item 非极性键合固定相:键合在载体表面的功能分子是烷基、苯基等非极性有机分子。如最常用的ODS(十八烷基键合硅胶)柱或$\ce{C18}$柱就是最典型的代表,其极性很小。
		\item 极性键合固定相:键合在载体表面的功能分子是具有二醇基、醚基、氰基、氨基等极性基团的有机分子
	\end{itemize}
	\item 流动相:
	\begin{itemize}
		\item 正相HPLC(normal  phase, HPLC):是由极性固定相和非极性(或弱极性)流动相所组成的HPLC体系。其代表性的固定相是改性硅胶、氰基柱等,代表性的流动相是正己烷。吸附色谱也属正相HPLC。
		\item 反相HPLC(reversed phase, HPLC):由非极性固定相和极性流动相所组成的液相色谱体系,与正相HPLC体系正好相反。其代表性的固定相是ODS柱,代表性的流动相是甲醇和乙腈。
	\end{itemize}
	\item 应用特点:考虑流动相极性、选择性(按接受质子能力、给出质子能力和偶极作用能力分)。
\end{itemize}

\subsection{离子交换色谱法(ion exchange chromatography, IEC)}
\begin{itemize}
	\item 原理: 通过不同离子与交换基团的作用力大小不同(则保留时间不同)来进行分离。
	\item 固定相:离子交换剂,表面有离子交换基团。
	\begin{itemize}
		\item 带负电:分离阳离子。如磺酸基、羧基;
		\item 带正电:分离阴离子。如季铵盐。
	\end{itemize}
	\item 应用特点:适于分离带电的物质,流动相常用含盐的缓冲液,有时也加入有机溶剂以增加某些组分的溶解度。
\end{itemize}

\subsection{体积排斥色谱/凝胶色谱/分子筛色谱}
\begin{itemize}
	\item 原理:多孔物质做固定相,样品分子受孔径大小影响而分离。
	\item 应用特点:不需要通过改变流动相组成的方法来控制分离度,故流动相仅需考虑对样品的溶解性、低粘度和与柱填料匹配的要求。
\end{itemize}

\subsection{选择分离类型的原则}
根据各方法的特点来选择,如分离结构异构体和族分离时用吸附色谱,组分在两相中溶解性明显不同时用分配色谱,分离带电组分用离子交换色谱,分离大小差异很大的分子(如生物大分子等)时使用体积排斥色谱。
	
\subsection{色谱在生物分析中的应用}
\begin{itemize}
	\item 用于分离提纯,如毛细管电泳分离和富集维甲酸异构体。
	\item 用于研究未知反应的生成物,如通过峰的多少来判断生成物数量等。
	\item 用于生物分子的鉴定、分析,可通过色谱分析氨基酸、DNA、兴奋剂等分子。
	\item 用于疾病诊断。如用气相色谱检测肿瘤早期病人呼吸气中特有的挥发性有机物(VOCs)或原有的VOC含量改变。
	\item 对单个细胞上的特定蛋白进行检测。如毛细管电泳法检测单个细胞上Pgp的含量。
\end{itemize}


\section{毛细管电动色谱和毛细管电泳}
%5.了解细管电动色谱、毛细管电泳。

\subsection{毛细管电泳}
以内径$20\sim 200\mathrm{\mu m}$的毛细管为分离通道,高压直流电场为驱动力,依据样品各组分淌度或和分配系数的差异实现分离的技术。

\subsubsection{原理}
物质离子在电场中迁移速度不同:
\begin{equation*}
	v=\dfrac{qE}{f}=\dfrac{q}{6\pi \gamma\eta}E
\end{equation*}
$\gamma$是离子的表观液态动力学半径,$\eta$为介质的粘度。

淌度$\mu$:\textbf{单位场强}下的平均电泳速度。
\begin{equation*}
	\mu=\dfrac{v}{E}=\dfrac{q}{6\pi \gamma\eta}
\end{equation*}

电渗流:固液间形成双电层,液体两端施加电压时就会形成电渗流。电渗流为平流,展宽很小。速度一般为离子迁移速度的$5\sim 7$倍。
\begin{equation*}
	v_{eof}=\mu E=L_{ef}/t_{eo}
\end{equation*}

$L_{ef}$为毛细管有效长度,$t_{eo}$是电渗流标记物(中性物质)的迁移时间。

电渗流的方向:取决于毛细管内表面电荷性质,内表面带负电,则溶液带正电,电渗流流向阴极,内表面带正电则相反。可通过毛细管改性和加电渗流反转剂(阳离子表面活性剂等)使电渗流方向改变。

离子的表观迁移速度等于离子在电场中迁移速度和电渗流速度的矢量和。电渗流速度大,于是可以带动电场中原本朝两个方向运动的离子(包括中性分子!)同向移动,一次完成检测。

\subsubsection{影响电渗流的因素}
\begin{itemize}
	\item 电场强度:电渗流速度与电场强度成正比。
	\item 毛细管材料表面电荷特性的影响。
	\item pH值的影响。溶液pH影响毛细管表面的电离。适应毛细管中,pH=7时电渗流最大,pH<3,表面完全被氢离子中和,电渗流为0。
	\item 缓冲液离子浓度:离子浓度越高,双电层厚度越小,电渗下降。此外也可影响溶液粘度和工作电流。
	\item 温度:温度变化来自焦耳热。温度越高,粘度越低,电渗流增大。
	\item 添加剂:高浓度中性盐使离子强度增大,溶液粘度增大,电渗流减小。不同电荷的表面活性剂可改变电渗流大小和方向。
\end{itemize}

\subsubsection{分离效果的评价}
\begin{itemize}
	\item 迁移时间(保留时间)
	\[t=\dfrac{L_{ef}L}{\mu_{ap}E}\]
	$E$为外加电压,$L$为毛细管总长度,$\mu_{ap}$为表观淌度,即离子淌度和电渗流淌度的矢量和。
	\item 分离效率(塔板数)
	\[n=5.54\dfrac{t_R}{W_{1/2}}\]
	毛细管电泳中仅存在纵向扩散,扩散系数小的物质分离效率高,这也是分离生物大分子的依据。
	\item 分离度
	\[R=\dfrac{2(t_2-t_1)}{W_1+w_2}\]
\end{itemize}

\subsubsection{影响分离的主要因素}
\noindent 影响分离度的主要因素:工作电压;毛细管有效长度与总长之比;有效淌度差。

\noindent \textbf{影响分离效率的因素}:
\begin{itemize}
	\item 纵向扩散:大分子的扩散系数小,这是大分子试样分离的依据。
	\item 进样:进样长度太大时,引起的峰展宽大于纵向扩散,导致分离效率下降。实际进样长度应小于等于毛细管总长的$1\%\sim 2\%$。
	\item 焦耳热和温度梯度:散热梯度中形成温度梯度(中心温度高)将破坏塞流,导致区带展宽。可通过减小毛细管内径和控制散热的方法缓解。
	\item 溶质与管壁间相互作用:蛋白、多肽等带电多,且含有较多疏水基,吸附问题比较严重。可用两性离子代替强电解质,浓度约为溶质的$100\sim 1000$倍时,抑制吸附且不增加溶液电导,对电渗流影响不大。
	\item 其他因素:电分散作用、层流现象等。
\end{itemize}

\subsubsection{毛细管电泳仪}
(1)高压电源:$0\sim 30\mathrm{kV}$,稳定、连续可调的直流电源。可恒压、恒流、恒功率输出,电源极性易转换。

(2)毛细管:需电绝缘、紫外/可见透明、富有弹性,内径一般在$10\sim 150\mathrm{\mu m}$。目前有玻璃、熔融石英和聚四氟乙烯等材质,在外围一般包一层聚合物薄膜。

(3)进样方法:分为流体力学进样,电动进样(适合粘度大的试样,易导致进样不均或离子丢失)和扩散进样。

(4)缓冲液池:要求化学惰性,机械稳定性好。

(5)柱恒温系统:要求稳定、快速。

(6)检测器:有紫外-可见、荧光和激光诱导荧光三种,灵敏度依次提高,后两种样品需衍生。

\noindent 毛细管电泳仪的特点:
\begin{itemize}
	\item 仪器简单,易自动化
	\item 分析速度快,分离效率高:理论塔板数达$10^5\sim 10^7$/m。
	\item 操作方便、消耗少:纳升级进样量,水介质中进行
	\item 应用范围极广
\end{itemize}


\subsection{毛细管电泳分离模式}
\begin{itemize}
	\item 毛细管区带电泳:最基本、应用最广的分离模式,利用各组分荷质比的差异实现分离与检测,分离基础是组分间淌度的差别。
	\item 胶束电动毛细管色谱:缓冲液加入离子型表面活性剂(如SDS),其浓度达到临界浓度,形成疏水内核、外部带负电的胶束(假固定相)。中性分子在胶束相和水相之间分配,疏水性强的组分与胶束结合的较牢,流出时间长。可用于分离中性物质。
	\item 毛细管凝胶电泳:将聚丙烯酰胺等在毛细管柱内交联生成凝胶,多孔性,具有分子筛作用。能有效减少组分扩散,所得峰形尖锐,分离效率高。蛋白质、DNA等质荷比与分子大小无关,而利用凝胶电泳可以很好分离,是DNA排序的重要手段。

	特点:抗对流性好、散热性好、分离度极高。

	无胶筛分技术:利用低粘度线性聚合物溶液代替高粘度交联聚丙烯酰胺。柱便宜,易于制备。
	\item 毛细管等电聚焦:根据等电点差别分离生物大分子的高分辨率电泳技术。毛细管内充有两性电解质,在直流电($6\sim 8\mathrm{V}$)作用下将形成由阳极到阴极逐渐升高的pH梯度。电渗流在该方法中不利,需尽量消除。
	\item 毛细管等速电泳:将两种淌度差别很大的缓冲液作为前导离子和尾随离子,试样离子的淌度全部位于两者之间并以同一速度移动。原理是淌度不同的离子形成区带的场强不同,可使同一淌度的离子聚集在一起。

	特点:界面明显,具有富集、浓缩作用。
	\item 毛细管电渗色谱:在毛细管壁键合或涂渍高效液相色谱的固定液,以电渗流为流动相,试样组分在两相间的分配为分离机制的电动色谱过程。
\end{itemize}