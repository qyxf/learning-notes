\usepackage[version=4]{mhchem}

%自定义微分号
\newcommand{\di}[1]{\mathrm{d}#1}
\newcommand{\p}[2]{\dfrac{\partial #1}{\partial #2}}
\newcommand{\pp}[2]{\dfrac{\partial ^2 #1}{\partial #2 ^2}}
\newcommand{\dy}[2]{\dfrac{\di{#1}}{\di{#2}}}
\newcommand{\ddy}[2]{\dfrac{\mathrm{d} ^2 #1}{\mathrm{d} #2 ^2}}

\newcommand{\tabincell}[2]{\begin{tabular}{@{}#1@{}}#2\end{tabular}}%单元格内强制换行

\usepackage{graphicx}   % 导入图片
\usepackage{subfigure}  % 并排子图
\usepackage{wrapfig} %图文混排
\usepackage{lastpage}
\usepackage{fontawesome5}
%\usepackage{unicode-math} % 数学字体
%\setmathfont{Cambria Math}
\newcommand{\solve}{\noindent \textcolor{main}{解}\hspace{1em}}

%模板中添加了
%\newtcbtheorem

% 本文档命令
\usepackage{array}
\newcommand{\ccr}[1]{\makecell{{\color{#1}\rule{1cm}{1cm}}}}