\chapter*{编者说明}

本笔记可以作为物理专业的理论力学以及工科专业理论力学中的分析力学部分的学习参考。作为一名医学生,曾经的物理竞赛党,笔者写作这篇笔记的原因是多方面的,最终希望呈现的效果是语言尽量易懂但不失严谨。

第一章从矢量力学到分析力学介绍了从牛顿力学引出拉格朗日力学的方法;第二章最小作用量原理简单介绍了泛函,分析力学的基本原理以及如何由其导出Euler-Largrange方程;第三章对称性与守恒律介绍了Euler-Largrange方程导出守恒量的方法以及拉格朗日力学的其他应用;第四章正则方程介绍了如何从Euler-Largrange方程出发,使用Legendre变换导出Hamilton正则方程并介绍了泊松括号;第五章Maupertuis原理介绍了从分析力学的基本原理——最小作用量原理导出正则方程和对保守系使用简约作用量确定运动轨迹方程的方法;最后一章Hamilton-Jacob方程介绍了正则变换以及利用正则变换结合Hamilton-Jacob方程求解力学问题的一般方法。本笔记主要参考了朗道的《力学》和刘川老师的《理论力学》,其中前三章属于拉格朗日力学,结合两本书的思路,使用非相对论体系;后三章属于哈密顿力学,主要参考了朗道的书,但因为一些原因没有写全。

由于本笔记的写作目的等原因,笔者没有写作分析力学在具体模型中的应用,而只写基本原理,还请见谅。笔者还将来计划写作电动力学,狭义相对论等方面的笔记,敬请期待。如对本笔记有任何意见和建议,欢迎联系笔者 邮箱:\texttt{RobinFeng@stu.xjtu.edu.cn},以便于笔者修改完善。

最后,感谢钱院学辅排版组对我“不堪入目”的手稿进行排版,谢谢钱院学辅及各位同学的支持!


\vspace{1em}
\makeatletter
\begin{flushright}
	\@author\\
	\@date
\end{flushright}
\makeatother


\cleardoublepage
\tableofcontents