\chapter{正则方程}
\section{Legendre变换与Hamilton方程}
我们已经知道,Largrange力学是使用广义坐标$q_i$和广义速度$\dot{q}_i$来描述系统的力学状态的,
而接下来要讲述的方法则是使用广义坐标$q_i$和广义动量$p_i$,即Hamilton力学,为此,我们需要从Largrange力学出发,
使用Legendre变换以实现变量代换,考虑一个系统的Largrange函数的微分,即
\begin{equation}
    \mathrm{d}L=\frac{\partial L}{\partial q_i}\mathrm{d}q_i+\frac{\partial L}{\partial\dot{q}_i}d\dot{q}_i+\frac{\partial L}{\partial t}\mathrm{d}t
\end{equation}
代入Eluer-Largrange方程,得
\begin{equation}
    \mathrm{d}L=\frac{\rm d}{\mathrm{d}t}\left(\frac{L}{\partial \dot{q}_i}\right)\mathrm{d}q_i+\frac{\partial L}{\partial\dot{q}_i}d\dot{q}_i
    +\frac{\partial L}{\partial t}\mathrm{d}t
\end{equation}
定义$\displaystyle{p_i=\frac{\partial L}{\partial \dot{q}_i}}$为广义动量,则
\begin{equation}
    \mathrm{d}L=\dot{p}_i\mathrm{d}q_i+p_id\dot{q}_i+\frac{\partial L}{\partial t}\mathrm{d}t
\end{equation}
变换右边第二项,得
\begin{equation}
    \mathrm{d}L=\dot{p}_i\mathrm{d}q_i+d\left(p_i\dot{q}_i\right)-\dot{q}_i\mathrm{d}p_i+\frac{\partial L}{\partial t}\mathrm{d}t
\end{equation}
\begin{equation}
    \mathrm{d}\left(p_i\dot{q}_i-L\right)=\dot{q}i\mathrm{d}p_i-\dot{p}_i\mathrm{d}q_i-\frac{\partial L}{\partial t}\mathrm{d}t
\end{equation}
令$p_i\dot{q}_i-L=H$,称其为Hamilton量,则
\begin{equation}
    \frac{\partial H}{\partial p_i}=\dot{q}_i,\frac{\partial H}{\partial q_i}=-\dot{p}_i
\end{equation}
称此式为Hamilton方程,亦称正则方程,称$p_i,q_i$为一对共轭变量\\
我们注意到,由$(4.5)$式得
\begin{equation}
    \frac{\mathrm{d}H}{\mathrm{d}t}=\dot{q}_i\dot{p}_i-\dot{p}_i\dot{q}_i-\frac{\partial L}{\partial t}=-\frac{\partial L}{\partial t}
\end{equation}
又由$(4.6)$式得
\begin{equation}
    \frac{\mathrm{d}H}{\mathrm{d}t}=\frac{\partial H}{\partial p_i}\dot{p}_i+\frac{\partial H}{\partial q_i}\dot{q}_i
    +\frac{\partial H}{\partial t}=\frac{\partial H}{\partial t}
\end{equation}
所以
\begin{equation}
    \frac{\mathrm{d}H}{\mathrm{d}t}=\frac{\partial H}{\partial t}=-\frac{\partial L}{\partial t}
\end{equation}
即若Largrange函数不显含时间,则Hamilton函数亦不显含时间,且其对时间得变化率为0.
考虑Hamilton函数的物理意义是用$p_i,q_i$表示的系统的总能量,我们得到能量守恒定律.
\section{Poisson括号}
考虑一个变量为$p_i,q_i,t$的力学量$f$对时间的导数,即
\begin{equation}
    \frac{\mathrm{d}f}{\mathrm{d}t}=\frac{\partial f}{\partial p_i}\dot{p}_i+\frac{\partial f}{\partial q_i}\dot{q}_i
    +\frac{\partial f}{\partial t}
\end{equation}
代入正则方程,得
\begin{equation}
    \frac{\mathrm{d}f}{\mathrm{d}t}=\frac{\partial H}{\partial p_i}\frac{\partial f}{\partial q_i}-
    \frac{\partial H}{\partial q_i}\frac{\partial f}{\partial p_i}+\frac{\partial f}{\partial t}
\end{equation}
定义$H,f$的Poisson括号$\displaystyle{\{H,f\}=\frac{\partial H}{\partial p_i}\frac{\partial f}{\partial q_i}
-\frac{\partial H}{\partial q_i}\frac{\partial f}{\partial p_i}}$,则
\begin{equation}
    \frac{\mathrm{d}f}{\mathrm{d}t}=\{H,f\}+\frac{\partial f}{\partial t}
\end{equation}
若力学量$f$对时间的变化率为0,称其为系统得运动积分(守恒量),若$f$为运动积分,则
\begin{equation}
    \{H,f\}+\frac{\partial f}{\partial t}=0
\end{equation}
特别地,当$f$不显含时间时,有
\begin{equation}
    \{H,f\}=0
\end{equation}
对于任意两个函数$f,g$也有Poisson括号$\displaystyle{\{f,g\}=\frac{\partial f}{\partial p_i}\frac{\partial g}{\partial q_i}
-\frac{\partial f}{\partial q_i}\frac{\partial g}{\partial p_i}}$,Poisson括号有许多性质,下面直接列举,略去证明
\begin{equation}
    \{f,g\}=-\{g,f\}
\end{equation}
\begin{equation}
    \{f,c\}=0\quad(c\mbox{为常数})
\end{equation}
\begin{equation}
    \{c_1f_1+c_2f_2,g\}=c_1\{f_1,g\}+c_2\{f_2,g\}
\end{equation}
\begin{equation}
    \{f_1f_2,g\}=f_1\{f_2,g\}+f_2\{f_1,g\}
\end{equation}
\begin{equation}
    \frac{\partial}{\partial t}\{f,g\}=\{\frac{\partial f}{\partial t},g\}+\{f,\frac{\partial g}{\partial t}\}
\end{equation}
特别地,对于$g=q_k,p_k$时,Poisson括号化为偏导数,即
\begin{equation}
    \{f,q_k\}=\frac{\partial f}{\partial p_k},\quad\{f,p_k\}=-\frac{\partial f}{\partial q_k}
\end{equation}
再令$f=p_i,q_i$,得到
\begin{equation}
    \{p_i,q_i\}=\delta_{ik},\quad \{p_i,p_k\}=0,\{q_i,q_k\}=0
\end{equation}
Poisson括号的一个重要性质是所谓Jacob恒等式,即
\begin{equation}
    \left\{f,\{g,h\}\right\}+\left\{g,\{h,f\}\right\}+\left\{h,\{f,g\}\right\}=0
\end{equation}
我们来简单地证明一下,注意到左边和式一定是由$f,g,h$的二阶偏导数组成的,
只要证明不存在$f,g,h$中任意一个函数的二阶偏导数,即得证,先考虑$f$,\\
令$D_1(\phi)=\{g,\phi\},\quad D_2(\phi)=\{h,\phi\}$,则含$f$的二阶偏导数的项为
\begin{equation}
    D_1\left(D_1(f)\right)-D_2\left(D_1(f)\right)=(D_1D_2-D_2D_1)f
\end{equation}
因为$D_1,D_2$可以写成
\begin{equation}
    D_1=\xi_k\frac{\partial}{\partial X_k},\quad D_2=\eta_k\frac{\partial}{\partial X_k}
\end{equation}
其中$\xi_k,\eta_k$为变量$X_k$的任意函数,则
\begin{equation}
    D_1D_2-D_2D_1=\left(\xi_k\frac{\partial\eta_l}{\partial X_k}-\eta_k\frac{\partial\xi_l}{\partial X_k}\right)
    \frac{\partial}{\partial X_l}
\end{equation}
不存在二阶偏导数项,所以式中不存在$f$的二阶偏导数,同理可证不存在$g,h$的二阶偏导数,Jacob恒等式得证.\par
Jacob恒等式的一个重要应用是所谓Poisson定理,即:任意2个运动积分的Poisson括号一定是运动积分,若$f,g$不显含时间,
则证明很易,只需令$(4.22)$式中$h=H$,即
\begin{equation}
    \left\{f,\{g,H\}\right\}+\left\{g,\{H,f\}\right\}+\left\{H,\{f,g\}\right\}=0
\end{equation}
由$(4.14)$式,得
\begin{equation}
    \left\{H,\{f,g\}\right\}=0
\end{equation}
即$\{f,g\}$为运动积分,若$f,g$显含时间,则将其对时间求偏导,得
\begin{equation}
    \frac{\partial}{\partial t}\{f,g\}=\{\frac{\partial f}{\partial t},g\}+\{f,\frac{\partial g}{\partial t}\}
\end{equation}
代入$(4.13)$式,得
\begin{equation}
    \frac{\partial}{\partial t}\{f,g\}=\left\{g,\{H,f\}\right\}-\left\{f,\{H,g\}\right\}
\end{equation}
运用Jacob恒等式,化简得
\begin{equation}
    \left\{H,\{f,g\}\right\}+\frac{\partial}{\partial t}\{f,g\}=0
\end{equation}
得证.
\chapter{\rm Maupertuis原理}
\section{从最小作用量原理到正则方程}
由定义我们知道$\displaystyle{s=\int_{t_1}^{t_2}L\mathrm{d}t}$,现在,
让我们从另一个角度看最小作用量原理,若用作用量$S$表示运动轨道固定,
起点$q(t_0)$固定,但终点$q(t)$变化的表示系统运动的量,对$S$变分,得到
\begin{equation}
    \partial S=\frac{\partial L}{\partial\dot{q}_i}\partial q_i\Big|_{t_1}^{t_2}+\int_{t_1}^{t_2}
    \left(\frac{\partial L}{\partial\dot{q}_i}-\frac{d}{\mathrm{d}t}\frac{\partial L}{\partial\dot{q}_i}\right)\delta q\mathrm{d}t
\end{equation}
由于轨道符合Euler-Largrange方程,故积分项为0,取$\delta {q_i}(t_1)=0,\delta {q_i}(t_2)=\delta {q_i}$,有
\begin{equation}
    \partial S=\frac{\partial L}{\partial\dot{q}_i}\partial q_i=p_i\delta {q_i}
\end{equation}
所以
\begin{equation}
    \frac{\partial S}{\partial q_i}=p_i
\end{equation}
由定义知
\begin{equation}
    \frac{\mathrm{d}S}{\mathrm{d}t}=L,\quad \frac{\mathrm{d}S}{\mathrm{d}t}=\frac{\partial S}{\partial t}+\frac{\partial S}{\partial q_i}\dot{q}_i
    =\frac{\partial S}{\partial t}+p_i\dot{q}_i
\end{equation}
所以
\begin{equation}
    \frac{\partial S}{\partial t}=-(p_i\dot{q}_i-L)=-H
\end{equation}
由$(5.3),(5.5)$式得(将$S$看作坐标和时间的函数)
\begin{equation}
    \mathrm{d}S=p_id\dot{q}_i-H\mathrm{d}t
\end{equation}
\begin{equation}
    S=\int(p_id\dot{q}_i-H\mathrm{d}t)
\end{equation}
对其变分,得
\begin{equation}
    \delta S=\int\left[\delta p_i\mathrm{d}q_i+p_id\delta q_i-\left(\frac{\partial H}{\partial p_i}\delta p_i
    +\frac{\partial H}{\partial q_i}\delta q_i\right)\mathrm{d}t\right]
\end{equation}
\begin{equation}
    \delta S=\int\left(\mathrm{d}q_i-\frac{\partial H}{\partial p_i}\mathrm{d}t\right)\delta p_i+p_i\delta q_i\Big|
    -\int\left(\mathrm{d}p_i+\frac{\partial H}{\partial q_i}\mathrm{d}t\right)\delta q_i
\end{equation}
由于$\delta S=0,\delta q_i=0,$所以
\begin{equation}
    \mathrm{d}q_i=\frac{\partial H}{\partial p_i}\mathrm{d}t,\quad \mathrm{d}p_i=-\frac{\partial H}{\partial q_i}\mathrm{d}t
\end{equation}
即正则方程
\section{Maupertuis原理}
对于一个保守体系,$H=E=Const$,由$(5.7)$式得
\begin{equation}
    S=\int p_i\mathrm{d}q_i-E(t-t_0)
\end{equation}
称$\int p_i\mathrm{d}q_i$为简约作用量,记为$S_0$,对$S$变分,得
\begin{equation}
    \delta S=\delta S_0-E\delta t
\end{equation}
由$(5.5)$式可得
\begin{equation}
    \delta S=-H\delta t-E\delta t
\end{equation}
故对于保守体系,我们得到
\begin{equation}
    \delta S_0=\delta\int p_i\mathrm{d}q_i=0
\end{equation}
在此基础上,由$\displaystyle{E\left(q,\frac{\mathrm{d}q}{\mathrm{d}t}\right)=E}$反解出$\mathrm{d}t$并代入定义式
$\displaystyle{p_i=\frac{\partial L(q,\frac{\mathrm{d}q}{\mathrm{d}t})}{\partial \dot{q}_i}}$,
结合$(5.14)$式可得到系统的轨道方程,称此原理为Maupertais原理。当Largrange函数形式可写成$(2.14)$式时,可得
\begin{equation}
    p_i=a_{ij}(q)\dot{q}_i
\end{equation}
\begin{equation}
    E=\frac{1}{2}a_{ij}(q)\dot{q}_i\dot{q}_j+V(q)
\end{equation}
由$(5.16)$式解出
\begin{equation}
    \mathrm{d}t=\sqrt{\frac{a_{ij}(q)\mathrm{d}q_i\mathrm{d}q_j}{2(E-V)}}
\end{equation}
代入$(5.15),(5.14)$式可得
\begin{equation}
    \delta S_0=\int\sqrt{2a_{ij}(q)\mathrm{d}q_i\mathrm{d}q_j(E-V)}=0
\end{equation}
由此可确定运动轨道方程,对$(5.16)$式积分,得
\begin{equation}
    \int\sqrt{\frac{a_{ij}(q)\mathrm{d}q_i\mathrm{d}q_j}{2(E-V)}}=t-t_0
\end{equation}
$(5.19)(5.18)$式共同确定系统的运动状态
\chapter{\rm Hamilton-Jacob方程}
\section{正则变换}
考虑一组从旧得坐标、动量向新的坐标、动量的变换
\begin{equation}
    P=P(p,q,t),\quad Q=Q(p,q,t),\quad H=H(P,Q,t)
\end{equation}
若其满足正则方程
\begin{equation}
    \dot{p}_i=-\frac{\partial H}{\partial Q_i},\quad \dot{Q}_i=\frac{\partial H}{\partial P_i}
\end{equation}
则称该变换为正则变换,由于其满足Hamilton方程,其必满足
\begin{equation}
    \delta\int(p_i\mathrm{d}q_i-h\mathrm{d}t)=\delta\int(P_i\mathrm{d}Q_i-H\mathrm{d}t)=0
\end{equation}
由此得到
\begin{equation}
    p_i\mathrm{d}q_i-h\mathrm{d}t-P_i\mathrm{d}Q_i-H\mathrm{d}t+\mathrm{d}F
\end{equation}
其中函数$F$在两积分极限时之差为对变分不起作用的常数,整理得
\begin{equation}
    \mathrm{d}F=\dot{p}_i\mathrm{d}q_i-P_i\mathrm{d}Q_i+(H-h)\mathrm{d}t
\end{equation}
所以
\begin{equation}
    \frac{\partial F}{\partial q_i}=p_i,\quad \frac{\partial F}{\partial Q_i}=-P_i,\quad \frac{\partial F}{\partial t}=H-h
\end{equation}
我们还可以对$(6.5)$式进行Legendre变换,以得到不同变量表示得像$F$一样得函数(称其为母函数),如:
\begin{equation}
    \mathrm{d}\varPhi=\mathrm{d}(F+P_iQ_i)=p_i\mathrm{d}q_i+Q_i\mathrm{d}P_i+(H-h)\mathrm{d}t
\end{equation}
得到
\begin{equation}
    \frac{\partial\varPhi}{\partial q_i}=p_i,\quad \frac{\partial\varPhi}{\partial P_i}=Q_i,
    \quad \frac{\partial\varPhi}{\partial t}=H-h 
\end{equation}
正则变换得一个重要性质是Poisson括号不变,即
\begin{equation}
    \{f,g\}_{P,Q}=\{f,g\}_{p,q}
\end{equation}
证明如下:
\begin{align*}
    \{f,g\}_{P,Q}
    =&\frac{\partial f}{\partial P_i}\frac{\partial g}{\partial Q_i}
    -\frac{\partial f}{\partial Q_i}\frac{\partial g}{\partial P_i}\\
    =&\left(\frac{\partial f}{\partial p_i}\frac{\partial p_i}{\partial P_i}
    +\frac{\partial f}{\partial q_i}\frac{\partial q_i}{\partial P_i}\right)
    \left(\frac{\partial g}{\partial p_j}\frac{\partial p_j}{\partial Q_i}
    +\frac{\partial g}{\partial q_j}\frac{\partial q_j}{\partial Q_i}\right)\\
    &-\left(\frac{\partial f}{\partial p_i}\frac{\partial p_i}{\partial Q_i}
    +\frac{\partial f}{\partial q_i}\frac{\partial q_i}{\partial Q_i}\right)
    \left(\frac{\partial g}{\partial p_j}\frac{\partial p_j}{\partial P_i}
    +\frac{\partial g}{\partial q_j}\frac{\partial q_j}{\partial P_i}\right)\\
    =&\frac{\partial f}{\partial p_i}\frac{\partial g}{\partial p_j}\{p_i,p_j\}_{P,Q}
    +\frac{\partial f}{\partial q_i}\frac{\partial g}{\partial p_j}\{q_i,p_j\}_{P,Q}\\
    &+\frac{\partial f}{\partial q_i}\frac{\partial g}{\partial q_j}\{q_i,q_j\}_{P,Q}
    +\frac{\partial f}{\partial p_i}\frac{\partial g}{\partial q_j}\{p_i,p_j\}_{P,Q}\\
    =&\left(\frac{\partial f}{\partial p_i}\frac{\partial g}{\partial q_i}
    -\frac{\partial f}{\partial q_i}\frac{\partial g}{\partial p_i}\right)\delta_{ij}
\end{align*}
\begin{equation}
    =\{f,g\}_{p,q}\qquad\qquad\qquad\qquad\qquad\qquad\quad
\end{equation}
\section{Hamilton-Jacob方程}
由$(5.5)$式可得
\begin{equation}
    \frac{\partial}{\partial t}S+H\left(q,\frac{\partial S}{\partial q},t\right)=0
\end{equation}
称其为Hamilton-Jacob方程,对于自由度为$s$的系统,我们不加证明地指出,解的形式为
\begin{equation}
    S=f(q_1,\cdots,q_s;\alpha_1,\cdots,\alpha_s;t)+A
\end{equation}
其中$\alpha_1,\cdots,\alpha_s$与$A$为任意常数,以$f$为母函数进行正则变换,以$\alpha_i$为新动量,
$\beta_i$为新坐标,由$(6.8)$式得
\begin{equation}
    p_i=\frac{\partial f}{\partial q_i},\quad\beta_i=\frac{\partial f}{\partial \alpha_i},\quad\frac{\partial f}{\partial t}+h=H
\end{equation}
由于$f$满足Hamilton-Jacob方程,故
\begin{equation}
    H=\frac{\partial f}{\partial t}+h=\frac{\partial S}{\partial t}+h=0
\end{equation}
由正则方程,
\begin{equation}
    \dot{\alpha}_i=0,\dot{\beta}_i=0
\end{equation}
即,$\alpha_i,\beta_i$为常数,同时利用$s$个方程
\begin{equation}
    \frac{\partial f}{\partial\alpha_i}=\beta_i
\end{equation}
可将坐标q用$2s$个常数和时间$t$表示出来,运动方程得解.由此总结求解力学问题的方法:
\begin{enumerate}[(1)]
    \item 列出Hamilton-Jacob方程
    \item 求出函数$S$包含常数$\alpha_1,\cdots,\alpha_s,A$.
    \item 将$S$对$\alpha_i$求偏导得到$\beta_i$
    \item 由$(6.16)$式反解出$q$作为$2s$个常数和$t$的函数
    \item $\displaystyle{p_i=\frac{\partial S}{\partial q_i}}$得到$p$关于时间的函数.
\end{enumerate}
\section{分离变量}
有时为了得到$S$,可利用分离变量的方法.假设某一坐标$q$,与$\displaystyle{\frac{\partial S}{\partial q}}$,
在Hamilton-Jacob方程中仅以组合$\displaystyle{\varphi(q_1,\frac{\partial S}{\partial q_1})}$形式出现,即方程可写为
\begin{equation}
    \varPhi\left\{q_i,t,\frac{\partial S}{\partial q_i},\frac{\partial S}{\partial t},
    \varphi\left(q_1,\frac{\partial S}{\partial q_1}\right)\right\}=0
\end{equation}
则$S$可写为
\begin{equation}
    S=S'(q_i,t)+S_1(q_1)
\end{equation}
代入$(6.17)$式,得
\begin{equation}
    \varPhi\left\{q_i,t,\frac{\partial S'}{\partial q_i},\frac{\partial S'}{\partial t},
    \varphi\left(q_1,\frac{\partial S_1}{\partial q_1}\right)\right\}=0
\end{equation}
假设$(6.18)$式已得,则$q_1$变化不应影响$(6.19)$式成立,故
\begin{equation}
    \varphi\left(q_1,\frac{\partial S_1}{\partial q_1}\right)=\alpha_1
\end{equation}
\begin{equation}
    \varPhi\left\{q_i,t,\frac{\partial S'}{\partial q_i},\frac{\partial S'}{\partial t},\alpha_1\right\}=0
\end{equation}
由$(6.20)$式可求出$S_1(q_1)$,依次类推将$S$完全求解
\newpage
\section*{参考书目}
\noindent
[1]\;L.D.朗道,E.M.粟弗席兹.力学[M].北京:高等教育出版社,2007\\\relax
[2]刘川.理论力学[M].北京:北京大学出版社,2019 \\\relax
[3]吴崇试,高春媛.数学物理方法[M].北京:北京大学出版社,2019
\end{document}