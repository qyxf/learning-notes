\chapter{}

    \begin{theorem}
        设u$\in{}$C$^{2}$(D)是实值函数,那么u是D上的次调和函数的充分必要条件,对任意的z$\in{}$D,有$\Delta{}$U(z)$\geq{}$0.
    \end{theorem}

    \begin{proof}
        ``$\leftarrow{}$''设$\Delta{}$u(z)$\geq{}$0在D内处处成立,则对于G$\in{}$D上的调和函数h,若u$\leq{}$h在$\partial{}$G上成立,设u$_{1}$=u-h,则$\Delta{}$u$_{1}$=$\Delta{}$u-$\Delta{}$h=$\Delta{}$u$\geq{}$0。故由定理7.3.6以及u$_{1}$在$\partial{}$G上有u$_{1}$(z)$\leq{}$0。故在G上有u$_{1}$(z)$\leq{}$0,即u(z)$\leq{}$h(z)。由G的任意性及定理7.3.3得u是D上的次调和函数。
    \end{proof}

    \begin{proof}
       ``$\rightarrow{}$''设u是D上的次调和函数,若存在a属于D使$\Delta{}$u(a)$<$0,则存在$\epsilon{}$$>$0,使得在B(a,
$\epsilon{}$)上有$\Delta{}$u<0,故在B(a,
$\epsilon{}$)上有$\Delta{}$(-u)$>$0,则由充分性证明我们得到-u是B(a,$\epsilon{}$)上的次调和函数,故在B(a,$\epsilon{}$)中的任意一个圆B(b,$\epsilon{}$)
$\subset{}$ B(a,$\epsilon{}$)有-u(b)$\leq{}$$\frac{1}{{2\pi }}\int_0^{2\pi }
{{\rm{ - u}}(b + \partial {e^{2\theta }})d\theta } $
,故u(b)$\geq{}$$\frac{1}{{2\pi }}\int_0^{2\pi } {{\rm{u}}(b + \partial {e^{2\theta
}})d\theta } $,而又由于u也是B(a,$\epsilon{}$)上的次调和函数,故u(b)$ \le \frac{1}{{2\pi
}}\int_0^{2\pi } {{\rm{u}}(b + \partial {e^{2\theta }}} )d\theta $
,故u在B(a,$\epsilon{}$)中满足平均值性质,故u为B(a,$\epsilon{}$)上的调和函数,故在B(a,$\epsilon{}$)上有$\Delta{}$u=0,矛盾!故$\forall{}$z$\in{}$D,有$\Delta{}$u(z)$\geq{}$0。
    \end{proof}

\begin{theorem}
        域D上的次调和函数域S线性空间,并且若V$_{1}$,V$_{2}$是域D上的次调和函数,则\\V(z)=max\{V$_{1}$(z),V$_{2}$(z)\}也是D上的次调和函数。
    \end{theorem}

\begin{theorem}
			若u(z)是区域D内连续的次调和函数,$\overline {B({z_0},z)}  \subset D$
。P$_{2}$(z)表示以z($\varphi $)在$\left| {z - {z_0}} \right| = r$
上的边值得到的\emph{Poisson}积分,则函数
$\widetilde U(z) = $ $\left\{ \begin{array}{cc}
{P_2}({\rm{z}}) & z \in B({z_0},r)\\
V(z) & z \in D\backslash B({z_0},r)
\end{array} \right.$ 
\end{theorem}

   \begin{proof}
  由于V是D上的次调和函数,故由定理7.2.2得$\widetilde U(z) \in C(D)$
,并且对于$\forall{}$$\left| {{z_1} - {z_0}} \right|$
=r,存在$\rho{}$$_{1}$>0,使得$V({z_1}) \le \int_0^{2\pi } {V({z_1} + \rho
{{\rm{e}}^{i\theta }})d\theta }$

\hspace{15pt}{\large 又由于z$\in{}$B(z$_{0}$,r),$\widetilde V({z_1} - V(z)) =
{P_2}(z) - V(z) = \frac{1}{{2\pi }}\int_{\left| {\xi  - {z_0}} \right| = r}
{\frac{{{r^2} - {{\left| {z - {z_0}} \right|}^2}}}{{{{\left| {\xi  - {z_0}}
\right|}^2}}}} V(\xi )d\theta  - V(z)$ ,又由定理7.1.6得$\widetilde V({z_1}) - u(z) =
{P_2}(z) - V(z) = \frac{1}{{2\pi }}\int_{\left| {\xi  - {z_0}} \right| = r}
{\frac{{{r^2} - {{\left| {z - {z_0}} \right|}^2}}}{{{{\left| {\xi  - {z_0}}
\right|}^2}}}} (V(\xi ) - u(z))d\theta
$,又由定理7.3.3得V(z)满足最大值原理,故任意的$\xi{}\in B({z_0},r)$ 有f($\xi
$)$\geq{}$V(z),故$\forall{}$z$\in{}$D有$\widetilde V(z) \ge
V(x)$,故$\forall{}$z$_{1}$满足 $\left| {{z_1} - {z_0}} \right| \ge r$ ,有$\widetilde
V({z_1}) \le \frac{1}{{2\pi }}\int_0^{2\pi } {\widetilde V({z_1} + \rho
{e^{i\theta }})} d\theta $,故$\widetilde V(z)$ 为D上的次调和函数。}
    \end{proof}

\begin{theorem}
设u$\in{}$C$^{2}$(D)是一个实值函数,若对任意的z$\in{}$D,有$\Delta{}$u(z)$\geq{}$0,那么对任意的域G$
\subset D$ ,u在G上的最大值必在$\partial G$ 上取到
\end{theorem}

\begin{proof}
先设对每个点z$\in{}$D,有$\Delta{}$u(z)$>$0。若u在G上的最大值在G的内点z$_{0}$内取到,记z$_{0}$=x$_{0}$+$\mathop
z\limits^. $ -y$_{0}$,g(t)=u(x$_{0}$,t),那么g在t=y$_{0}$处有最大值,故${\left.
{\frac{{{\partial ^2}u}}{{\partial {y^2}}}} \right|_{{z_0}}} = {g^{''}}({y_0})
\le 0$ 同理得${\left. {\frac{{{\partial ^2}u}}{{\partial {x^2}}}} \right|_{{z_0}}}
\le 0$,矛盾!\hspace{15pt}$\cdot{}$现设$\Delta{}$u(z)$\geq{}$0,令${u_\varepsilon }(z) =
u(z) + \varepsilon {\left| z \right|^2},\varepsilon  > 0$ ,故$\Delta
{u_\varepsilon }(z) = \Delta u(z) + \varepsilon  > 0$,故

${u_\varepsilon }${\large 在G上的最大值在$\partial G$上取到,故${u_\varepsilon }(z) \le
\overbrace {z \in \partial G}^{sub}{u_\varepsilon }(z),\varepsilon  \to 0$,得$u(z)
\le \overbrace {z \in \partial G}^{sub}u(z)$}
    \end{proof}


