\chapter{解析函数的{\rm Taylor}展开及其应用}
\section{Weierstrass定理}
    对于复数上的数列,其收敛性可以类似于$\mathbb{R}^2$中点集的收敛性即可得到.同样的我们有柯西收敛原理,此处略去\par
    对于复数上的数项级数$\displaystyle{\sum\limits_{n=0}^\infty z_n}$, 
    类似于实数中的常数项级数$\displaystyle{\sum\limits_{n=0}^\infty a_n}$.
    可以定义绝对收敛和条件收敛,以及级数极限的存在性和柯西收敛原理.\par
    下面讨论对于复变函数上的函数项级数$\displaystyle{\sum\limits_{n=1}^\infty f_n(z)}$:

\begin{mypro}
    设$\displaystyle{\sum\limits_{n=1}^\infty f_n(z)}$是定义在$\mathbb{E}$上的级数.
    我们说$\displaystyle{\sum\limits_{n=1}^\infty f_n(z)}$在$\mathbb{E}$上一致收敛到
    $f(z)$.$(\mbox{记为}\ \displaystyle{\sum\limits_{k=1}^n f_k(z)\Rightarrow f(z)})$,
    指$\forall\epsilon>0,\exists.N\in \mathbb{N}^*\quad s.t.\ \forall n>N$.有$\left|S_n(z)-f(z)\right|<\epsilon, \forall z\in E$,
    其中$\displaystyle{S_n(z)=\sum\limits_{k=1}^n f_k(z)}$.
\end{mypro}

\begin{mypro}
    级数$\displaystyle{\sum\limits_{n=1}^\infty f_n(z)}$在$\mathbb{E}$上一致收敛的充要条件是
    $\forall\epsilon>0,\exists.N\in \mathbb{N}^*\quad s.t.\ \forall n>N.\forall p\in N$
    有$\displaystyle{\left|\sum\limits_{i=1}^pf_{n+i}(z)\right|<\epsilon}$.对$\forall z\in\mathbb{E}$均成立.
\end{mypro}
\begin{proof}
    略.
\end{proof}

\begin{mypro}[函数项级数的weierstass判别法,略]
\end{mypro}

\begin{mypro}
    设级数$\displaystyle{\sum\limits_{n=1}^\infty f_n(z)\Rightarrow f(z)}.z\in \mathbb{E},\forall n\in\mathbb{N}^*, f_n\in C(E)$.
    则$f\in C(E)$.
\end{mypro}
\begin{proof}
    $\forall\epsilon>0.\exists N\in\mathbb{N}.\quad s.t.\ n>N$时
    $\displaystyle{\left|f(z)-S_n(z)\right|<\frac{\epsilon}{3}\ \forall z\in\mathbb{E}}$.\\
    对于给定的大于$N$的$n_0$.显然$S_{n_0}\in C(\mathbb{E})$.现任取一个$a\in \mathbb{E}$.故$S_{n_0}$在$a$处连续.\\
    故$\forall\epsilon>0,\exists\delta>0.\quad s.t.\ \forall z\in B(z,\delta)$有
    $\displaystyle{\left|f(z)-f(a)\right|<\frac{\epsilon}{3}}$.\\
    于是$z\in\mathbb{E}\bigcap B(a,\delta)$时有:\\
    $\displaystyle{\left|f(z)-f(a)\right|\leqslant\left|f(z)-S_{n_0}(z)\right|
    +\left|S_{n_0}(z)-S_{u_0}(a)\right|+\left|S_{u_0}(a)-f(a)\right|<\epsilon}$
\end{proof}

\begin{mypro}
    设级数$\sum\limits_{n=1}^\infty f_u(z)$.在可求长曲线$\gamma$上一致收敛到$f(z)$,若$\forall n\in\mathbb{N}^*,f_n\in C(\gamma),$
    则$\displaystyle{\int_{\gamma}f(z)dz=\sum\limits_{n=1}^\infty\int_{\gamma}f_n(z)dz}$.
\end{mypro}
\begin{proof}
    由定理\emph{4.1.4}.\ $f\in C(\gamma)$.\\
    由于$f_k\Rightarrow f$.故$\forall\epsilon>0.\exists N\in\mathbb{N}^*\quad s.t.\ n>N$时,
    有$\displaystyle{\left|\sum\limits_{k=1}^nf_k(z)-f(z)\right|<\epsilon}.\forall z\in \gamma$.
    故$n>N$时有$\displaystyle{\left|\sum\limits_{k=1}^n\int_\gamma f_k(z)dz-\int_\gamma f(z)dz\right|
    =\left|\int_\gamma\left(\sum\limits_{k=1}^nf_k(z)-f(z)\right)dz\right|<\epsilon\cdot|\gamma|}$
\end{proof}

\begin{mypro}
    若级数$\displaystyle{\sum\limits_{n=1}^\infty f_a(z)}$在区域$\mathbb{D}$的任一紧子集$K$上一致收敛,
    则称$\displaystyle{\sum\limits_{n=1}^\infty f_a(z)}$在$\mathbb{D}$上是内闭一致收敛的.
\end{mypro}
\begin{proof}
    类似于实值函数中的例子,函数项级数$\displaystyle{1+\sum\limits_{k=1}^\infty f_k(z),\ f_k(z)=z^k-z^{k-1}}$
    部分和为$z^k$显然下单位球上内闭一致收敛,但不一致收敛
\end{proof}

\begin{mypro}[Weierstrass I]
    设$D$是$\mathbb{C}$中的域.若$f_n\in C(D),n=1,2\dots\ ,$并且$\displaystyle{\sum\limits_{n=1}^\infty f_n(z)}$在$D$中内闭一致收敛到$f(z)$
    则$f\in H(D)$,并且级数$\displaystyle{\sum\limits_{n=1}^\infty f_n^{(k)}(z)}$在上内闭一致收敛到$f^{(k)}(z).\quad k\in\mathbb{N}$.
\end{mypro}
\begin{proof}
    任取$z_0\in D$,由于$D$为开集,故存在$\delta>0,\quad s.t.\ \overline{B(z_0,\delta)}\subset D$\\
    由定理\emph{4.1.4},$f\in C(\overline{B(z_0,\delta)})$在$B(z_0,\delta)$中任做一个可求长闭曲线$r$,
    由定理\emph{4.1.5}和定理\emph{3.2.4}得$\displaystyle{\int_rf(z)dz=\sum\limits_{n=1}^\infty\int_rf_n(z)dz=0}$.
    故由\emph{Morera}定理得$f\in H(B(z_0,\delta))$\\
    故$f\in H(D)$\\
    任取$\xi\in\partial B(z_0,\delta) ,\forall z\in B(z_0,\frac{\delta}{2})$.
    有$\displaystyle{\left|\frac{1}{(\xi-z)^{k+1}}\right|\leqslant(\frac{2}{\delta})^{k+1}}$\\
    由一致收敛性得,$\forall\epsilon>0,\ \exists N\in\mathbb{N}^*,\quad s.t.\ \forall n>N,\forall\xi\in\partial B(z_0,\delta)$,有:
    \begin{align*}
        &\left|\sum\limits_{j=1}^nf_j(\xi)-f(\xi)\right|
        <\frac{\epsilon}{k!\cdot\delta}(\frac{\delta}{2})^{k+1}\\
        \Rightarrow&\sum_{j=1}^n\frac{f_j(\xi)}{(\xi-z)^{k+1}}-\frac{f(\xi)}{(\xi-z)^{k+1}}<\frac{\epsilon}{k!\cdot\delta}
    \end{align*}
    故当$z\in B(z_0,\frac{\delta}{2})$时,有:
    \begin{align*}
        \left|\sum\limits_{j=1}^nf_j^{(k)}(z)-f^{(k)}(z)\right|
        &=\frac{k!}{2\pi}\cdot\left|\sum\limits_{j=1}^n\int_{|\xi-z_0|=\delta}\frac{f_j(\xi)d\xi}{(\xi-z)^{k+1}}
        -\int_{|\xi-z_0|=\delta}\frac{f(\xi)d\xi}{(\xi-z)^{k+1}}\right|\\
        &\leqslant\frac{k!}{2\pi}\int_{|\xi-z_0|=\delta}\left|\sum_{j=1}^n\frac{f_j(\xi)}{(\xi-z)^{k+1}}
        -\frac{f(\xi)}{(\xi-z)^{k+1}}\right|d\xi\\
        &<\epsilon
    \end{align*}
    故$\displaystyle{\sum_{j=1}^nf_j^{(k)}(z)\Rightarrow f^{(k)}(z),z\in B(z_0,\frac{\delta}{2})}$,
    对于$D$的任一紧集$K,K$有有限开球覆盖$SI_kS_{k=1}^n$,故$\displaystyle{\sum\limits_{j=1}^nf_j^{(k)}\Rightarrow f^{(k)}(z),z\in K}$.
\end{proof}

\begin{mypro}[Weierstrass \uppercase\expandafter{\romannumeral2}]
    设$D$为有界区域,对于函数列$\{f_n(z)\}$.有$f_n(z)\in H(D)\bigcap C(\overline{D})$
    且级数$\displaystyle{\sum\limits_{n=1}^\infty f_n(z)}$在$\partial D$上一致收敛,
    则$\displaystyle{\sum\limits_{n=0}^\infty f_n(z)}$在$\overline{D}$上一致收敛.
\end{mypro}
\begin{proof}
    $\forall\epsilon>0,\ \exists N,\ s.t.\ n\geqslant N, p\geqslant 1$时,有$\left|f_{n+1}(z)+f_{n+2}(z)+\dots+f_{n+p}(z)\right|<\epsilon.$
    对任意$z\in\partial D$均成立,由最大模定理,上述不等式在$\overline{D}$上成立,故级数在$\overline{D}$上一致收敛
\end{proof}

\section{幂函数}
幂级数:$\displaystyle{\sum\limits_{n=0}^\infty} a_n(z-z_0)^n=a_0+a_1(z-z_0)+a_2(z-z_0)^2+\dots+a_n(z-z_0)^n+\dots$\\
其中$a_0,a_1\dots$均为复常数,做平移$\omega=z-z_0$,得$\displaystyle{\sum\limits_{n=0}^\infty z^n=a_0+a_1z+\dots+a_nz^n+\dots(*)}$.
\begin{mypro}
    (收敛半径)(收敛圆).用数分中定义,略
\end{mypro}

\begin{mypro}
    级数$(*)$存在收敛半径$R=(\overline{\lim\limits_{n\to\infty}}\sqrt[n]{|a_n|})^{-1}$,则:\\
    (1)当$R=0$时,$\displaystyle{\sum\limits_{n\equiv0}^\infty a_nz^n}$只在$z=0$处收敛.\\
    (2)当$R=+\infty$时,$\displaystyle{\sum\limits_{n\neq0}^\infty a_nz^n}$在$\mathbb{C}$上收敛.\\
    (3)当$0<R<+\infty$时,$\displaystyle{\sum\limits_{n=0}^\infty a_nz^n}$在$\{z:|z|<R\}$中收敛.在$\{z:|z|>R\}$中发散.
\end{mypro}
\begin{proof}
    同数学分析中的讨论,此处略去
\end{proof}

\begin{mypro}[\emph{Abel} I]
    如果$\displaystyle{\sum\limits_{n=0}^\infty a_n\cdot z^n}$在$z=z_0\neq0$处收敛,则在$\{z:|z|<z_0\}$中内闭一致收敛.
\end{mypro}
%\noindent\emph{证明.}\ 
\begin{proof}
	设$K$为$\{z:|z|<|z_0|\}$中的一个紧集,取$r<|z_0|$,使得$k\subset B(0,r)$
\begin{wrapfigure}[4]{r}{2cm}
    \centering
    \includegraphics[width=3cm,height=2.7cm]{ch4_p3.png}
\end{wrapfigure}
由于$\displaystyle{\sum\limits_{n=0}^\infty a_nz_0^n}$收敛,故$\left|a_nz_0^n\right|\rightarrow0$
故存在$M\in\mathbb{R}.\ s.t.\ \left|a_nz_0^n\right|<M$\\
故当$z\in k$时,
$\displaystyle{\left|a_k\cdot a^k\right|=\left|a_n\cdot z_o^k\cdot \frac{z^k}{z_0^k}\right|\leqslant M\left(\left|\frac{z}{z_0}\right|\right)^k}$\\
由于$\displaystyle{\sum\limits_{n=0}^\infty\left|\frac{z}{z_0}\right|^n}$时收敛的,
故由\emph{Weierstrass}判别法得$\displaystyle{\sum\limits_{n=0}^\infty a_nz^n}$在$k$中一致收敛。\
%\\rightline{$\square$}
\end{proof}

\noindent 由定理\emph{4.2.3}以及\emph{Weierstrass}定理可以得到:

\begin{mypro}
    幂级数在其收敛圆内确定了一个全纯函数\par
    \qquad\quad\, 那么在收敛圆圆周上的收敛性如何呢?
\end{mypro}

\begin{eg}
    级数$\displaystyle{\sum\limits_{n=0}^\infty z^n}$的收敛半径为1,它在收敛圆周$|z|=1$上处处发散.
\end{eg}

\begin{eg}
    级数$\displaystyle{\sum\limits_{n=0}^\infty \frac{z^n}{n^2}}$的收敛半径为1,它在收敛圆周$|z|=1$上处处收敛.
\end{eg}

\begin{eg}
    级数$\displaystyle{\sum\limits_{n=0}^\infty \frac{z^n}{n}}$的收敛半径为1,
    它在收敛圆周$|z|=1$上在$z=e^{i\theta}(0<\theta<2\pi)$处收敛,在1处发散.
\end{eg}

\begin{proof}
    显然$z=1$时级数时发散的\\
    当$\theta\neq0$时$\displaystyle{\sum\limits_{n=0}^\infty\frac{z^n}{n}=
    \sum\limits_{n=0}^\infty\frac{\cos n\theta}{n}+i\cdot\sum\limits_{n=0}^\infty \frac{\sin n\theta}{n}}$
    由实数项级数的\emph{Dirichlet}判别法,\\
    得$\displaystyle{\sum\limits_{n=0}^\infty\frac{\cos n\theta}{n},\ \sum\limits_{n=0}^\infty \frac{\sin n\theta}{n}}$收敛\\
\end{proof}
由上述例子可知,收敛圆周上的收敛性无法确定\par 这也是我们下面要探讨的问题\par
设级数$\displaystyle{\sum\limits_{n=0}^\infty a_n(z-z_0)^n}$的收敛半径为$R$,
我们来研究$\displaystyle{\xi\in B(z_0,R),\sum\limits_{n=0}^\infty a_n(\xi-z_0)^n}$与和函数$f$的关系:
令$\displaystyle{\omega=\frac{z-z_0}{\xi-z_0}}$
故$\omega\in B(0,1)$,级数可改写为$\displaystyle{\sum\limits_{n=0}^\infty b_n\omega^n},b_n=a_n(\xi-z_0)^n$,
故新的幂级数的收敛半径为1,故以下我们在收敛半径为1的情况下做讨论.

\begin{mypro}[*]
    设$g$是定义在单位圆中的函数,$e^{i\theta_0}$是单位圆上的一点,记$S_\alpha(e^{i\theta_0})$为以$e^{i\theta_0}$为顶点,
    以$e^{i\theta_0}$点对应的极径在单位圆内侧向两侧张出的张角为$2\alpha$
    的角形区域$(\alpha<\frac{\pi}{2})$
\end{mypro}

%排的有问题
\begin{wrapfigure}[2]{r}{3cm}
    \centering
    \includegraphics[width=3cm,height=2cm]{ch4_p4_1st.png}
\end{wrapfigure}
\noindent \emph{若$z$在$S_\alpha(e^{i\theta_0})$中趋于$e^{i\theta_0}$时$g(z)$有极限$C$则称$g$在$\theta_0$处有非切向极限$C$,记为
$\displaystyle{\lim\limits_{\substack{z\to e^{i\theta_0}\\z\in S_\alpha(e^{i\theta_0})}}g(z)=C}$}
\\
\\
\begin{mypro}[*](\rm \emph{Abel} \uppercase\expandafter{\romannumeral2})
    设$\displaystyle{f(z)=\sum\limits_{n=0}^\infty a_nz^n}$的收敛半径为$R=1$,
    且级数在$z=1$处收敛于$S$,则$f$在$z=1$处有非切向极限$S$
\end{mypro}
\noindent\emph{证明.}
记$\displaystyle{\sigma_{n,\rho}=\sum\limits_{i=1}^p a_{n+i}}$由条件得$\displaystyle{\sum\limits_{n=0}^\infty a_n}$收敛.\\
故$\forall\epsilon>0 ,\ \exists$正整数$N,\  s.t.\ $当$n>N$时,对任意自然数$p$有$|\sigma_{n,p}|<\epsilon$\\
由于
\begin{align*}
    \sum\limits_{i=1}^pa_{n+i}\cdot z^{n+i}&=\sum\limits_{i=1}^{p-1}(\sigma_{n,i+1}-\sigma_{n,i})z^{n+1+i}+\sigma_{n,1}z^{n+1}\\
    &=\sum\limits_{i+1}^{p-1}\sigma_{n,i}z^{n+i}\cdot(1-z)+\sigma_{n,p}z^{n+p}\\
    &=z^{n+1}(1-z)\cdot\sum\limits_{i=1}^{p-1}(\sigma_{n,i}\cdot z^{i-1})+\sigma_{n,p}\cdot z^{n+p}
\end{align*}
故当$|z|<1,n>N$.时有:$\displaystyle{\left|\sum\limits_{i=1}^pa_{n+i}\cdot z^{n+i}\right|
<\epsilon\cdot|1-z|\cdot\sum\limits_{n=p}^\infty|z^n|+\epsilon=\epsilon\left(\frac{|1-z|}{1-|z|}+1\right)}$\\
任取$\displaystyle{z\in S_\alpha(1)\cap B(1,\delta)}$记$|z|=r,|1-z|=\rho$则由余弦定理得$r^2=1+\rho-2\rho\cos\theta$\\
\begin{wrapfigure}[3]{r}{3cm}
    \centering
    \includegraphics[width=3.5cm,height=2.0cm]{ch4_p4_2nd.png}
\end{wrapfigure}
故$\displaystyle{\frac{|1-z|}{1-|z|}=\frac{\rho}{1-r}=\frac{\rho(1+r)}{1-r^2}
\leqslant\frac{2\rho}{2\rho\cos\theta-\rho^2}=\frac{2}{2\cos\theta-\rho}}$\\
又由于$z\in B(1,\delta).$故$\rho<\delta<\cos\alpha<\cos\theta$故$\displaystyle{\frac{|1-z|}{1-|z|}<\frac{2}{\cos\alpha}}$\\
故$\displaystyle{\left|\sum\limits_{i=1}^pa_{n+i}\cdot z^{n+i}\right|<\epsilon(\frac{2}{\cos\alpha}+1)}$
,故$\displaystyle{\sum\limits_{n=0}^\infty a_nz^n}$在$\displaystyle{S_\alpha(1)\cap B(1,\delta)}$上一致收敛
设和函数为$f$,由一致收敛性得$f$在$\displaystyle{S_\alpha(1)\cap B(1,\delta)}$上连续,
故$\displaystyle{\lim\limits_{\substack{z\in S_\alpha(1)\\z\to 1}}f(z)=f(1)=S}$\\\rightline{$\square$}

%p1-4






%以下是p.9-13
\begin{mypro}
	设$f \in f(\omega)$,$\omega$为开集,$k\in\omega$为紧集,则存在$\omega-k$中的有线条可求长$r_{1},r_{2}......r_{n}$,使得
	\begin{equation*}
	f(z)=\sum_{j=1}^{N}\frac{1}{2\pi i}\int_{r_{n}}\frac{f(\xi)}{\xi-z}\di{\xi}
	\end{equation*}
\end{mypro}
\begin{proof}
	设$d=c \cdot d(k,\omega^{c})$,其中$c$为常数且$0<c<\frac{1}{\sqrt{2}}$,考虑边平行于轴的边长为$d$的正方形组成的网格。
	故在$\Omega-K$中必定包含一个完整的方格,并且与$k$相交的方格不与$\Omega$相交。
	
	设$Q$为与$K$相交的方格的全集,由于$K$是紧集,故$Q$为有限集,设$Q={Q_{1},Q_{2}......Q_{n}}$
	我们取$r_{1},r_{2}......r_{n}$为$Q$中的方格中所有只属于唯一方格的方格边框。如图
	%@TODO:Here should be the figure%
	
	任取$K$中非方格边框的一点$z$,存在$1 \le j \le m , s.t.  z \in Q_{j},z\notin Q_{k} (k \ne j)$
	故由\emph{Cauchy}积分公式得
	
	\begin{equation*}
	f(z)=\int_{\partial Q_{k}}\frac{f(\xi)}{\xi - z}\di{\xi}=\delta_{n-j}\cdot f(z)
	\end{equation*}
	故
	\begin{equation*}
	f(z)=\sum_{k=1}^{m}\int_{\partial Q_{k}}\frac{f(\xi)}{\xi -z}\di{\xi}
	\end{equation*}
	
	而由于在对每个$Q_{k}$的边界进行线积分的时候,对于除去$r_{1},r_{2}......r_{n}$之外的$Q$中方格边框均正向反向各积分一次
	,故
	\begin{equation*}
	f(z)=\int_{r_{k}}\dfrac{f(\xi)}{\xi - z}\di{\xi}
	\end{equation*}
	而可得$r_{k} \in \Omega /k,k=1,2.....m$,显然$r_{1},r_{2}......r_{n}$首尾相连得到了一条$\Omega /k$中的可求长闭曲线        
\end{proof}

\begin{eg}
	\color{blue}设$G={z\in C,0<\arg z<\frac{\pi}{4}}$,设$f\in H(G)\cup C(\bar G)$。若$f$在实轴上区间$[a,b]$恒为$0$,则$f(x)\equiv 0,x\in G$
	
	\color{black}
	\begin{proof}
	在$[a,b]$下方做延拓(如图),
	%@TODO:A figure here%
	
	有$\partial G \cap \partial D=[a,b]$。做$f_{1}:D\cup [a,b] \to {0} ,f_{1}(z)\equiv 0,z\in D\cup [a,b]$。
	故$\forall z\in [a,b]$有$f_{1}(z)=f(z)=0$。记$G\cup D \cup [a,b]=M$,故由解析延拓定理得存在$F(z)\in H(M)$
	且$f_{1}(z)=F(z),z\in D\cup [a,b],F(z)=f(z),z\in G$,显然存在一个M中的点列${Z_{n}}$,且有极限点$a\in M^{o}$,
	并且$F(Z_{n})=0。\forall n\in N。$由零点孤立性定理得$f(z)\equiv 0,z\in M$,故$f(z)\equiv 0,z\in G$
	
	类似于上一例题的推理过程,对于区域G,设$f\in H(G)\cap C(\bar{G})$,若存在一条可求长连续曲线$r \in \partial G$,
	有$f(z)\equiv 0,z\in r$,并且可以在r的异于G一侧做一区域E,使得$r\in \partial E$,且$E\cap G=\phi$,完全与上一例题中的过程相同可得$f(z)\equiv 0,\forall z\in G$
	\end{proof}
\end{eg}
\begin{eg}
	\color{blue}设$f_{n}(z)\in H(D)$,D为中区域,$\sum_{n=1}^{\infty}|f_{n}(z)|$在$D$内一致收敛,证明:$\sum_{n=1}^{\infty}|f_{n}(z)|$在D内内闭一致收敛。
	
	\color{black}
	\begin{proof}
	任取$D$中的紧集$K$,设$d=d(\partial D,k)$,记$\rho=\frac{d}{2}$
	
	故$z\in k$时
	\begin{equation*}
	\sum\limits_{i=1}^{n}\left|f'_{n+i}(z)\right|=\sum\limits_{i=1}^{p}\left|\int_{|\xi -z|=\rho }\dfrac{f_{i}(\xi)\di{\xi}}{(\xi -z)^{2}}\right|\cdot \dfrac{1}{2\pi}
	\end{equation*}
	\begin{equation*}
	\le \int_{|\xi -z|=\rho }\frac{\sum_{i=1}^{p}|f_{i}(\xi)|}{|(\xi -z)^{2}|}|\di{\xi}|\cdot \dfrac{1}{2\pi}\le \dfrac{\epsilon}{\rho}
	\end{equation*}
	\end{proof}
\end{eg}
\begin{eg}
	\color{blue}写出$e^{\frac{1}{1-z}}$,在$|z|<1$和$1<z<+ \infty$的部分\emph{Laurent}展式
	
	\color{black}设$f(z)=e^{z}$,显然$f$为常    函数,故$f(z)=\sum_{n=0}^{\infty}\frac{z^{n}}{n!}$,可以利用4.3.7将R上的展开到 D上
	可得$f(\frac{1}{1-z})=\sum\limits_{n=0}^{\infty}\frac{1}{n!}(\frac{1}{1-z})^{n}$,当$|z|<1$时,有如下错误解法。
	\begin{jie}{(错误)}
		\begin{equation*}
		\frac{1}{1-z}=\sum_{n=0}^{\infty}z^{n}=1+z+z^{2}+o|z|^{2} 
		\end{equation*}
		\begin{equation*}
		\Rightarrow f(\frac{1}{1-z})=1+\frac{1}{1-z}+\frac{1}{2}(\frac{1}{1-z})^{2}+...\dots
		\end{equation*}
		做到这里你就发现做不下去了,为什么?因为$(\frac{1}{1-z})^{n}=1+o(|z|)$
	\end{jie}
	\begin{jie}{(正确)}
		考虑展开$e^{\frac{z}{1-z}}$
		\begin{equation*}
		\dfrac{z}{1-z}=\sum\limits_{n=1}^{\infty}z^{n}=z+z^{2}+o|z|^{2} =o(1)
		\end{equation*}
		\begin{equation*}
		\Rightarrow f(\dfrac{z}{1-z})=1+\dfrac{z}{1-z}+\dfrac{1}{2}(\dfrac{z}{1-z})^{2})+o|z|^{2}
		\end{equation*}
		\begin{equation*}
		=1+(z+z^{2}+o|z|^{2})+\frac{1}{2}(z+z^{2}+o|z|^{2})^{2}+o|z|^{2}
		\end{equation*}
		\begin{equation*}
		=1+z+\dfrac{3}{2}z^{2}+o|z|^{2}
		\end{equation*}
		\begin{equation*}
		\Rightarrow e^{\dfrac{1}{1-z}}=e+ez+\dfrac{3e}{2}z^{2}+o|z|^{2}
		\end{equation*}
	\end{jie}
	当$1<|z|<+\infty$时,同上述讨论。
\end{eg}
\begin{eg}
	\color{blue}若$f(z)$在$0<|z-a|<R$上解析,$f$不为常值函数,且圆环上有一列点$z_{n}\rightarrow a$,有$f(z_{n})=0$。证明:$a$为$f(z)$的本性奇点。
	
	\color{black}
	\begin{proof}
	若$a$为可去奇点,则$\displaystyle\lim_{z\rightarrow a}f(z)$存在,故$\displaystyle\lim_{z\rightarrow a}f(z)=0$,补充定义$f(a)=0$故$f$在$|z-a|<R$上解析。由零点孤立性定理得$f(z)\equiv 0 , 0\le |z-a|<R$矛盾,
	
	若$a$为极点,故$\displaystyle\lim_{z\rightarrow a}f(z)=+\infty$,而$\displaystyle\lim_{n\rightarrow \infty}f(z_{n})=0$矛盾!
	\end{proof}	
\end{eg}
\begin{eg}
	\color{blue}设$f(z)$在圆环$0<r<|z-a|<R<+\infty$内解析,在闭圆环$r\le |z-a|\le R$上连续,且$f(Re^{i\Theta} )=0,(0\le \theta \le 2\pi)$。证明:$f(z)\equiv 0,(r<|z-a|<R)$
	\color{black}
	\begin{jie}{1}
		
		
		设$F(z)=f(\frac{r^{2}}{\bar{z}}),z\in \{z|\frac{r^{2}}{R} \le |z-a|\le r\}$,显然,$F(z)$在$\{z|\frac{r^{2}}{R} < |z-a|< r\}$
		上解析,在$\{z|\frac{r^{2}}{R} \le |z-a|< r\}$上连续。并且当$|z|=r$时,$F(z)=f(\frac{r^{2}}{\bar{z}})=f(z)$。$|z|=\frac{r^{2}}{R}$时,$F(z)=0$.
		
		设$D=\{z|\frac{r^{2}}{R} \le |z-a| < r\}$,故由解析延拓定理得存在$F_{1}\in H(D)\cap C(\bar{D})$。并且$F_{1}(z)=F(z),z\in \{z|\frac{r^{2}}{R} \le |z-a|\le r\}$,且$F_{1}(z)=f(z),z\in \{z|\frac{r^{2}}{R} \le |z-a|\le r\}$。
		
		又由于$\partial D =\{ z:|z|=\frac{r^{2}}{R} $或$R\}$。$\Rightarrow \forall z \in \partial D$有$F_{1}(D)=0$故最大模存理$F_{1}(z)\equiv 0 ,z\in \bar{D}$故$f(z)\equiv 0.(r<|z-a|<R)$
	\end{jie}
	\begin{jie}{2}
		
		
		同4.6.5第一个例子
	\end{jie}
\end{eg}
\begin{eg}
	\color{blue}若$f\in H(\{ z:0<|z-a|<k\}  )$且$\displaystyle\lim_{z \rightarrow a}(z-a)f(z)=0$。证明:$a$是$f(z)$的可去奇点。
	
	\color{black}
	\begin{proof}
	显然$a$是$g(z)=(z-a)f(z)$的可去奇点。利用定理4.5.1,再直接讨论\emph{Laurent}级数的系数即可
	\end{proof}

\end{eg}
\begin{eg}
	\color{blue}若$f$为整函数,并且$\infty$为$f$的可去奇点,证明$f$为常值函数。
	
	\color{black}
	\begin{proof}
	由定理4.5.1得$f$在$\infty$处的\emph{Laurent}展开式中幂次大于0的系数均为0。
	又由于$f$为整函数,故$f(z)$为常数。
	\end{proof}
\end{eg}
\begin{eg}
	\color{blue}设$f\in H(B(a,R )|\{ a\})$,且不为常值函数。记$u(z)=Ref(z)$,若$u(z)$有界,则$a$是$f$的可去奇点。
	\color{black}
	\begin{jie}{1}
		
		
		由\emph{Laurent}级数系数公式得$1\le n$时,$G_{n}=\frac{1}{2\pi i}\int_{|z-a|=\rho}f(z)(z-a)^{n-1}\di{z} \quad(0<\rho<R)$
		
		另一方面
		\begin{align*}
		G_{n}&=\frac{1}{2\pi i}\int_{|z-a|=\rho}f(z)(z-a)^{n-1}\di{z} \quad(0<\rho<R)\\
		&=\frac{1}{2\pi i}\int_{|z-a|=\rho}(\sum_{k=-\infty}^{\infty }\overline{C_{k}}\cdot\overline{(z-a)^{k}})(z-a)^{n-1}\di{z} \quad(0<\rho<R)\\
		&=\frac{1}{2\pi i}\int_{|z-a|=\rho}\sum_{k=-\infty}^{\infty }\overline{C_{k}} \cdot \rho^{n+k-1}(\frac{z-a}{|z-a|})^{n-1}\di{z} \quad(0<\rho<R)\\
		&=\frac{1}{2\pi}\sum_{k=-\infty}^{+\infty}\overline{C_{k}}\int_{0}^{2\pi}\rho^{k+n}e^{i(n-k)\theta}\di{\theta}=\overline{C_{n}}\rho^{2n}
		\end{align*}
		
		故我们可得$C_{-n}+\overline{C_{n}}\rho^{2n}=\frac{1}{\pi i}\int_{|z-a|=\rho}u(z)(z-a)^{n-1}\di{z}$
		
		故$|C_{-n}+\overline{C_{n}}\rho^{2n}|\le \frac{1}{\pi}|\int_{|z-a|=\rho}u(z)(z-a)^{n-1}\di{z}|<\frac{2\pi \rho}{\pi}\cdot M \cdot \rho^{n-1}=2M\rho^{n}(1\le n)$
		
		当$\rho\rightarrow 0$。得$C_{-n}=0(1\le n)$。故$a$为$f(z)$可去奇点。
	\end{jie}
	\begin{jie}{2}
		
		
		设$F(z)=e^{f(z)}$,故$F(z)$在$B(a,R)\{ a\}$上有界。故$|F(z)=e^{u(z)}\le e^{M}|$。故由定理4.5.1得。$a$为$F(z)$的可去奇点,补充定义$F(a)$使得$F(z)\in H(B(a,R))$
		
		又有$|F(z)|=e^{u(z)}\ge e^{-M}>0$。故在$a$的邻域内对数函数可取出单值解析分支。故$a$是$f(z)=\lg F(z)$的可去奇点。
	\end{jie}
	\begin{jie}{3}
			
		记$F(z)=\frac{f(z)}{f(z)-2M}$。故在$B(a,R)\{a\}$上有$|F(z)|\le 1$
		故$a$是$F(z)$的可去奇点。补充定义$F(a)$使得$F(z)\in H(B(a,R))$。
		
		由最大模原理得$|F(a)|<1$。而可解出$f(z)=\frac{2MF(z)}{1-F(z)}$。故由表达式得$\displaystyle\lim_{z\rightarrow a}f(z)$存在。故$a$为$f(z)$的可去奇点。
	\end{jie}
\end{eg}
