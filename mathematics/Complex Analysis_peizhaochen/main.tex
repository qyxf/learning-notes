\documentclass[
    b5paper,  % 默认为 a4paper
    %sourcefont, % 电脑上未装齐对应字体时,请勿打开该选项
    %opensource, % 输出开源信息
    decoration,  % 打开装饰
]{qyxf-book}  

\title{复分析笔记}
\subtitle{Notes on Complex Analysis}
\author{数试82\ 裴兆辰}
\typo{钱院学辅排版组}  % 排版人员信息
\date{2020 年 4 月 9 日}
\version{v1.0}
%\sourcepage{\url{https://github.com/qyxf/BookHub/}}

% 开启 opensource 选项时,以下信息必须填写
% \sourcepage{https://example.com/}

\graphicspath{{figs/}}

%定理环境配置
\usepackage{amsthm}%调用定理环境宏包
\newtheorem{mypro}{定理}[section]%定理环境
\newtheorem*{eg}{eg}%例题环境
\newtheorem*{jie}{解}%求解环境
%定理环境
\usepackage{wrapfig}

%微积分符号
\newcommand{\di}[1]{\mathrm{d}#1}
\newcommand{\p}[2]{\frac{\partial #1}{\partial #2}}
\newcommand{\pp}[2]{\frac{\partial ^2 #1}{\partial #2 ^2}}
\newcommand{\dy}[2]{\frac{\di{#1}}{\di{#2}}}
\newcommand{\ddy}[2]{\frac{\mathrm{d} ^2 #1}{\mathrm{d} #2 ^2}}

\begin{document}

\maketitle
\chapter*{编者说明}




\vspace{1em}
\makeatletter
\begin{flushright}
	\@author\\
	\@date
\end{flushright}
\makeatother


\cleardoublepage
\tableofcontents
\tableofcontents

\chapter{从矢量力学到分析力学}




\chapter{\emph{Lebesgue}测度}
\section{第一组}
\exercise{1}设$E\subset \R{R}{\ },$且存在$q:0<q<1,$使得对任一区间$(a,b),$都有开区间列$\{I_n\}:$ $$E\cap (a,b)\ \subset   \bigcup\limits_{n=1}^{\infty}I_n\ ,\quad \sum_{n=1}^{\infty}m(I_k)<(b-a)q\quad,$$试证明$m(E)=0$
\begin{proof}
	由于$m^*(E)=m^*\left( E\cap \bigcup\limits_{n=1}^{\infty}I_n\right)\leq \sum\limits_{n=1}^{\infty}m^*\left( E\cap I_n\right) ,$所以只需要证明$m^*\left( (a,b)\cap E\right)=0,$\par 
	由条件知存在$I_n(a_n,b_n),(n\in \R{N}{\ }),   \bigcup\limits_{n=1}^{\infty}I_n\supset\ E\cap (a,b)\ , \ s.t.\ \sum\limits_{n=1}^{\infty}\left(b_n-a_n \right)\leq q(b-a)\ .$下面我们对于每个$(a_n,b_n)$再做如上操作,所以存在$I_n^{(1)}=(a_n^{(1)},b_n^{(1)}),n\in\R{N}{\ }\ ,\ \bigcup\limits_{n=1}^{\infty}I_n^{(1)}\supset\ E\cap (a_k,b_k)\ ,$且$\sum\limits_{n=1}^{\infty}m(I_n^{(1)})<(b_k-a_k)q$.我们再对k求和,故我们得到了可数多个开集$\{J_n\}_{n=1}^{\infty},\quad s.t.\ \bigcup\limits_{n=1}^{\infty}J_n\subset\ \left( E\cap\bigcup\limits_{k=1}^{\infty}(a_k,b_k)\right) ,$且$\sum\limits_{n=1}^{\infty}m(J_n)\leq q\sum\limits_{n=1}^{\infty}(b_n-a_n)<q^2(b-a)$,故$m^*\left( (a,b)\cap E\right)<q^2(b-a).$\par 以此类推我们重复上面的过程,可得$m^*\left( (a,b)\cap E\right)=0$
\end{proof}


\exercise{2}设$A_1,A_2\subset\R{R}{n},A_1\subset A_2,A_1$是可测集,且$m(A_1)=m^*(A_2)<+\infty,$试证明$A_2$是可测集
\begin{proof}
	由于$A_1$可测,故$m^*(A_2)=m^*(A_1)+m^*(A_2\setminus A_1)$,故$A_2\setminus A_1$是零测集,故$A_2\setminus A_1$是可测集,故$A_2=A_1\cup A_2\setminus A_1$是可测集
\end{proof}


\exercise{3}设$A,B\subset\R{R}{n}$都是可测集,试证明$$m^*(A\cup B)+m^*(A\cap B)=m^*(A)+m^*(B)$$
\begin{proof}
	设$A,B$都是可测集,故$$m^*(A\cup B)=m^*(A)+m^*(B\setminus A)=m^*(B)+m^*(A\setminus B),$$$$ m^*(A)=m^*(A\cap B)+m^*(A\setminus B),$$$$m^*(B)=m^*(A\cap B)+m^*(B\setminus A).$$联立上述三个式子即得到所证明的式子
\end{proof}


\exercise{4}试问:是否存在闭集$F,F\subset [a,b]$且$F\neq[a,b]$,而$m(F)=b-a$
\begin{proof}
	不存在,假设存在,记为$F$,记$F_0=F\setminus\{a,b\}$,故$F_0\subsetneqq (a,b)$且$m(F_0)=b-a$.则$(a,b)\setminus F_0$为非空开集,故$(a,b)\setminus F_0$可写为$\R{R}{\ }$中的至少一个开区间的并,故存在开区间$(s,t)$,满足$(s,t)\subset (a,b)\setminus F_0$.故$m^*\left((a,b)\setminus F_0 \right)\geq \ s-t\ >0, $显然这个与$m(F_0)=b-a$矛盾!故满足条件的闭集不存在!
\end{proof}

\exercise{7}设$\{E_k\}$是$\R{R}{n}$上的可测集合列,若$m\left(\bigcup\limits_{k=1}^{\infty}E_k \right)<+\infty $,试证明$$m\left(\varlimsup_{k \to +\infty}E_k \right)\geq \varlimsup\limits_{k \to +\infty}m(E_k) $$
\begin{proof}
	与推论2.9的证明类似,此处省略
\end{proof}



\exercise{8}设$\{E_k\}$是$[0,1]$中的可测集合列,$m(E_k)=1(k=1,2,...)$,试证明$m\left( \bigcap\limits_{k=1}^{\infty}E_k\right)=1 $
\begin{proof}
	记$F_k=[0,1]\setminus E_k$,由于$E_k$可测,故$F_k$为零测集.故$\bigcup\limits_{k=1}^{\infty}F_k$也是零测集\par 
	又由题目条件得$\bigcap\limits_{k=1}^{\infty}E_k$是可测集,故用$[0,1]$做实验集即得到题目结果
\end{proof}


\exercise{9}设$E_1,E_2,...E_k$是$[0,1]$中的可测集,且有$\sum\limits_{i=1}^{\infty}m(E_i)>k-1,$试证明$m\left(\bigcap\limits_{i=1}^{\infty}E_i \right)>0 $
\begin{proof}
	与8类似,此处省略
\end{proof}


\exercise{12}设$\{B_k\}$是$\R{R}{n}$中递减可测集列,$m^*(A)<\infty$,令$E_k=A\cap B_k\ (k=1,2,...),E=\bigcap\limits_{k=1}^{\infty}E_k,$试证明$\li{k\to\infty}m^*(E_k)=m^*(E)$
\begin{proof}
	由于$\bigcup\limits_{k=1}^{\infty}E_k$可测,故$m^*(A)=m^*(\bigcup\limits_{k=1}^{\infty}E_k)+m^*\left(A\cap (\bigcup\limits_{k=1}^{\infty}E_k)^c \right) $\par
	又由于$E_k$可测,故$m^*(A)=m^*(E_k)+m^*(A\cap B_k^c),$由于$\{A\cap B_k^c\}$是递增集合列,在此式两端令$k\to\infty$,由推论2.17得$\li{k\to\infty}m^*(A\cap B_k^c)=m^*\left( A\cap(  \li{k\to\infty}B_k) ^c\right)=m^*\left( A\cap(  \bigcap\limits_{k=1}^{\infty}B_k) ^c\right) $ ,直接带入即得证
\end{proof}


\exercise{14}试证明点集$E$可测的充分必要条件是,对任给的$\epsilon>0$,存在开集$G_1,G_2:G_1\subset E,G_2\subset E^c$,使得$m(G_1\cap G_2)<\epsilon$
\begin{proof}
\\	$"\Rightarrow"$\par
	由定理2.13得,$\forall \epsilon>0$,存在开集$F$以及闭集$G$使得$F\supset E\supset G$,并且$m(F\setminus E)<\displaystyle{\frac{\epsilon}{2}},m(E\setminus G)<\displaystyle{\frac{\epsilon}{2}},$取$G_1=F,G_2=G^c$,故$m(G_1\cap G_2)<\epsilon$\\
	$"\Leftarrow"$\par 
	假设存在这样的集合$G_1,G_2$,记$F=G_2^c$,故显然$F\subset E$且$m(G_1\setminus F)<\epsilon,$故$m(G_1\setminus E)<\epsilon$,故E为可测集
\end{proof}



\exercise{15}设$E\subset [0,1]$是可测集,且有$m(E)\geq \epsilon>0,\quad x_i\in[0,1],\ i=1,2,...,n$,其中$\displaystyle{n>\frac{2}{\epsilon}}$,试证明$E$中存在着两个点其距离等于$\{x_1,x_2,...,x_n\}$中某两个点间的距离.
\begin{proof}
	设$E_k=E+\{x_k\},(k=1,2...,n),$故由定理2.18得$m(E_k)=m(E)\geq\epsilon.$故$\sum\limits_{k=1}^{\infty}m^*(E_k)\geq n\epsilon>2,$而$E_k\subset[0,2]$,故一定存在$i,j\in\{1,2,...,n\}$且$i\neq j,\ s.t.\ E_i\cap E_j\neq \emptyset.$任取$x\in E_i\cap E_j.$故$x-x_i,x-x_j\in E,$g故$|(x-x_i)-(x-x_j)|=|x_i-x_j|$
\end{proof}



\exercise{16}设$W$是$[0,1]$中的不可测集,试证明存在$\epsilon:0<\epsilon<1,$使得对于$[0,1]$中任一满足$m(E)\geq\epsilon$的可测集$E$,$W\cap E$是不可测集
\begin{proof}
	故$\forall 1>\epsilon>0,$存在$E\subset[0,1],m(E)\leq \epsilon,$有$W\cap E$可测,故我们可以取$\{\epsilon_n\},\ s.t.\ \epsilon_n\to 1$.\par 设对于$\epsilon_n$,满足上述条件的集合记为$E_n$,记$E=\bigcup\limits_{n=1}^{\infty}E_n,$故$E$可测,且$m(E)=1$,故$m^*\left( W\cap ([0,1]\setminus E)\right) =0 ,$故$W\cap \left( [0,1]\setminus E\right) $可测.而另一方面$W=\left(\bigcup\limits_{n=1 }^{\infty}(W\cap E_n)\right)\cup \left( W\cap ([0,1]\setminus E)\right) , $故W可测,矛盾!
\end{proof}



\section{第二组}
\exercise{1}设$\{r_n\}$是$\R{R}{1}$中的全体有理数,令$$G=\bigcup_{n=1}^{\infty}\left( r_n-\frac{1}{n^2},r_n+\frac{1}{n^2}\right), $$试证明对$\R{R}{1}$中的任一闭集$F$,有$m(G\triangle F)>0$
\begin{proof}
	故$G$为开集,显然$G\setminus F$为开集.\par 若$G\setminus F\neq\emptyset$,则$G\setminus F$为$\R{R}{\ }$上的非空开集,则$m(G\setminus F)>0$,故$m(G\triangle F)>0$\par 
	若$G\setminus F=\emptyset$,则$G\subset F$.若$F\neq \R{R}{\ }$,则取$x\in F^c,$显然x为无理数,并且为$F^c$的内点,故存在$\delta>0$,使得$B(x,\delta)\subset F^c.$然而$\R{Q}{\ }$在$\R{R}{\ }$中稠密,故$B(x,\delta)$中必包含有理数,矛盾!故$F=\R{R}{\ }$,$m(F\setminus G)=m(G^c)$.而$m(G)\leq \sum\limits_{n=1}^{\infty}\displaystyle{\frac{1}{n^2}}<+\infty$.故$m(G^c)>0$\par 综上$m(G\triangle F)>0$
\end{proof}


\exercise{2}设$E\subset[a,b]$是可测集,$I_k\subset[a,b](k=1,2,...)$是开区间列,满足$m(I_k\cap E)\geq \displaystyle{\frac{2}{3}|I_k|}(k=1,2,...),$试证明$$m\left(\left(\bigcup_{k=1}^{\infty}I_k  \right)\cap E  \right)\ \geq \ \frac{1}{3}m\left(\bigcup_{k=1}^{\infty}I_k \right)  $$
\begin{proof}
	略
\end{proof}



\exercise{3}设$\{E_n\}$是$[0,1]$中的互不相同的可测集合列,且存在$\epsilon>0,m(E_n)\geq \epsilon(n=1,2,...),$试问是否存在子列$\{E_{n_i}\}$,使得$m\left(\bigcap\limits_{i=1}^{\infty}E_{n_i} \right)>0 $
\begin{proof}
	设$E_{2n}=\left[ 0,\displaystyle{\frac{1}{2}+\frac{1}{2n}}\right],E_{2n-1}=\left[ \displaystyle{\frac{1}{2}-\frac{1}{2n+1}},1\right],n\in \R{N}{*} .$故$m(E_k)>\displaystyle{\frac{1}{2}}.$而对于任一子列$\{E_{n_i}\}$,$\bigcap\limits_{i=1}^{\infty}E_{n_i}=\left\lbrace \displaystyle{\frac{1}{2}}\right\rbrace $
\end{proof}


\exercise{4}设$\{E_n\}$是$[0,1]$中的可测集合列,且满足$\varlimsup\limits_{n \to +\infty}m(E_n)=1,$试证明对$0<a<1$,必存在$\{E_{n_k}\}$,使得$m\left(\bigcap\limits_{k=1}^{\infty}E_{n_k} \right)>a $
\begin{proof}
	故$\forall \epsilon>0$,存在子列$\{E_{n_k}\}$,使得$m(E_{n_k})>\displaystyle{1-\frac{\epsilon}{2^k}}.$记$F_n=[0,1]\setminus E_n,$故$m(F_k)<\displaystyle{\frac{\epsilon}{2^k}}.$故$m\left(\bigcap\limits_{k=1}^{\infty}E_{n_k} \right)=1-m\left(\bigcup\limits_{k=1}^{\infty}F_{n_k} \right)>1-\sum\limits_{k=1}^{\infty}m(F_{n_k})>1-\epsilon $ 
\end{proof}

\exercise{5}设$m^*(E)<\infty$,试证明存在$G_\delta$集$H:H\subset E$,使得对于任一可测集$A$,都有$m^*(E\cap A)=m(H\cap A)$
\begin{proof}
	对于任一可测集$A$,有$m^*(E\cap A)=m^*(E)-m^*(E\setminus A).$由定理2.15得存在 $G_\delta$集$H_0=\bigcap\limits_{k=1}^{\infty} I_k\supset E,$其中$I_k$是开集,使得$m(H_0)=m^*(E),$故\\ $m^*(E\cap A)\geq m(H_0)-m^*\left(\bigcap\limits_{k=1}^{\infty}(I_k\setminus A) \right)=M(h_0\cap A) .$\par 而$(H_0\cap A)\setminus(E\cap A)\subset H_0\setminus E$,故$(H_0\cap A)\setminus(E\cap A)$是零测集,是可测集.故$m^*(E\cap A)=m(H_0\cap A)$
\end{proof}

\exercise{6}设$A,B\subset\R{R}{\ },A\cup B$是可测集,且$m(A\cup B)<\infty,$若$m(A\cup B)=m^*(A)+m^*(B)$,试证明$A,B$均为可测集
\begin{proof}
	做$A,B$的等测包$A_1,B_1$,故$A_1,B_1$是$G_\delta$集,$A_1\supset A,B_1\supset B,m(A_1)=m^*(A),m(B_1)=m^*(B_1)$.故$$m(A_1)+m(B_1)\geq m(A_1\cup B_1)\geq m^*(A\cup B)=m^*(A)+m^*(B)=m(A_1)+m(B_1).$$\par 故上述不等号均取为等号,故$m^*\left( (A_1\cup B_1)\setminus(A\cup B)\right) =0,$故$A_1\setminus A,B_1\setminus B$均为零测集,故$A,B$均为可测集
\end{proof}
\chapter{紫外-可见吸收光谱}
\begin{introduction}
    \item 紫外可见光谱及其产生机理(理解)
    \item 分子对紫外可见光谱的影响(了解)
    \item 紫外可见光谱仪(熟悉)
    \item 紫外可见吸收分析(熟悉)
\end{introduction}
\section{紫外可见光谱及其产生机理}
\subsection{紫外可见光谱}
\begin{definition*}{分子光谱}
    处于气态或溶液中的分子,当发生能级跃迁时,发射或吸收一定频率范围内电磁辐射所组成的带状光谱。
\end{definition*}
\begin{definition*}{紫外-可见吸收光谱法}
    利用紫外-可见分光光度计测量物质对紫外-可见光的吸收程度(即吸光度)或紫外-可见吸收光谱来确定物质的组成、含量等,推理物质结构的分析方法。紫外-可见分光光度法又分为比色法和分光光度法,均属于微量分析。
\end{definition*}
特点:
\begin{itemize}
    \item 灵敏度高,可用于微量组分的测定;
    \item 准确度能满足微量组分测定的要求,相对误差$2\%-5\%$;
    \item 测量仪器相对简单,价格便宜;
    \item 应用广泛,可测定无机物和有机物、可定量或定性分析、可同时测定单一组分或多组分等,还可用于测定络合物的组成、有机酸(碱)及络合物的平衡常数。
\end{itemize}
\subsection{紫外可见光谱的研究对象}
研究对象:在$200\sim 380\mathrm{nm}$的近紫外光区和$380\sim 780\mathrm{nm}$的可见光区有吸收的物质。
\begin{note}
    不能测量$<200\mathrm{nm}$的原因:氧气、氮气、水蒸气存在吸收区,造成严重干扰
\end{note}
\subsection{紫外可见光谱的产生原理}
紫外-可见吸收光谱是由于分子中价电子的跃迁而产生的。
\subsubsection*{有机化合物}
对于有机化合物来说,其分子中存在的电子跃迁有四种:$\pi \rightarrow \pi^{\star}$,$n\rightarrow \pi^{\star}$,$\sigma \rightarrow  \sigma^{\star}$,$n\rightarrow \sigma^{*}$。其中$n\rightarrow\sigma^{*}$能量大,波长短,吸收峰处于远紫外区。在紫外-可见吸收光谱中,可能观测到的跃迁有:
\begin{itemize}
    \item 非共轭体系:所有可能的跃迁中,只有$n\rightarrow \pi^{*}$的跃迁的能量足够小,相应的吸收光波长在$200\sim 800 \mathrm{nm}$范围内,其他的跃迁能量都太大,它们的吸收光波长均在$200 \mathrm{nm}$以下,无法观察到紫外光谱
    \item 共轭体系:除$n\rightarrow n^{*}$外,还可能有$\pi \rightarrow \pi^{*}$跃迁,它们的吸收光可能落在紫外-可见区。
\end{itemize}
\subsubsection*{无机化合物}
对于无机化合物来说,存在两种跃迁:
\begin{itemize}
    \item 电荷转移跃迁:
    \begin{definition*}{电荷转移跃迁}
        配合物分子吸收辐射后,分子中的电子从主要定域在金属离子$\ce{M}$的轨道上转移到配位体$\ce{L}$的轨道上或按相反方向转移,这种跃迁叫电荷转移,产生的吸收光谱叫电荷转移光谱。
    \end{definition*}
    \begin{note}
        电荷转移越迁摩尔吸光系数大(>$10^{5}$),可用于定量分析
    \end{note}
    电荷转移越迁有如下三种类型
    \begin{itemize}
        \item $\ce{L}\rightarrow \ce{M}$电荷转移:较易还原的金属离子+较易氧化的配体
        \begin{example}
            \ce{TiI4} 紫色,\ce{HgI2}红色,\ce{AgI}黄色,\ce{VO_{4}^{3-}}无色,\ce{CrO_{4}^{2-}}黄色,\ce{MnO_{4}^{2-}}紫色
        \end{example}
        \item $\ce{M}\rightarrow \ce{L}$电荷转移:
        \begin{itemize}
            \item 金属离子容易被氧化,如\ce{Ti^{3+}}, \ce{Fe^{2+}},\ce{ V(II)}, \ce{Cu(I)}。
            \item 配体易被还原,具有空位反键轨道,可接受从金属离子中转移的电子:
            \begin{figure}[h]
                \centering
                \includegraphics[width=10cm]{image/chp3_L.png}
                \label{fig:chp3L}
                \caption{易被还原的配体}
            \end{figure}
        \end{itemize}
        \item $\ce{M}\rightarrow \ce{M}$电荷转移:配合物中含有两种不同氧化态的金属离子,电荷在不同氧化态金属离子之间转移。
        \begin{figure}[h]
            \centering
            \includegraphics[width=10cm]{image/chp3_MM.png}
            \label{fig:chp3MM}
            \caption{\ce{M -> M}电荷转移示例}
        \end{figure}
    \end{itemize}
    \item 配位场跃迁
    \begin{definition*}{配位场跃迁}
        过渡元素、镧系和锕系元素在真空下,原子、离子的\ce{d}轨道和\ce{f}轨道是简并的,在配位体场影响下,简并能级发生分裂成不同能量组轨道,包括\ce{d-d}跃迁,\ce{f-f}跃迁。
    \end{definition*}
    \begin{itemize}
        \item \ce{d-d}跃迁:一些\ce{d}电子层未充满的第四周期、第五周期的过渡金属元素的吸收光谱,它们的吸收光谱往往比较宽,且容易受环境因素的影响,例如水合铜离子是浅蓝色的,它 的氨络合物是深蓝色的;
        \item \ce{f-f} 跃迁:大多数镧系和锕系元素的离子在紫外-可见区有吸收,并且吸收峰比较窄。
        \begin{note}
            \ce{d-d}跃迁的吸收峰较宽,\ce{f-f}跃迁的吸收峰较窄的原因:
            \begin{itemize}
                \item 外层\ce{d}电子跃迁时容易受外界环境(溶剂、配位体)的影响
                \item 电子在内层,受外层轨道电子的屏蔽,不易受溶剂、配位体影响。
            \end{itemize}
        \end{note}
    \end{itemize}
\end{itemize}

\section{影响紫外-可见光谱的因素}
%修改了排版,待考虑

\begin{definition*}{生色团}
    在某一段光波内发生吸光的基团,
    \begin{example}
        碳碳共轭结构、含有杂原子的共轭结构、\ce{C=O}、\ce{C=C}、\ce{C#C}等。
    \end{example}
\end{definition*}
\begin{definition*}{助色团}
    具有非键电子的基团连在双键或共轭体系上,形成非键电子与$\pi$电子的共轭,即$p-\pi$共轭,使电子活动范围增大,吸收向长波方向位移,使颜色加深的基团。这些基团本身在200$\mathrm{nm}$以上不产生吸收,但这些基团的存在可以增强生色团的生色能力。
    \begin{example}
        $\ce{-OH}$,$\ce{-OR}$,$ \ce{-NH_{2}}$,$ \ce{-NR_{2}}$,$ \ce{-SR}$, 卤素等
    \end{example}
\end{definition*}

\subsection{取代基的影响} 
共轭双键两端有容易使电子流动性增强的基团(给电子基或吸电子基)时,极化现象增强,均使得$\lambda_{max}$红移,$\varepsilon$(摩尔吸光系数)增大。当给电子基与吸电子基同时存在时,产生分子内电荷转移吸收,同样使$\lambda_{max}$红移,$\varepsilon$增大。

\subsection{红移与蓝移}
\begin{definition*}{红移}
	由于取代基或溶剂的影响,最大吸收峰向长波方向移动的现象称为红移现象 (red shift)。
\end{definition*}
\begin{definition*}{紫移}
	又叫蓝移(blue shift),是指物质受取代基或溶剂的影响,最大吸收峰向短波方向移动的现象。
\end{definition*}

\begin{itemize}
    \item 共轭效应的影响:$\pi -\pi$共轭,吸收光红移,最大吸收波长由远紫外向近紫外移动;共轭双键数越多,红移越强。
    \begin{itemize}
        \item  π电子共轭体系增大,$\lambda_{max}$红移,$\varepsilon$(摩尔吸光系数)增大;
        \item 空间阻碍使共轭体系破坏,$\lambda_{max}$蓝移,$\varepsilon$(摩尔吸光系数)减小:取代基越大,分子共平面性越差,最大吸收波长蓝移,摩尔吸光系数降低。
    \end{itemize}
    \item 超共轭效应:烃基与$\pi$体系相连,$\pi-\sigma$ 超共轭,紫外吸收红移;连接的烃基基团越大,红移越强。
    \begin{figure}[h]
        \centering
        \includegraphics[width=10cm]{image/chp3_ultra.png}
        \label{fig:chp3ultra}
        \caption{超共轭效应}
    \end{figure}
    \item 溶剂的影响:由于受到溶剂和溶质分子间形成氢键、偶极极化等影响,会使溶质吸收波长发生位移:
    \begin{itemize}
        \item 溶剂极性增大时,$\pi \rightarrow \pi^{*}$跃迁吸收带红移,$n\rightarrow \pi^{*}$跃迁吸收带蓝移;
        \item 质子性溶剂容易与吸光分子形成氢键,生色团为质子供体时吸收峰红移,生色团为质子受体时吸收峰蓝移。
    \end{itemize}
\end{itemize}

\subsection{增色与减色效应}
\begin{definition*}{增色和减色效应}
	指由于化合物结构改变或其他原因,使吸收强度增加(或减弱)的效应,称为增色(减色)效应。
\end{definition*}

\begin{note}
	在吸收光谱中,$\varepsilon $(摩尔吸光系数)值与电子跃迁前后所占据轨道的能差及它们的相互位置有关,轨道间能差越小,分子越处于共平面时,电子的跃迁概率较大,$\varepsilon $值增大,吸收强度增加;反之$\varepsilon$值减小,吸收强度减弱。
\end{note}

\section{紫外-可见分光光度计}

结构:由\textbf{光源、单色器、样品室和检测器}构成。光从光源发出后,经单色器分光,穿过狭缝进入样品室,最终由检测器接收光信号并转换为电信号。
\begin{itemize}
    \item 光源:提供入射光,要求发射连续的具有足够强度并且稳定的紫外和可见光,辐射强度随波长变化尽可能小,如:钨丝灯(可见光区)、氢灯和氘灯(紫外区);
    \item  单色器:将复合光色散成单色光的光学装置。一般由狭缝、色散元件及透镜系统组成。最常用的色散元件是光栅和棱镜;
    \item  样品室:用于盛放试液的装置,\textcolor{red}{可见光区使用玻璃吸收池},\textcolor{blue}{紫外光区使用石英吸收池};
    \item  检测器:将光信号转变成电信号的装置,如光电管、光电倍增管、光电二极管阵列检测器等。
\end{itemize}
紫外可见光谱仪的分类
\begin{itemize}
    \item 单光束分光光度计:
    \begin{figure}[h]
        \centering
        \includegraphics[width=10cm]{image/chp3_single_beam.png}
        \label{fig:chp3singlebeam}
        \caption{单光束分光光度计}
    \end{figure}
    \begin{note}
        单光束光度计缺点:
        \begin{itemize}
    \item  操作麻烦,需扣除背景;
    \item  不能进行吸收光谱的自动扫描;
    \item  光源不稳定性影响测量精密度。
\end{itemize}
    \end{note}
    \item 双光束分光光度计:
    \begin{itemize}
        \item  单波长双光束分光光度计
        \begin{figure}[h]
            \centering
            \includegraphics[width=10cm]{image/chp3_single_wl.png}
            \label{fig:chp3singlewl}
            \caption{单波长双光束分光光度计}
        \end{figure}
        
        从光源发出的光经分光后分成两束,交替通过参比池和样品池,测得的是透过样品溶液和参比溶液的光信号强度之比,克服了光源不稳定引起的误差,实现了对全波段、快速自动吸收光谱扫描。
        \item 双波长双光束分光光度计
        \begin{figure}[h]
            \centering
            \includegraphics[width=10cm]{image/chp3_double_wl.png}
            \label{fig:chp3doublewl}
            \caption{双波长双光束分光光度计}
        \end{figure}
        
        可消除光谱重叠干扰和背景干扰,主要特点是可以降低杂散光,光谱精度高。
    \end{itemize}
\end{itemize}
\section{紫外可见吸收光谱}
\subsection{紫外可见吸收分析的定量基础}
\begin{theorem*}{朗伯比尔定律}
    当一束平行单色光垂直通过某一均匀非散射的吸光物质时,其吸光度$\mathrm{A}$与吸光物质的浓度$\mathrm{c}$及吸收层厚度$\mathrm{b}$成正比,而与透光度T呈反相关
    \begin{equation*}
        A=\varepsilon bc =\lg(\frac{1}{T})\\
        T=\dfrac{I}{I_0}
    \end{equation*}
\end{theorem*}
\begin{note}
$A$:吸光度,无量纲

$\varepsilon$:摩尔吸光系数,指的是浓度为1$ \mathrm{mol/L}$的物质在1$\mathrm{cm}$厚的吸收池内产生的吸光度,单位为$\mathrm{L/(mol\cdot cm)}$;

$b$:液层厚度,单位为$\mathrm{cm}$;

$c$:溶液中吸光物质的浓度,单位为$\mathrm{mol/L}$。
\end{note}

吸收定律的性质:
\begin{itemize}
    \item 吸收定律具有加和性:如果某一试液中多个组分对同一波长的光有吸收作用,则总吸光度等于各组分的吸光度之和;
    \item 吸收定律只适合单色光:描述吸光度值时需说明光源的波长;
    \item 吸收定律因化学反应而偏离:因解离、络合等原因,待测物质并不都以能对特定频率辐射进行有效吸收的形态存在,可能导致吸收定律结果的偏离。
\end{itemize}

\subsection{紫外-可见吸收光谱的误差}
\begin{itemize}
    \item  溶液偏离朗伯-比尔定律所引起的误差:
    \begin{figure}[h]
        \centering
        \includegraphics[width=6cm]{image/chp3_billlaw_offset.png}
        \label{fig:chp3offset}
        \caption{溶液偏离朗伯-比尔定律}
    \end{figure}
采用工作曲线的直线段测定待测溶液浓度。减少由入射光为非单色光所引起的误差。利用试剂空白以及确定适宜的浓度范围来减少溶液本身所引起的误差。
    \item  仪器误差:
    \begin{itemize}
        \item 机械系统:吸收池的质量,检流计的灵敏度;
        \item 光学系统:光源不稳定,棱镜的性能,光电管质量等。
    \end{itemize}
    \item  操作误差:由所采用的实验条件与正确的条件有差别引起,
    如:显色条件和测量条件掌握不够好。
\end{itemize}

\subsection{紫外-可见吸收光谱测量条件的选择}
\begin{itemize}
    \item 入射光波长的选择:
    
    以最大吸收波长为入射光波长。此处波长的吸光系数最大,测定的灵敏度更高,且此处波长处吸光度有一小的平坦区,能减少和消除由于单色光的不纯而引起的对朗伯-比尔定律的偏移,从而提高测定的准确度。
    
    \item 吸光度读数范围的选择:
    \begin{equation*}
        \frac{\Delta c}{c}=\frac{0.434}{T \ln T}\Delta T
    \end{equation*}
    透射率$T$为$20\%\sim 65\%$时,测量误差$<2\%$。
    可通过控制溶液的浓度来控制透射率。
    
    \item 参比溶液的选择:
     \begin{itemize}
        \item 纯溶剂空白:当试液、试剂、显色剂均为无色时,可直接用纯溶剂(或蒸馏水)作为参比溶液;
        \item  试剂空白:试液无色,而试剂或者显色剂有色时,可在同一显色反应条件下,加入相同量的显色剂和试剂,稀释同一体积,以此作为参比溶液;
        \item  试液空白:试剂和显色剂均为无色,试液中其他离子有色时,可采用不加显色剂的试液作为参比溶液。
    \end{itemize}

\end{itemize}    
\chapter{解析函数的{\rm Taylor}展开及其应用}
\section{Weierstrass定理}
    对于复数上的数列,其收敛性可以类似于$\mathbb{R}^2$中点集的收敛性即可得到.同样的我们有柯西收敛原理,此处略去\par
    对于复数上的数项级数$\displaystyle{\sum\limits_{n=0}^\infty z_n}$, 
    类似于实数中的常数项级数$\displaystyle{\sum\limits_{n=0}^\infty a_n}$.
    可以定义绝对收敛和条件收敛,以及级数极限的存在性和柯西收敛原理.\par
    下面讨论对于复变函数上的函数项级数$\displaystyle{\sum\limits_{n=1}^\infty f_n(z)}$:

\begin{mypro}
    设$\displaystyle{\sum\limits_{n=1}^\infty f_n(z)}$是定义在$\mathbb{E}$上的级数.
    我们说$\displaystyle{\sum\limits_{n=1}^\infty f_n(z)}$在$\mathbb{E}$上一致收敛到
    $f(z)$.$(\mbox{记为}\ \displaystyle{\sum\limits_{k=1}^n f_k(z)\Rightarrow f(z)})$,
    指$\forall\epsilon>0,\exists.N\in \mathbb{N}^*\quad s.t.\ \forall n>N$.有$\left|S_n(z)-f(z)\right|<\epsilon, \forall z\in E$,
    其中$\displaystyle{S_n(z)=\sum\limits_{k=1}^n f_k(z)}$.
\end{mypro}

\begin{mypro}
    级数$\displaystyle{\sum\limits_{n=1}^\infty f_n(z)}$在$\mathbb{E}$上一致收敛的充要条件是
    $\forall\epsilon>0,\exists.N\in \mathbb{N}^*\quad s.t.\ \forall n>N.\forall p\in N$
    有$\displaystyle{\left|\sum\limits_{i=1}^pf_{n+i}(z)\right|<\epsilon}$.对$\forall z\in\mathbb{E}$均成立.
\end{mypro}
\begin{proof}
    略.
\end{proof}

\begin{mypro}[函数项级数的weierstass判别法,略]
\end{mypro}

\begin{mypro}
    设级数$\displaystyle{\sum\limits_{n=1}^\infty f_n(z)\Rightarrow f(z)}.z\in \mathbb{E},\forall n\in\mathbb{N}^*, f_n\in C(E)$.
    则$f\in C(E)$.
\end{mypro}
\begin{proof}
    $\forall\epsilon>0.\exists N\in\mathbb{N}.\quad s.t.\ n>N$时
    $\displaystyle{\left|f(z)-S_n(z)\right|<\frac{\epsilon}{3}\ \forall z\in\mathbb{E}}$.\\
    对于给定的大于$N$的$n_0$.显然$S_{n_0}\in C(\mathbb{E})$.现任取一个$a\in \mathbb{E}$.故$S_{n_0}$在$a$处连续.\\
    故$\forall\epsilon>0,\exists\delta>0.\quad s.t.\ \forall z\in B(z,\delta)$有
    $\displaystyle{\left|f(z)-f(a)\right|<\frac{\epsilon}{3}}$.\\
    于是$z\in\mathbb{E}\bigcap B(a,\delta)$时有:\\
    $\displaystyle{\left|f(z)-f(a)\right|\leqslant\left|f(z)-S_{n_0}(z)\right|
    +\left|S_{n_0}(z)-S_{u_0}(a)\right|+\left|S_{u_0}(a)-f(a)\right|<\epsilon}$
\end{proof}

\begin{mypro}
    设级数$\sum\limits_{n=1}^\infty f_u(z)$.在可求长曲线$\gamma$上一致收敛到$f(z)$,若$\forall n\in\mathbb{N}^*,f_n\in C(\gamma),$
    则$\displaystyle{\int_{\gamma}f(z)dz=\sum\limits_{n=1}^\infty\int_{\gamma}f_n(z)dz}$.
\end{mypro}
\begin{proof}
    由定理\emph{4.1.4}.\ $f\in C(\gamma)$.\\
    由于$f_k\Rightarrow f$.故$\forall\epsilon>0.\exists N\in\mathbb{N}^*\quad s.t.\ n>N$时,
    有$\displaystyle{\left|\sum\limits_{k=1}^nf_k(z)-f(z)\right|<\epsilon}.\forall z\in \gamma$.
    故$n>N$时有$\displaystyle{\left|\sum\limits_{k=1}^n\int_\gamma f_k(z)dz-\int_\gamma f(z)dz\right|
    =\left|\int_\gamma\left(\sum\limits_{k=1}^nf_k(z)-f(z)\right)dz\right|<\epsilon\cdot|\gamma|}$
\end{proof}

\begin{mypro}
    若级数$\displaystyle{\sum\limits_{n=1}^\infty f_a(z)}$在区域$\mathbb{D}$的任一紧子集$K$上一致收敛,
    则称$\displaystyle{\sum\limits_{n=1}^\infty f_a(z)}$在$\mathbb{D}$上是内闭一致收敛的.
\end{mypro}
\begin{proof}
    类似于实值函数中的例子,函数项级数$\displaystyle{1+\sum\limits_{k=1}^\infty f_k(z),\ f_k(z)=z^k-z^{k-1}}$
    部分和为$z^k$显然下单位球上内闭一致收敛,但不一致收敛
\end{proof}

\begin{mypro}[Weierstrass I]
    设$D$是$\mathbb{C}$中的域.若$f_n\in C(D),n=1,2\dots\ ,$并且$\displaystyle{\sum\limits_{n=1}^\infty f_n(z)}$在$D$中内闭一致收敛到$f(z)$
    则$f\in H(D)$,并且级数$\displaystyle{\sum\limits_{n=1}^\infty f_n^{(k)}(z)}$在上内闭一致收敛到$f^{(k)}(z).\quad k\in\mathbb{N}$.
\end{mypro}
\begin{proof}
    任取$z_0\in D$,由于$D$为开集,故存在$\delta>0,\quad s.t.\ \overline{B(z_0,\delta)}\subset D$\\
    由定理\emph{4.1.4},$f\in C(\overline{B(z_0,\delta)})$在$B(z_0,\delta)$中任做一个可求长闭曲线$r$,
    由定理\emph{4.1.5}和定理\emph{3.2.4}得$\displaystyle{\int_rf(z)dz=\sum\limits_{n=1}^\infty\int_rf_n(z)dz=0}$.
    故由\emph{Morera}定理得$f\in H(B(z_0,\delta))$\\
    故$f\in H(D)$\\
    任取$\xi\in\partial B(z_0,\delta) ,\forall z\in B(z_0,\frac{\delta}{2})$.
    有$\displaystyle{\left|\frac{1}{(\xi-z)^{k+1}}\right|\leqslant(\frac{2}{\delta})^{k+1}}$\\
    由一致收敛性得,$\forall\epsilon>0,\ \exists N\in\mathbb{N}^*,\quad s.t.\ \forall n>N,\forall\xi\in\partial B(z_0,\delta)$,有:
    \begin{align*}
        &\left|\sum\limits_{j=1}^nf_j(\xi)-f(\xi)\right|
        <\frac{\epsilon}{k!\cdot\delta}(\frac{\delta}{2})^{k+1}\\
        \Rightarrow&\sum_{j=1}^n\frac{f_j(\xi)}{(\xi-z)^{k+1}}-\frac{f(\xi)}{(\xi-z)^{k+1}}<\frac{\epsilon}{k!\cdot\delta}
    \end{align*}
    故当$z\in B(z_0,\frac{\delta}{2})$时,有:
    \begin{align*}
        \left|\sum\limits_{j=1}^nf_j^{(k)}(z)-f^{(k)}(z)\right|
        &=\frac{k!}{2\pi}\cdot\left|\sum\limits_{j=1}^n\int_{|\xi-z_0|=\delta}\frac{f_j(\xi)d\xi}{(\xi-z)^{k+1}}
        -\int_{|\xi-z_0|=\delta}\frac{f(\xi)d\xi}{(\xi-z)^{k+1}}\right|\\
        &\leqslant\frac{k!}{2\pi}\int_{|\xi-z_0|=\delta}\left|\sum_{j=1}^n\frac{f_j(\xi)}{(\xi-z)^{k+1}}
        -\frac{f(\xi)}{(\xi-z)^{k+1}}\right|d\xi\\
        &<\epsilon
    \end{align*}
    故$\displaystyle{\sum_{j=1}^nf_j^{(k)}(z)\Rightarrow f^{(k)}(z),z\in B(z_0,\frac{\delta}{2})}$,
    对于$D$的任一紧集$K,K$有有限开球覆盖$SI_kS_{k=1}^n$,故$\displaystyle{\sum\limits_{j=1}^nf_j^{(k)}\Rightarrow f^{(k)}(z),z\in K}$.
\end{proof}

\begin{mypro}[Weierstrass \uppercase\expandafter{\romannumeral2}]
    设$D$为有界区域,对于函数列$\{f_n(z)\}$.有$f_n(z)\in H(D)\bigcap C(\overline{D})$
    且级数$\displaystyle{\sum\limits_{n=1}^\infty f_n(z)}$在$\partial D$上一致收敛,
    则$\displaystyle{\sum\limits_{n=0}^\infty f_n(z)}$在$\overline{D}$上一致收敛.
\end{mypro}
\begin{proof}
    $\forall\epsilon>0,\ \exists N,\ s.t.\ n\geqslant N, p\geqslant 1$时,有$\left|f_{n+1}(z)+f_{n+2}(z)+\dots+f_{n+p}(z)\right|<\epsilon.$
    对任意$z\in\partial D$均成立,由最大模定理,上述不等式在$\overline{D}$上成立,故级数在$\overline{D}$上一致收敛
\end{proof}

\section{幂函数}
幂级数:$\displaystyle{\sum\limits_{n=0}^\infty} a_n(z-z_0)^n=a_0+a_1(z-z_0)+a_2(z-z_0)^2+\dots+a_n(z-z_0)^n+\dots$\\
其中$a_0,a_1\dots$均为复常数,做平移$\omega=z-z_0$,得$\displaystyle{\sum\limits_{n=0}^\infty z^n=a_0+a_1z+\dots+a_nz^n+\dots(*)}$.
\begin{mypro}
    (收敛半径)(收敛圆).用数分中定义,略
\end{mypro}

\begin{mypro}
    级数$(*)$存在收敛半径$R=(\overline{\lim\limits_{n\to\infty}}\sqrt[n]{|a_n|})^{-1}$,则:\\
    (1)当$R=0$时,$\displaystyle{\sum\limits_{n\equiv0}^\infty a_nz^n}$只在$z=0$处收敛.\\
    (2)当$R=+\infty$时,$\displaystyle{\sum\limits_{n\neq0}^\infty a_nz^n}$在$\mathbb{C}$上收敛.\\
    (3)当$0<R<+\infty$时,$\displaystyle{\sum\limits_{n=0}^\infty a_nz^n}$在$\{z:|z|<R\}$中收敛.在$\{z:|z|>R\}$中发散.
\end{mypro}
\begin{proof}
    同数学分析中的讨论,此处略去
\end{proof}

\begin{mypro}[\emph{Abel} I]
    如果$\displaystyle{\sum\limits_{n=0}^\infty a_n\cdot z^n}$在$z=z_0\neq0$处收敛,则在$\{z:|z|<z_0\}$中内闭一致收敛.
\end{mypro}
%\noindent\emph{证明.}\ 
\begin{proof}
	设$K$为$\{z:|z|<|z_0|\}$中的一个紧集,取$r<|z_0|$,使得$k\subset B(0,r)$
\begin{wrapfigure}[4]{r}{2cm}
    \centering
    \includegraphics[width=3cm,height=2.7cm]{ch4_p3.png}
\end{wrapfigure}
由于$\displaystyle{\sum\limits_{n=0}^\infty a_nz_0^n}$收敛,故$\left|a_nz_0^n\right|\rightarrow0$
故存在$M\in\mathbb{R}.\ s.t.\ \left|a_nz_0^n\right|<M$\\
故当$z\in k$时,
$\displaystyle{\left|a_k\cdot a^k\right|=\left|a_n\cdot z_o^k\cdot \frac{z^k}{z_0^k}\right|\leqslant M\left(\left|\frac{z}{z_0}\right|\right)^k}$\\
由于$\displaystyle{\sum\limits_{n=0}^\infty\left|\frac{z}{z_0}\right|^n}$时收敛的,
故由\emph{Weierstrass}判别法得$\displaystyle{\sum\limits_{n=0}^\infty a_nz^n}$在$k$中一致收敛。\
%\\rightline{$\square$}
\end{proof}

\noindent 由定理\emph{4.2.3}以及\emph{Weierstrass}定理可以得到:

\begin{mypro}
    幂级数在其收敛圆内确定了一个全纯函数\par
    \qquad\quad\, 那么在收敛圆圆周上的收敛性如何呢?
\end{mypro}

\begin{eg}
    级数$\displaystyle{\sum\limits_{n=0}^\infty z^n}$的收敛半径为1,它在收敛圆周$|z|=1$上处处发散.
\end{eg}

\begin{eg}
    级数$\displaystyle{\sum\limits_{n=0}^\infty \frac{z^n}{n^2}}$的收敛半径为1,它在收敛圆周$|z|=1$上处处收敛.
\end{eg}

\begin{eg}
    级数$\displaystyle{\sum\limits_{n=0}^\infty \frac{z^n}{n}}$的收敛半径为1,
    它在收敛圆周$|z|=1$上在$z=e^{i\theta}(0<\theta<2\pi)$处收敛,在1处发散.
\end{eg}

\begin{proof}
    显然$z=1$时级数时发散的\\
    当$\theta\neq0$时$\displaystyle{\sum\limits_{n=0}^\infty\frac{z^n}{n}=
    \sum\limits_{n=0}^\infty\frac{\cos n\theta}{n}+i\cdot\sum\limits_{n=0}^\infty \frac{\sin n\theta}{n}}$
    由实数项级数的\emph{Dirichlet}判别法,\\
    得$\displaystyle{\sum\limits_{n=0}^\infty\frac{\cos n\theta}{n},\ \sum\limits_{n=0}^\infty \frac{\sin n\theta}{n}}$收敛\\
\end{proof}
由上述例子可知,收敛圆周上的收敛性无法确定\par 这也是我们下面要探讨的问题\par
设级数$\displaystyle{\sum\limits_{n=0}^\infty a_n(z-z_0)^n}$的收敛半径为$R$,
我们来研究$\displaystyle{\xi\in B(z_0,R),\sum\limits_{n=0}^\infty a_n(\xi-z_0)^n}$与和函数$f$的关系:
令$\displaystyle{\omega=\frac{z-z_0}{\xi-z_0}}$
故$\omega\in B(0,1)$,级数可改写为$\displaystyle{\sum\limits_{n=0}^\infty b_n\omega^n},b_n=a_n(\xi-z_0)^n$,
故新的幂级数的收敛半径为1,故以下我们在收敛半径为1的情况下做讨论.

\begin{mypro}[*]
    设$g$是定义在单位圆中的函数,$e^{i\theta_0}$是单位圆上的一点,记$S_\alpha(e^{i\theta_0})$为以$e^{i\theta_0}$为顶点,
    以$e^{i\theta_0}$点对应的极径在单位圆内侧向两侧张出的张角为$2\alpha$
    的角形区域$(\alpha<\frac{\pi}{2})$
\end{mypro}

%排的有问题
\begin{wrapfigure}[2]{r}{3cm}
    \centering
    \includegraphics[width=3cm,height=2cm]{ch4_p4_1st.png}
\end{wrapfigure}
\noindent \emph{若$z$在$S_\alpha(e^{i\theta_0})$中趋于$e^{i\theta_0}$时$g(z)$有极限$C$则称$g$在$\theta_0$处有非切向极限$C$,记为
$\displaystyle{\lim\limits_{\substack{z\to e^{i\theta_0}\\z\in S_\alpha(e^{i\theta_0})}}g(z)=C}$}
\\
\\
\begin{mypro}[*](\rm \emph{Abel} \uppercase\expandafter{\romannumeral2})
    设$\displaystyle{f(z)=\sum\limits_{n=0}^\infty a_nz^n}$的收敛半径为$R=1$,
    且级数在$z=1$处收敛于$S$,则$f$在$z=1$处有非切向极限$S$
\end{mypro}
\noindent\emph{证明.}
记$\displaystyle{\sigma_{n,\rho}=\sum\limits_{i=1}^p a_{n+i}}$由条件得$\displaystyle{\sum\limits_{n=0}^\infty a_n}$收敛.\\
故$\forall\epsilon>0 ,\ \exists$正整数$N,\  s.t.\ $当$n>N$时,对任意自然数$p$有$|\sigma_{n,p}|<\epsilon$\\
由于
\begin{align*}
    \sum\limits_{i=1}^pa_{n+i}\cdot z^{n+i}&=\sum\limits_{i=1}^{p-1}(\sigma_{n,i+1}-\sigma_{n,i})z^{n+1+i}+\sigma_{n,1}z^{n+1}\\
    &=\sum\limits_{i+1}^{p-1}\sigma_{n,i}z^{n+i}\cdot(1-z)+\sigma_{n,p}z^{n+p}\\
    &=z^{n+1}(1-z)\cdot\sum\limits_{i=1}^{p-1}(\sigma_{n,i}\cdot z^{i-1})+\sigma_{n,p}\cdot z^{n+p}
\end{align*}
故当$|z|<1,n>N$.时有:$\displaystyle{\left|\sum\limits_{i=1}^pa_{n+i}\cdot z^{n+i}\right|
<\epsilon\cdot|1-z|\cdot\sum\limits_{n=p}^\infty|z^n|+\epsilon=\epsilon\left(\frac{|1-z|}{1-|z|}+1\right)}$\\
任取$\displaystyle{z\in S_\alpha(1)\cap B(1,\delta)}$记$|z|=r,|1-z|=\rho$则由余弦定理得$r^2=1+\rho-2\rho\cos\theta$\\
\begin{wrapfigure}[3]{r}{3cm}
    \centering
    \includegraphics[width=3.5cm,height=2.0cm]{ch4_p4_2nd.png}
\end{wrapfigure}
故$\displaystyle{\frac{|1-z|}{1-|z|}=\frac{\rho}{1-r}=\frac{\rho(1+r)}{1-r^2}
\leqslant\frac{2\rho}{2\rho\cos\theta-\rho^2}=\frac{2}{2\cos\theta-\rho}}$\\
又由于$z\in B(1,\delta).$故$\rho<\delta<\cos\alpha<\cos\theta$故$\displaystyle{\frac{|1-z|}{1-|z|}<\frac{2}{\cos\alpha}}$\\
故$\displaystyle{\left|\sum\limits_{i=1}^pa_{n+i}\cdot z^{n+i}\right|<\epsilon(\frac{2}{\cos\alpha}+1)}$
,故$\displaystyle{\sum\limits_{n=0}^\infty a_nz^n}$在$\displaystyle{S_\alpha(1)\cap B(1,\delta)}$上一致收敛
设和函数为$f$,由一致收敛性得$f$在$\displaystyle{S_\alpha(1)\cap B(1,\delta)}$上连续,
故$\displaystyle{\lim\limits_{\substack{z\in S_\alpha(1)\\z\to 1}}f(z)=f(1)=S}$\\\rightline{$\square$}

%p1-4






%以下是p.9-13
\begin{mypro}
	设$f \in f(\omega)$,$\omega$为开集,$k\in\omega$为紧集,则存在$\omega-k$中的有线条可求长$r_{1},r_{2}......r_{n}$,使得
	\begin{equation*}
	f(z)=\sum_{j=1}^{N}\frac{1}{2\pi i}\int_{r_{n}}\frac{f(\xi)}{\xi-z}\di{\xi}
	\end{equation*}
\end{mypro}
\begin{proof}
	设$d=c \cdot d(k,\omega^{c})$,其中$c$为常数且$0<c<\frac{1}{\sqrt{2}}$,考虑边平行于轴的边长为$d$的正方形组成的网格。
	故在$\Omega-K$中必定包含一个完整的方格,并且与$k$相交的方格不与$\Omega$相交。
	
	设$Q$为与$K$相交的方格的全集,由于$K$是紧集,故$Q$为有限集,设$Q={Q_{1},Q_{2}......Q_{n}}$
	我们取$r_{1},r_{2}......r_{n}$为$Q$中的方格中所有只属于唯一方格的方格边框。如图
	%@TODO:Here should be the figure%
	
	任取$K$中非方格边框的一点$z$,存在$1 \le j \le m , s.t.  z \in Q_{j},z\notin Q_{k} (k \ne j)$
	故由\emph{Cauchy}积分公式得
	
	\begin{equation*}
	f(z)=\int_{\partial Q_{k}}\frac{f(\xi)}{\xi - z}\di{\xi}=\delta_{n-j}\cdot f(z)
	\end{equation*}
	故
	\begin{equation*}
	f(z)=\sum_{k=1}^{m}\int_{\partial Q_{k}}\frac{f(\xi)}{\xi -z}\di{\xi}
	\end{equation*}
	
	而由于在对每个$Q_{k}$的边界进行线积分的时候,对于除去$r_{1},r_{2}......r_{n}$之外的$Q$中方格边框均正向反向各积分一次
	,故
	\begin{equation*}
	f(z)=\int_{r_{k}}\dfrac{f(\xi)}{\xi - z}\di{\xi}
	\end{equation*}
	而可得$r_{k} \in \Omega /k,k=1,2.....m$,显然$r_{1},r_{2}......r_{n}$首尾相连得到了一条$\Omega /k$中的可求长闭曲线        
\end{proof}

\begin{eg}
	\color{blue}设$G={z\in C,0<\arg z<\frac{\pi}{4}}$,设$f\in H(G)\cup C(\bar G)$。若$f$在实轴上区间$[a,b]$恒为$0$,则$f(x)\equiv 0,x\in G$
	
	\color{black}
	\begin{proof}
	在$[a,b]$下方做延拓(如图),
	%@TODO:A figure here%
	
	有$\partial G \cap \partial D=[a,b]$。做$f_{1}:D\cup [a,b] \to {0} ,f_{1}(z)\equiv 0,z\in D\cup [a,b]$。
	故$\forall z\in [a,b]$有$f_{1}(z)=f(z)=0$。记$G\cup D \cup [a,b]=M$,故由解析延拓定理得存在$F(z)\in H(M)$
	且$f_{1}(z)=F(z),z\in D\cup [a,b],F(z)=f(z),z\in G$,显然存在一个M中的点列${Z_{n}}$,且有极限点$a\in M^{o}$,
	并且$F(Z_{n})=0。\forall n\in N。$由零点孤立性定理得$f(z)\equiv 0,z\in M$,故$f(z)\equiv 0,z\in G$
	
	类似于上一例题的推理过程,对于区域G,设$f\in H(G)\cap C(\bar{G})$,若存在一条可求长连续曲线$r \in \partial G$,
	有$f(z)\equiv 0,z\in r$,并且可以在r的异于G一侧做一区域E,使得$r\in \partial E$,且$E\cap G=\phi$,完全与上一例题中的过程相同可得$f(z)\equiv 0,\forall z\in G$
	\end{proof}
\end{eg}
\begin{eg}
	\color{blue}设$f_{n}(z)\in H(D)$,D为中区域,$\sum_{n=1}^{\infty}|f_{n}(z)|$在$D$内一致收敛,证明:$\sum_{n=1}^{\infty}|f_{n}(z)|$在D内内闭一致收敛。
	
	\color{black}
	\begin{proof}
	任取$D$中的紧集$K$,设$d=d(\partial D,k)$,记$\rho=\frac{d}{2}$
	
	故$z\in k$时
	\begin{equation*}
	\sum\limits_{i=1}^{n}\left|f'_{n+i}(z)\right|=\sum\limits_{i=1}^{p}\left|\int_{|\xi -z|=\rho }\dfrac{f_{i}(\xi)\di{\xi}}{(\xi -z)^{2}}\right|\cdot \dfrac{1}{2\pi}
	\end{equation*}
	\begin{equation*}
	\le \int_{|\xi -z|=\rho }\frac{\sum_{i=1}^{p}|f_{i}(\xi)|}{|(\xi -z)^{2}|}|\di{\xi}|\cdot \dfrac{1}{2\pi}\le \dfrac{\epsilon}{\rho}
	\end{equation*}
	\end{proof}
\end{eg}
\begin{eg}
	\color{blue}写出$e^{\frac{1}{1-z}}$,在$|z|<1$和$1<z<+ \infty$的部分\emph{Laurent}展式
	
	\color{black}设$f(z)=e^{z}$,显然$f$为常    函数,故$f(z)=\sum_{n=0}^{\infty}\frac{z^{n}}{n!}$,可以利用4.3.7将R上的展开到 D上
	可得$f(\frac{1}{1-z})=\sum\limits_{n=0}^{\infty}\frac{1}{n!}(\frac{1}{1-z})^{n}$,当$|z|<1$时,有如下错误解法。
	\begin{jie}{(错误)}
		\begin{equation*}
		\frac{1}{1-z}=\sum_{n=0}^{\infty}z^{n}=1+z+z^{2}+o|z|^{2} 
		\end{equation*}
		\begin{equation*}
		\Rightarrow f(\frac{1}{1-z})=1+\frac{1}{1-z}+\frac{1}{2}(\frac{1}{1-z})^{2}+...\dots
		\end{equation*}
		做到这里你就发现做不下去了,为什么?因为$(\frac{1}{1-z})^{n}=1+o(|z|)$
	\end{jie}
	\begin{jie}{(正确)}
		考虑展开$e^{\frac{z}{1-z}}$
		\begin{equation*}
		\dfrac{z}{1-z}=\sum\limits_{n=1}^{\infty}z^{n}=z+z^{2}+o|z|^{2} =o(1)
		\end{equation*}
		\begin{equation*}
		\Rightarrow f(\dfrac{z}{1-z})=1+\dfrac{z}{1-z}+\dfrac{1}{2}(\dfrac{z}{1-z})^{2})+o|z|^{2}
		\end{equation*}
		\begin{equation*}
		=1+(z+z^{2}+o|z|^{2})+\frac{1}{2}(z+z^{2}+o|z|^{2})^{2}+o|z|^{2}
		\end{equation*}
		\begin{equation*}
		=1+z+\dfrac{3}{2}z^{2}+o|z|^{2}
		\end{equation*}
		\begin{equation*}
		\Rightarrow e^{\dfrac{1}{1-z}}=e+ez+\dfrac{3e}{2}z^{2}+o|z|^{2}
		\end{equation*}
	\end{jie}
	当$1<|z|<+\infty$时,同上述讨论。
\end{eg}
\begin{eg}
	\color{blue}若$f(z)$在$0<|z-a|<R$上解析,$f$不为常值函数,且圆环上有一列点$z_{n}\rightarrow a$,有$f(z_{n})=0$。证明:$a$为$f(z)$的本性奇点。
	
	\color{black}
	\begin{proof}
	若$a$为可去奇点,则$\displaystyle\lim_{z\rightarrow a}f(z)$存在,故$\displaystyle\lim_{z\rightarrow a}f(z)=0$,补充定义$f(a)=0$故$f$在$|z-a|<R$上解析。由零点孤立性定理得$f(z)\equiv 0 , 0\le |z-a|<R$矛盾,
	
	若$a$为极点,故$\displaystyle\lim_{z\rightarrow a}f(z)=+\infty$,而$\displaystyle\lim_{n\rightarrow \infty}f(z_{n})=0$矛盾!
	\end{proof}	
\end{eg}
\begin{eg}
	\color{blue}设$f(z)$在圆环$0<r<|z-a|<R<+\infty$内解析,在闭圆环$r\le |z-a|\le R$上连续,且$f(Re^{i\Theta} )=0,(0\le \theta \le 2\pi)$。证明:$f(z)\equiv 0,(r<|z-a|<R)$
	\color{black}
	\begin{jie}{1}
		
		
		设$F(z)=f(\frac{r^{2}}{\bar{z}}),z\in \{z|\frac{r^{2}}{R} \le |z-a|\le r\}$,显然,$F(z)$在$\{z|\frac{r^{2}}{R} < |z-a|< r\}$
		上解析,在$\{z|\frac{r^{2}}{R} \le |z-a|< r\}$上连续。并且当$|z|=r$时,$F(z)=f(\frac{r^{2}}{\bar{z}})=f(z)$。$|z|=\frac{r^{2}}{R}$时,$F(z)=0$.
		
		设$D=\{z|\frac{r^{2}}{R} \le |z-a| < r\}$,故由解析延拓定理得存在$F_{1}\in H(D)\cap C(\bar{D})$。并且$F_{1}(z)=F(z),z\in \{z|\frac{r^{2}}{R} \le |z-a|\le r\}$,且$F_{1}(z)=f(z),z\in \{z|\frac{r^{2}}{R} \le |z-a|\le r\}$。
		
		又由于$\partial D =\{ z:|z|=\frac{r^{2}}{R} $或$R\}$。$\Rightarrow \forall z \in \partial D$有$F_{1}(D)=0$故最大模存理$F_{1}(z)\equiv 0 ,z\in \bar{D}$故$f(z)\equiv 0.(r<|z-a|<R)$
	\end{jie}
	\begin{jie}{2}
		
		
		同4.6.5第一个例子
	\end{jie}
\end{eg}
\begin{eg}
	\color{blue}若$f\in H(\{ z:0<|z-a|<k\}  )$且$\displaystyle\lim_{z \rightarrow a}(z-a)f(z)=0$。证明:$a$是$f(z)$的可去奇点。
	
	\color{black}
	\begin{proof}
	显然$a$是$g(z)=(z-a)f(z)$的可去奇点。利用定理4.5.1,再直接讨论\emph{Laurent}级数的系数即可
	\end{proof}

\end{eg}
\begin{eg}
	\color{blue}若$f$为整函数,并且$\infty$为$f$的可去奇点,证明$f$为常值函数。
	
	\color{black}
	\begin{proof}
	由定理4.5.1得$f$在$\infty$处的\emph{Laurent}展开式中幂次大于0的系数均为0。
	又由于$f$为整函数,故$f(z)$为常数。
	\end{proof}
\end{eg}
\begin{eg}
	\color{blue}设$f\in H(B(a,R )|\{ a\})$,且不为常值函数。记$u(z)=Ref(z)$,若$u(z)$有界,则$a$是$f$的可去奇点。
	\color{black}
	\begin{jie}{1}
		
		
		由\emph{Laurent}级数系数公式得$1\le n$时,$G_{n}=\frac{1}{2\pi i}\int_{|z-a|=\rho}f(z)(z-a)^{n-1}\di{z} \quad(0<\rho<R)$
		
		另一方面
		\begin{align*}
		G_{n}&=\frac{1}{2\pi i}\int_{|z-a|=\rho}f(z)(z-a)^{n-1}\di{z} \quad(0<\rho<R)\\
		&=\frac{1}{2\pi i}\int_{|z-a|=\rho}(\sum_{k=-\infty}^{\infty }\overline{C_{k}}\cdot\overline{(z-a)^{k}})(z-a)^{n-1}\di{z} \quad(0<\rho<R)\\
		&=\frac{1}{2\pi i}\int_{|z-a|=\rho}\sum_{k=-\infty}^{\infty }\overline{C_{k}} \cdot \rho^{n+k-1}(\frac{z-a}{|z-a|})^{n-1}\di{z} \quad(0<\rho<R)\\
		&=\frac{1}{2\pi}\sum_{k=-\infty}^{+\infty}\overline{C_{k}}\int_{0}^{2\pi}\rho^{k+n}e^{i(n-k)\theta}\di{\theta}=\overline{C_{n}}\rho^{2n}
		\end{align*}
		
		故我们可得$C_{-n}+\overline{C_{n}}\rho^{2n}=\frac{1}{\pi i}\int_{|z-a|=\rho}u(z)(z-a)^{n-1}\di{z}$
		
		故$|C_{-n}+\overline{C_{n}}\rho^{2n}|\le \frac{1}{\pi}|\int_{|z-a|=\rho}u(z)(z-a)^{n-1}\di{z}|<\frac{2\pi \rho}{\pi}\cdot M \cdot \rho^{n-1}=2M\rho^{n}(1\le n)$
		
		当$\rho\rightarrow 0$。得$C_{-n}=0(1\le n)$。故$a$为$f(z)$可去奇点。
	\end{jie}
	\begin{jie}{2}
		
		
		设$F(z)=e^{f(z)}$,故$F(z)$在$B(a,R)\{ a\}$上有界。故$|F(z)=e^{u(z)}\le e^{M}|$。故由定理4.5.1得。$a$为$F(z)$的可去奇点,补充定义$F(a)$使得$F(z)\in H(B(a,R))$
		
		又有$|F(z)|=e^{u(z)}\ge e^{-M}>0$。故在$a$的邻域内对数函数可取出单值解析分支。故$a$是$f(z)=\lg F(z)$的可去奇点。
	\end{jie}
	\begin{jie}{3}
			
		记$F(z)=\frac{f(z)}{f(z)-2M}$。故在$B(a,R)\{a\}$上有$|F(z)|\le 1$
		故$a$是$F(z)$的可去奇点。补充定义$F(a)$使得$F(z)\in H(B(a,R))$。
		
		由最大模原理得$|F(a)|<1$。而可解出$f(z)=\frac{2MF(z)}{1-F(z)}$。故由表达式得$\displaystyle\lim_{z\rightarrow a}f(z)$存在。故$a$为$f(z)$的可去奇点。
	\end{jie}
\end{eg}



\end{document}