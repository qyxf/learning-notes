\chapter{可测函数}
\section{第一组}


\exercise{1} 设有指标集$I\ $,$\  f_{\alpha}(x):\alpha $是$\mathbb{R}^n$上的可测函数,试问:函数$S(x)=sup\left\{f_{\alpha}(x)\ |\ \alpha\in I\right\}$在$\mathbb{R}^n$上是可测的吗?
\begin{proof}
不一定\par 设$ I $是$ \mathbb{R}^n $上的不可测集,且记$W=\left\{x_{\alpha}|\alpha\in I\right\}$,设$f_{\alpha}(x)=\frac{1}{|x-x_{\alpha}|},\alpha\in I\ ,$故$f_{\alpha}(x)$是$\mathbb{R}^n$上的可测函数,故$\left\lbrace x\in \mathbb{R}^n |\ S(x)=+\infty\right\rbrace =\left\lbrace x\in \mathbb{R}^n |\ f_{\alpha}(x)=+\infty\right\rbrace =W$是不可测集,故$S(x)$是不可测含函数
\end{proof}


\exercise{2} 设$z=f(x,y)$是$\mathbb{R}^2$上的连续函数,$g_1(x),g_2(x)$是$[a,b]$上的实值可测函数,试证明$F(x)=f(g_1(x),g_2(x))$是$ [a,b]$上的可测函数
\begin{proof}
由简单函数逼近定理得,存在函数列$\{\phi_k(x)\},\{\psi_k(x)\}\ ,\ s.t.$ $$\li{k\to\infty}\phi_k(x)=g_1(x)\ ,\ \li{k\to\infty}\psi_k(x)=g_2(x)$$
故$f(\phi_k(x),\psi_k(x))$也是在$[a,b]$上的简单可测函数列,由于$z=f(x,y)$是$\mathbb{R}^2$上的连续函数
,故$\li{k\to\infty}f(\phi_k(x),\psi_k(x))=f(g_1(x),g_2(x)),$故得证
\end{proof}


\exercise{3} 设$f(x)$在$[a,b)$上存在右导数$f'_{+}(x)\ $,试证明$f'_{+}(x)\ $是$[a,b)$上的可测函数
\begin{proof}
故$f(x)$是右连续的,即$\forall\epsilon>0,\forall x\in\mathbb{R},\exists \delta>0\ ,\ s.t.\ |f(x+\delta)-f(x)|<\epsilon \ ,$故$\forall\epsilon>0,\forall x\in\mathbb{R},\exists \delta>0\ ,\ s.t.\ \forall x_1,x_2\in (x,x+\delta),$有$|f(x_1)-f(x_2)|<2\epsilon $,故$f(x)$在$(x,x+\delta)$上连续,故$f$的不可测点集是零测集,故$f(x)$可测,故$f(x+\frac{1}{n})$可测,又由于$\li{n\to\infty}n\left(f(x+\frac{1}{n})-f(x) \right)= f'_{+}(x)\ \ ,$故$f'_{+}(x)\ $可测
\end{proof}


\exercise{4}设$f(x)$是$E\subset \R{R}{n}$上几乎处处有限的可测函数,$m(E)<+\infty$,试证明对任意的$\epsilon>0$,存在E上的有界可测函数$g(x)$,使得$$m\left(\{x\in E:|f(x)-g(x)|>0\} \right)<\epsilon $$
\begin{proof}
设$E_k=\left\lbrace x\in E:|f(x)|>k\right\rbrace $,故$\{E_k\}$是递减集合列,且$$\li{k \to \infty}m(E_k)=m(\left\lbrace x\in E:|f(x)|=\infty \right\rbrace )=0,$$故$\forall\epsilon>0,$存在$k_0\in \R{N}{*},\ s.t.\ m(E_{k_0})<\epsilon$,
记\[g(x)=\begin{cases}
f(x)&x\in E\setminus E_{k_0}\\
0&x\in E_k\\
\end{cases}\]
显然$g(x)$可测且有界,且$m\left(\{x\in E:|f(x)-g(x)|>0\} \right)<\epsilon $
\end{proof}


\exercise{5}设$f(x)$以及$f_n(x)(n=1,2,...)$都是$A \subset \R{R}{\ }$上几乎处处有限的可测函数,对任给的$\epsilon>0$,存在$A$的可测子集$B:m(A\setminus B)<\epsilon$,使得$f_n(x)$在$B$上一致收敛于$f(x)$,试证明$f_n(x)$在$A$上几乎处处收敛于$f(x)$
\begin{proof}
	$\forall n\in\R{N}{\ }$存在$A$的可测子集$B_n,\ s.t.\ m(A\setminus B_n)<\frac{1}{2^n}$,使得$f_k(x)$在$B_n$上一致收敛于$f(x)$,设$B=\varlimsup\limits_{n \to +\infty}A\setminus B_n,$故$m(B)=0,$且$A\setminus B=\varliminf\limits_{k \to +\infty}B_k\ \subset\  B_n,\forall n \in \R{N}{\ }$,故$\{f_k(x)\}$的不收敛点集一定在$B$中,故$f_n(x)$在$A$上几乎处处收敛于$f(x)$
\end{proof}


\exercise{6}设$\{f_n(x)\}$是$E\subset \R{R}{n}$上的实值可测函数列,$m(E)<+\infty$,试证明$\li{j\to \infty}f_k(x)=0\ ,\ a.e.\ x\in E$的充分必要条件是:对任意的$\epsilon>0$有$$\li{j\to\infty}m\left( \left\lbrace x\in E: \sup_{k\geq j} \{\ |f_k(x)|\ \}\geq \epsilon \right\rbrace \right)=0 $$
\begin{proof}
	设$E_j(\epsilon)=\left\lbrace x\in E: \sup\limits_{k\geq j} \{\ |f_k(x)|\ \}\geq \epsilon \right\rbrace$,显然对于给定的$\epsilon$,$\{E_j(\epsilon)\}$是递增集合列\\
	$"\Rightarrow"$\par 
	假设结论不成立,则存在$\epsilon_0\ $,$\ $s.t.$$\li{j\to\infty}m\left( \left\lbrace x\in E: \sup\limits_{k\geq j} \{\ |f_k(x)|\ \}\geq \epsilon_0>0 \right\rbrace \right)=a>0 ,$$则当j充分大时有$m\left( \left\lbrace x\in E: \sup\limits_{k\geq j} \{\ |f_k(x)|\ \}\geq \epsilon \right\rbrace \right)>\displaystyle{\frac{a}{2}}>0$,故$\{f_n(x)\}$存在一个子列$\{f_{n_k}(x)\}$,使得$\forall k\in \R{N}{\ }$,有$m\left( \left\lbrace x\in E:  \{\ |f_{n_k}(x)|\ \}\geq \epsilon \right\rbrace \right)>0$,这与$\li{j\to \infty}f_k(x)=0\ ,\ a.e.\ x\in E$矛盾,故假设不成立\\
	$"\Leftarrow"$\par 类似于必要性的证明使用反证法可以得到,此处略去
\end{proof}
我们再给出一种不用反证法的做法,同样的我们只给出一个方向的证明,另一个方向大致同理
\begin{proof}
	由书中P11例8我们可知函数列$\{f_n(x)\}$的不收敛于0的点集为$$D=\bigcup_{k=1}^{\infty}\bigcap_{N=1}^{\infty}\bigcup_{n=N}^{\infty}\left\lbrace x:|f_n(x)|\geq\frac{1}{k} \right\rbrace ,$$\\
	$"\Rightarrow"$\\
	$\qquad$故由条件得$m(D)=0$,推出$\forall k\in \R{N}{\ },$有$m\left(\bigcap\limits_{N=1}^{\infty}\bigcup\limits_{n=N}^{\infty}\left\lbrace x:|f_n(x)|\geq\frac{1}{k} \right\rbrace \right)=0 $,\\即$m\left(\varlimsup\limits_{n \to +\infty}\left\lbrace x:|f_n(x)|\geq\frac{1}{k} \right\rbrace \right)=0,$故得证.
\end{proof}


\exercise{7}设$f(x),f_1(x),f_2(x),...,f_k(x),...$是$[a,b]$上几乎处处有限的可测函数,且有$\li{k\to\infty}f_k(x)=f(x)\ \ a.e.\ \ x\in [a,b]$,试证明存在$E_n\supset [a,b](n=1,2,...)$,使得$$m\left([a,b]\setminus \bigcup\limits_{n=1}^{\infty}E_n \right)=0,$$而$\{f_k(x)\}$在每个$E_n$上一致收敛于$f(x)$
\begin{proof}
	不难看出题目满足Egrov定理的条件,所以对于$\delta_n=\frac{1}{n}$,存在$E_n\subset[a,b]$,使得$m([a,b]\setminus E_n)<\frac{1}{n}$,$\{f_n(x)\}$在$E_n$上一致收敛于$f(x)$,因为$m([a,b]\setminus \bigcup\limits_{n=1}^{\infty}E_n)\leq m([a,b]\setminus E_n)<\frac{1}{n},\forall n \in \R{N}{\ }$,所以$m\left([a,b]\setminus \bigcup\limits_{n=1}^{\infty}E_n \right)=0$
\end{proof}


\exercise{8}设$\{f_k(x)\}$在$E$上依测度收敛于$f(x)$,$\{g_k(x)\}$在$E$上依测度收敛于$g(x)$,试证明$\{f_k(x)+g_k(x)\}$在$E$上依测度收敛于$f(x)+g(x)$
\begin{proof}
	具体证明过程此处忽略,具体可以仿照书中定理3.13的过程进行操作
	
\end{proof}

\exercise{9}设$m(E)<+\infty,f(x),f_1(x),f_2(x),...f_k(x),...$是$E$上几乎处处有限的可测函数,试证明$\{f_k(x)\}$在$E$上依测度收敛于$f(x)$的充分必要条件是$$\li{k\to+\infty}\inf\limits_{\alpha>0}\left\lbrace \alpha+m\left(\{x\in E:|f_k(x)-f(x)|>\alpha\} \right) \right\rbrace=0 $$
\begin{proof}
	\\$"\Rightarrow"$
	\[\begin{split}
	&\li{k\to+\infty}\inf\limits_{\alpha>0}\left\lbrace \alpha+m\left(\{x\in E:|f_k(x)-f(x)|>\alpha\} \right) \right\rbrace\\
	\leq&\ \inf\limits_{\alpha>0}\li{k\to+\infty}\left\lbrace \alpha+m\left(\{x\in E:|f_k(x)-f(x)|>\alpha\} \right) \right\rbrace\\
	=&	\inf\limits_{\alpha>0}\{\alpha+0\}\ =\ 0
	\end{split}\]
	$"\Leftarrow"$\par 
	记$b_k=\inf\limits_{\alpha>0}\left\lbrace \alpha+m\left(\{x\in E:|f_k(x)-f(x)|>\alpha\} \right) \right\rbrace,$则由条件知$b_k\to 0.$显然对于任意的$k\in\R{N}{\ }$,存在$\{a_k\},\ s.t.\ b_k\leq a_k+m\left(\{x\in E:|f_k(x)-f(x)|>a_k\} \right)< b_k+\displaystyle{\frac{1}{k}}$,令$k\to +\infty$,则$a_k\to 0$,则$\li{k\to\infty}m\left(\{x\in E:|f_k(x)-f(x)|>a_k\} \right)=0$,故对于任意的$\epsilon>0$,有$\li{k\to\infty}m\left(\{x\in E:|f_k(x)-f(x)|>\epsilon\} \right)=0$,得证
	%故存在趋向于0的正数列$\{a_n\}$使得$\li{n\to\infty}m\left(\{x\in E:|f_k(x)-f(x)|>a_n\} \right)\to 0$
\end{proof}



\exercise{10}设$f_n(x)(n=1,2,...)$是$[0,1]$上的递增函数,且$\{f_n(x)\}$在$[0,1]$上依测度收敛于$f(x)$,试证明在$f(x)$的连续点$x_0$上,有$f_n(x_0)\to f(x_0)\quad(n\to\infty)$\par
 此题利用反证法是不难得到的,但这种方法便失去了分析的意义,下面我们给出利用所学定理直接进行证明的方法:
\begin{proof}
	由于$x_0$是$f(x)$的连续点,故$\forall \epsilon>0,\exists\delta>0,\ s.t.\ \forall x\in(x-\delta,x+\delta),$有$\displaystyle{|f(x)-f(x_0)|<\frac{\epsilon}{18}}.\ $\par 由Riesz定理,$f_n(x)$有子列$f_{n_k}(x)$在$[0,1]$上几乎处处收敛于$f(x)$.\par 由Egrov定理,存在$[0,1]$的子集$E_\delta$,满足$\displaystyle{m(E_\delta)<\frac{\delta}{2}},\ s.t.\ \{f_{n_k}(x)\}$在$[0,1]\setminus E_\delta$上一致收敛于$f(x).$\par 则当k充分大时有$\displaystyle {m\left(x\in [0,1]:|f_{n_k}(x)-f(x)|>\frac{\epsilon}{9}  \right)<\frac{\delta}{2}} ,$故存在$x_1\in(x_0-\delta,x_0),x_2\in(x_0,x_0+\delta),\ s.t.\ \displaystyle{ |f_{n_k}(x_1)-f(x_1)|<\frac{\epsilon}{9}},\displaystyle{ |f_{n_k}(x_2)-f(x_2)|<\frac{\epsilon}{9}}.$ 
	所以我们得到$\displaystyle{|f_{n_k}(x_1)-f_{n_k}(x_2)|\leq |f_{n_k}(x_1)-f(x_1)|+|f(x_1)-f(x_2)|+|f_{n_k}(x_2)-f(x_2)|< \frac{\epsilon}{3}}.$\par 又由于$f_{n_k}(x)$是递增函数,则$\displaystyle{|f_{n_k}(x_1)-f_{n_k}(x_0)|\leq|f_{n_k}(x_1)-f_{n_k}(x_2)|< \frac{\epsilon}{3}},$故$|f_{n_k}(x_0)-f(x_0)|\leq |f_{n_k}(x_1)-f_{n_k}(x_0)|+|f_{n_k}(x_1)-f(x_1)|+|f(x_1)-f(x_0)|<\epsilon,$所以$\{f_{n_k}(x)\}$在$x_0$处收敛于$f(x_0),$.\par 
	另一方面,由于$\displaystyle{\left\lbrace x\in[0,1]:|f_n(x)-f_{n_k}(x)|>\epsilon\right\rbrace \subset \left\lbrace x\in [0,1]:|f(x)-f_{n_k}(x)|>\frac{\epsilon}{2}  \right\rbrace  }$\\ $\cup
	\displaystyle{ \left\lbrace x\in [0,1]:|f(x)-f_n(x)|>\frac{\epsilon}{2}  \right\rbrace}.$
	而当k充分大时,$\displaystyle{x_0\notin\left\lbrace x\in [0,1]:|f(x)-f_{n_k}(x)|>\frac{\epsilon}{2}  \right\rbrace,}$故当n与k均充分大时,有$|f_n(x_0)-f_{n_k}(x_0)|\leq\epsilon,$故当n充分大时$|f_n(x_0)-f(x_0)|<2\epsilon$,得证
\end{proof}


\exercise{11}设$f:\R{R}{n}\to \R{R}{\ }$,且对任意的$\epsilon>0$,存在$\R{R}{n}$中的开集$G$,$m(G)<\epsilon,$使得$f\in C(\R{R}{n}\setminus G)$,试证明$f(x)$是$\R{R}{n}$上的可测函数
\begin{proof}
	略
\end{proof}


\exercise{12}设$\{f_k(x)\}$与$\{g_k(x)\}$在$E$上依测度收敛于0,试证明$\{f_k(x)g_k(x)\}$在$E$上依测度收敛于0
\begin{proof}
	具体证明过程此处忽略,具体可以仿照上面第八题的过程进行操作。(提示:类似于第8题加法的情况将$\epsilon$拆为两个$\epsilon$/2的和,那么对于此题乘法的情况呢?)
\end{proof}

%\exercise{13}设$\{f_k(x)\}$在$[a,b]$上依测度收敛于$f(x)$,$\{g(x)\}$是$\R{R}{\ }$上的连续函数,试证明$\{g(f_k(x))\}$在$[a,b]$上依测度收敛于$g(f(x))$.若讲$[a,b]$换为$[0,+\infty)$,结论还成立吗?


%\exercise{14}设$f(x)$是定义在可测集$E\subset\R{R}{n}$上的函数,且对任意的$\delta>0$,存在$E$中的闭集$F$,$m(E\setminus F)<\delta$,使得$f(x)$在$F$上连续,试证明$f(x)$是$E$上的可测函数


\exercise{15}设$\{f_n(x)\}$是$[a,b]$上的可测函数列,$f(x)$是$[a,b]$上的实值函数,若对任意的$\epsilon>0$都有$$\li{n\to\infty}m^{*}\left(\{ x\in[a,b]:|f_n(x)-f(x)|>\epsilon\} \right)=0 ,$$试问$f(x)$是E上的可测函数吗?
\begin{proof}
	$\forall k\in\R{N}{\ },$存在数列$\displaystyle{\{n_k\},\ s.t.\ m^{*}(E_k)<\frac{1}{2^{n_k}}}$,其中$E_k=\displaystyle{\{ x\in[a,b]:|f_{n_k}(x)-f(x)|>\frac{1}{2^n}\}}$,根据题目条件,$m^{*}\left( \varlimsup\limits_{k\to\infty}E_k\right)=0 $.\par 
	故$\ a.e.\ x\in [a,b], $有$x \in \varliminf\limits_{k\to \infty}{E_k}^{c},$即$\ a.e.\ x\in [a,b]$,存在正整数$K_x,\ s.t.\ $当$k>K_x$时,有$x\in {E_n}^{c}$.故当$k>K_x$时,有$|f_{n_k}(x)-f(x)|\leq \frac{1}{2^k},$故$f_{n_k}(x)$在$[a,b]$上几乎处处收敛于$f(x)$,故$f(x)$可测
\end{proof}

\exercise{16}设$f(x),f_k(x),(k=1,2,3...)$是$E\subset\R{R}{\ }$上的实值可测函数,若对任给的$\epsilon>0$,必有$$\li{j\to\infty}m\left( \bigcup_{k=j}^{\infty}\left\lbrace x:|f_k(x)-f(x)|>\epsilon\right\rbrace \right)=0,$$试证明对任给的$\delta>0$,存在$e\subset E$且$m(e)<\delta$,使得$\{f_k(x)\}$在$E\setminus e$上一致收敛于$f(x)$
\begin{proof}
	完全类似于定理3.12的证明,此处略去
\end{proof}